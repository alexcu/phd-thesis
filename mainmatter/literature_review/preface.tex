\label{sec:lit-review:preface}

In \cref{ch:introduction}, we defined a common set of intelligence-based cloud services that we label `\glsplx{cis}'. Specifically, we scope the primary body of this study's work on computer vision \glspl{cis} (e.g., Google Cloud Vision \citep{GoogleCloud:Home}, AWS Rekognition \citep{AWS:Home}, Azure Computer Vision \citep{Azure:Home}, Watson Visual Recognition \citep{IBM:Home} etc.) We presented the claim that developers have a distinctly deterministic mindset ($2+2$ \textit{always}  equals 4) whereas a \gls{cis}'s `intelligence' component (a black box) may return probabilistic results ($2+2$ \textit{might} equal 4 \textit{with a confidence of} 95\%). Thus, there is a mindset mismatch between probabilistic results (from the \gls{api} provider) and results which are certain (from the \gls{api} consumer).

What impact does this mindset mismatch have on software quality in the applications built using a \gls{cis}? What assurances are needed for developers to know that this mindset mismatch exists, and what can we learn from common software engineering practices (e.g., \citep{Pressman:2005vf,Sommerville:2011uc})? Throughout this chapter, we will review the core principles of this mindset mismatch from the anchoring perspective of software quality, particularly around \gls{vv}.