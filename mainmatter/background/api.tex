\section{Application Programming Interfaces}
\label{sec:background:api}

\Glsplx{api} are the interface between a developer needs and the software components at their disposal \citep{Arnold:2005vc} by abstracting the underlying component behind a subroutine, protocol or specific tool. Therefore, it is natural to assess internal quality (and external quality if the software is in itself a service to be used by other developers---in this case an \gls{iws}) is therefore directly related to the quality the \gls{api} offers \citep{Ko:2004td}. 

Good \glspl{api} are known to be intuitive and require less documentation browsing \citep{Piccioni:2013em}, thereby increasing developer productivity. Conversely, poor APIs are those that are hard to interpret, thereby reducing developer productivity and product quality. The consequences of this have shown a higher demand of technical support (as measured in \citep{Henning:2009hz}) that, ultimately, causes the maintenance to be far more expensive, a phenomenon widely known in software engineering economics (see \cref{ssec:background:software-quality:v-and-v}).

While there are different types of \glspl{api}, such as software library/framework \glspl{api} for building desktop software, operating system \glspl{api} for interacting with the operating system, remote \glspl{api} for communication of varying technologies through common protocols, we focus on web \glspl{api} for communication of resources over the web (being the common architecture of cloud-based services). Further information on the development, usage and documentation of web \glspl{api} is provided in \cref{ssec:background:api:usage}.

% TODO: Moved API usage to appendix

\subsection{API Usability}
\label{ssec:background:api:usability}

If a developer doesn't understand the overarching concepts of the context behind the \gls{api} they wish to use, then they cannot formulate what gaps in their knowledge is missing. For example, a developer that knows nothing about \gls{ml} techniques in \gls{cv} cannot effectively formulate queries to help bridge those gaps in their understanding to figure out more about the \gls{cvs} they wish to use. 

Balancing the understanding of the information need (both conscious and unconscious), how to phrase that need and how to query it in an information retrieval system is concept long studied in the information sciences \citep{Taylor:1968tq}. In \gls{api} design, the most common form to convey knowledge to developers is through annotated code examples and overviews to a platform's architectural and design decisions \citep{Myers:2011bt,Robillard:2011uv,Dorn:2010wl,Brandt:2009tm} though these studies have not effectively communicated \textit{why} these artefacts are important. What makes the developer \textit{conceptually understand} these artefacts?

\citet{Robillard:2011uv} conducted a multi-phase, mixed-method approach to create knowledge grounded in the professional experience of 440 software engineers at Microsoft of varying experience to determine what makes \glspl{api} hard to learn, the results of which previously published in an earlier report \citep{Robillard:2009uk}. Their results demonstrate that `documentation-related obstacles' are the biggest hurdle in learning new \glspl{api}. One of these implications are the \textit{intent documentation} of an \gls{api} (i.e., \textit{what is the intent for using a particular \gls{api}?}) and such documentation is required only where correct \gls{api} usage is not self-evident, where advanced uses of the \gls{api} are documented (but not the intent), and where performance aspects of the \gls{api} impact the application developed using it. They conclude that professional developers do not struggle with learning the \textit{mechanics} of the \gls{api}, but in the \textit{understanding} of how the \gls{api} fits in upwards to its problem domain and downward to its implementation:

\begin{quote}
  \itshape
  In the \textup{upwards} direction, the study found that developers need help mapping desired scenarios in the problem domain to the content of the API, and in understanding what scenarios or usage patterns the API provider intends and does not intend to support. In the \textup{downwards} direction, developers want to understand how the API's implementation consumes resources, reports errors and has side effects. 
  \upshape
  \citep{Robillard:2011uv}
\end{quote}


These results particularly corroborate to that of previous studies where developers quote that they feel that existing learning content currently focuses on ``\textit{how} to do things, not necessarily \textit{why}''~\citep{Nykaza:2002td}. This thereby reiterates the conceptual understanding of an \gls{api} as paramount.

A later study by \citet{Ko:2011fb} assessed the importance of a programmer's conceptual understanding of the background behind the task before implementing the task itself, a notion that we find most relevant for users of \gls{iws} \glspl{api}. While the study did not focus on developing web \glspl{api} (rather implementing a Bluetooth application using platform-agnostic terminology), the study demonstrated how developers show little confidence in their own metacognitive judgements to understand and assess the feasibility of the intent of the \gls{api} and understand the vocabulary and concepts within the domain (i.e., wireless connectivity). This indecision over what search results were relevant in their searches ultimately hindered their progress implementing the functionality, again decreasing productivity. \citeauthor{Ko:2011fb} suggest to improve API usability by introducing the background of the API and its relevant concepts using glossaries linked to tutorials to each of the major concepts, and then relate it back to how to implement the particular functionality.

Thus, an analysis of the conceptual understanding of \gls{iws} \glspl{api} by a range of developers (from beginner to professional) is critical to best understand any differences between existing studies and those that are nondeterministic. Our proposal is to perform similar survey research (see \cref{ch:research-methodology}) in the search for further insight into the developer's approach toward existing \gls{iws} \glspl{api}.
