\label{sec:research-strategy:preface}

Investigating software engineering practices is oft a complex task as it is imperative to understand the social and cognitive processes around software engineers and not just the tools and processes used \citep{Easterbrook:2007ws}. This chapter explores the research design utilised in this study by exploring six key elements of empirical software engineering research: firstly, we provide an extended focus to the study by reviewing our research questions (see \cref{sec:introduction:hypohtesis}) anchored under the context of an existing classification taxonomy \c, characterise our research goals through an explicit philosophical stance, explain how the stance selected impacts our selection of research methods and data collection techniques, discuss a set of criteria for assessing the validity of our study design and the findings of our research, discuss practical considerations of our methods, and lastly use explain a theory to review our data and relate it other studies in literature and our research questions. The foundations for developing this research strategy is expanded from that proposed by \citet{Easterbrook:2007ws}.