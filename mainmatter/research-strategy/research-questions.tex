\section{Research Questions Revisited}
\label{sec:research-strategy:research-questions}

In \cref{sec:introduction:hypohtesis}, we introduce three hypothesis of this study (RH1--3), namely: (i) existing \gls{cis} \glspl{api} are poorly documented for general use (RH1); (ii) existing \gls{cis} \glspl{api} do not provide sufficient metadata when used in context-specific use cases (RH2); and (iii) the combination of improving documentation and metadata will ultimately improve one of software quality, developer productivity and/or developer understanding (RH3).

To discuss our research strategy, we revisit our research questions through the classification technique discussed by \citet{Easterbrook:2007ws}, a technique originally proposed in the field of psychology by \citet{Meltzoff:1998wg} but adapted to software engineering. Our research study involves a mix of five \textit{knowledge questions}, that focus on existing practices and the ways in which they work, and two \textit{design questions}, that that focuses on designing better ways to approach software engineering tasks \citep{Simon:1996uw}. Both classes of questions are respectively concerned with empirical and non-empirical software engineering that, in practice, are best combined in long-term software engineering research studies (such as this one) as they assist in tackling the investigation of a specific problem, approaches to solve that problem and finding what solutions work best \citep{Wieringa:2006vd}.

\subsection{Knowledge Questions}
\label{ssec:research-strategy:research-questions:knowledge-questions}

In total, five knowledge questions are posed in this study to help us understand the way developers currently interact and work with a \gls{cis} \gls{api}; two exploratory, one base-rate, and two relationship and causality questions.

We begin by formulating two \textit{exploratory questions} to attempt to better understand the phenomena of poor API documentation and metadata; both \ref{rqs:apidoc:what-is-in-use} and \ref{rqs:metadata:what-problems-du                                                                e-to-lack-of-metadata} respectively describe and classify what practices are in use for existing \gls{cis} \gls{api} documentation and what problems currently exist when no metadata is returned. Answering these two questions assists in refining preciser terms of the phenomena, ways in which we find evidence for them and ensuring the data found is valid.

By answering these questions, we have a clearer understanding of the phenomena; we then follow up by posing an additional \textit{base-rate question} that helps provide a basis to confirm that the phenomena occurring is normal (or unusual) behaviour by investigating the patterns of phenomena's occurrence. \ref{rqs:apidoc:how-do-devs-understand-it} is a descriptive-process question to help us understand how the developer currently understands existing \gls{cis} \gls{api} documentation, given their lack of formal extended training in artificial intelligence. This achieves us an insight into the developer's mindset and regular thought patterns toward these \glspl{api}.
%Lack of formal training = exlusion critiera

Lastly, we investigate the relationship between the improved documentation and improvements to other aspects of the software development process. Chiefly, \ref{rqs:implications:do-metrics-improve} is concerned with whether any improvements to metadata or documentation correlate to improvements in software quality, developer productivity, or developer education (and is a \textit{relationship establishment question}). If we establish such a relationship, we refine the question and investigate the specific causes using three \textit{causality questions} defined under \ref{rqs:implications:aspects}, namely by associating three classes of measurable metrics (internal quality metrics, external quality metrics, developer education insight metrics) to the improved documentation.

\subsection{Design Questions}

\ref{rqs:apidoc:what-additional-information-needed} and \ref{rqs:metadata:what-metadata-do-devs-want-and-why} are both \textit{design questions}; they are concerned with ways in which we can improve a \gls{cis} by investigating what additional attributes are needed in both the documentation and metadata that assist developers to achieve their goals.  They are not classified as knowledge questions as we investigate what \textit{will be} and not \textit{what is}. By understanding the process by which developers desire additional attributes of metadata and documentation, we can help shape improvements to the existing design of a \gls{cis}. 