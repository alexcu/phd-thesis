\section{Philosophical Stances}
\label{sec:research-strategy:philosophical-stances}

\citet{Creswell:2017vn} characterise four dominant philosophical stances that help us constitute what is valid knowledge: positivism, constructivism, pragmatism and critical theory. To construct such a `validity of truth', we will review these four philosophical stances in this section, and state the stance that we explicitly adopt and our reasoning for this.

Positivists claim truth to be all observable facts, reduced piece-by-piece to smaller components which is incrementally verifiable to form truth. We do not base our work on the positivistic stance as the theories governing verifiable hypothesis must be precise from the start of the research. Moreover, due to its reductionist approach, it is quite difficult to isolate these hypotheses and study them in isolation from context. As our hypotheses are not context-agnostic, we steer clear from this stance.

Constructivists see knowledge embedded within the human context; truth is the \textit{interpretive} observation by understanding the differences in human thought between meaning and action \tocite{Klein and Myers 1999}. That is, the interpretation of the theory is just as important to the empirical observation itself. We patricianly adopt a constructivist stance as we attempt to model the developer's mindset, being an approach that is rich in qualitative data on human activity.

Pragmatism is a less dogmatic approach that encourages the incomplete and approximate nature of knowledge and is dependent on the methods in which the knowledge was extracted. The utility of consensually agreed knowledge is the key outcome, and is therefore relative to those who seek utility in the knowledge---what is the useful for one person is not so for the other. While we value the utility of knowledge, it is difficult to obtain consensus especially on an ill-researched topic such as ours, and therefore we do not adopt this stance.

This study, therefore, chiefly adopts the philosophy of critical theory \tocite{Calhoun:1995}. A key outcome of the study is to shift the developer's restrictive deterministic mindset and shed light on developing a new framework actively with the developer community that seeks to improve the process of using such \glspl{api}. In software engineering, critical theory is used to ``actively [seek] to challenge existing perceptions about software practice'' \citep{Easterbrook:2007ws}, and this study utilises such an approach to shift the mindset of \gls{cis} consumers and providers alike on how the documentation and metadata should not be written with the `traditional' deterministic mindset at heart. Thus, our key philosophical approach is critical theory to seek out \textit{what-can-be} using  partial constructivism to model the current \textit{what-is}.
