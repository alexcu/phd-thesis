

\def\cn{}
\def\cy{\checkmark}

\begin{table*}[hbt]
  \caption{An overview of the 5 dimensions and categories (sub-dimensions) within our proposed taxonomy.}
  \label{tab:taxonomy}
  \begin{tabular}{|rp{0.58\linewidth}||p{0.2\linewidth}|c|}
    \hline

    \textbf{Key} &
    \textbf{Description: Dimensions A=\dima{}; B=\dimb{}; C=\dimc{}; D=\dimd{}; E=\dime{}} &
    \textbf{Primary Studies } &
    \textbf{Total (\%)} \\

    \hline
    \hline
    [A1]&
    Quick-start guides to rapidly get started using the API in a specific programming language.
    &
    [S4, S9, S10] &
    3/21 (14\%)\\

    \hline
    \textbf{[A2]}&
    \textbf{Low-level reference manual documenting all API components to review fine-grade detail.}
    &
    \textbf{[S1, S3, S4, S8, S9, S10, S11, S12, S15, S16, S17]} &
    \textbf{11/21 (52\%)}\\

    \hline
    [A3]&
    Explanations of the API's high-level architecture to better understand intent and context.
    &
    [S1, S2, S4, S11, S14, S16, S19, S20] &
    8/21 (38\%)\\

    \hline
    [A4]&
    Source code implementation and code comments (where applicable) to understand the API author's mindset.
    &
    [S1, S4, S7, S12, S13, S17, S20] &
    7/21 (33\%)\\

    \hline
    \textbf{[A5]}&
    \textbf{Code snippets (with comments) of no more than 30 LoC to understand a basic component functionality within the API.}
    &
    \textbf{[S1, S2, S4, S5, S6, S7, S9, S10, S11, S14, S15, S16, S18, S20, S21]} &
    \textbf{15/21 (71\%)}\\

    \hline
    \textbf{[A6]}&
    \textbf{Step-by-step tutorials, with screenshots to understand  how to build a non-trivial piece of functionality with multiple components of the API.}
    &
    \textbf{[S1, S2, S4, S5, S7, S9, S10, S15, S16, S18, S20, S21]} &
    \textbf{12/21 (57\%)}\\

    \hline
    [A7]&
    Downloadable source code of production-ready applications that use the API to understand implementation in a large-scale solution.
    &
    [S1, S2, S5, S9, S15] &
    5/21 (24\%)\\

    \hline
    [A8]&
    Best-practices of implementation to assist with debugging and efficient use of the API.
    &
    [S1, S2, S4, S5, S7, S8, S9, S14] &
    8/21 (38\%)\\

    \hline
    [A9]&
    An exhaustive list of all major components that exist within the API.
    &
    [S4, S16, S19] &
    3/21 (14\%)\\

    \hline
    [A10]&
    Minimum system requirements and dependencies to use the API.
    &
    [S4, S7, S13, S17, S19] &
    5/21 (24\%)\\

    \hline
    [A11]&
    Instructions to install or begin using the API and details on its release cycle and updating it.
    &
    [S4, S7, S8, S9, S11, S13, S16, S19] &
    8/21 (38\%)\\

    \hline
    [A12]&
    Error definitions that describe how to address a specific problem.
    &
    [S1, S2, S4, S5, S9, S11, S13] &
    7/21 (33\%)\\

    \hline
    \hline
    \textbf{[B1]}&
    \textbf{A brief description of the purpose or overview of the API as a low barrier to entry.}
    &
    \textbf{[S1, S2, S4, S5, S6, S8, S10, S11, S15, S16]} &
    \textbf{10/21 (48\%)}\\

    \hline
    [B2]&
    Descriptions of the types of applications the API can develop.
    &
    [S2, S4, S9, S11, S15, S18] &
    6/21 (29\%)\\

    \hline
    [B3]&
    Descriptions of the types of users who should use the API.
    &
    [S4, S9] &
    2/21 (10\%)\\

    \hline
    [B4]&
    Descriptions of the types of users who will use the product the API creates.
    &
    [S4] &
    1/21 (5\%)\\

    \hline
    [B5]&
    Success stories about the API used in production.
    &
    [S4] &
    1/21 (5\%)\\

    \hline
    [B6]&
    Documentation to compare similar APIs within the context to this API.
    &
    [S2, S6, S13, S18] &
    4/21 (19\%)\\

    \hline
    [B7]&
    Limitations on what the API can and cannot provide.
    &
    [S4, S5, S8, S9, S14, S16] &
    6/21 (29\%)\\

    \hline
    \hline
    [C1]&
    Descriptions of the relationship between API components and domain concepts.
    &
    [S3, S10] &
    2/21 (10\%)\\

    \hline
    [C2]&
    Definitions of domain-terminology and concepts, with synonyms if applicable.
    &
    [S2, S3, S4, S6, S7, S10, S14, S16] &
    8/21 (38\%)\\

    \hline
    [C3]&
    Generalised documentation for non-technical audiences regarding the API and its domain.
    &
    [S4, S8, S16] &
    3/21 (14\%)\\

    \hline
    \hline
    [D1]&
    A list of FAQs.
    &
    [S4, S7] &
    2/21 (10\%)\\

    \hline
    [D2]&
    Troubleshooting suggestions.
    &
    [S4, S8] &
    2/21 (10\%)\\

    \hline
    [D3]&
    Diagrammatically representing API components using visual architectural representations.
    &
    [S6, S13, S20] &
    3/21 (14\%)\\

    \hline
    [D4]&
    Contact information for technical support.
    &
    [S4, S8, S19] &
    3/21 (14\%)\\

    \hline
    [D5]&
    A printed/printable resource for assistance.
    &
    [S4, S6, S7, S9, S16] &
    5/21 (24\%)\\

    \hline
    [D6]&
    Licensing information.
    &
    [S7] &
    1/21 (5\%)\\

    \hline
    \hline
    [E1]&
    Searchable knowledge base.
    &
    [S3, S4, S6, S10, S14, S17, S18] &
    7/21 (33\%)\\

    \hline
    [E2]&
    Context-specific discussion forum.
    &
    [S4, S10, S11] &
    3/21 (14\%)\\

    \hline
    [E3]&
    Quick-links to other relevant documentation frequently viewed by developers.
    &
    [S6, S16, S20] &
    3/21 (14\%)\\

    \hline
    [E4]&
    Structured navigational style (e.g., breadcrumbs).
    &
    [S6, S10, S20] &
    3/21 (14\%)\\

    \hline
    [E5]&
    Visualised map of navigational paths to certain API components in the website.
    &
    [S6, S14, S20] &
    3/21 (14\%)\\

    \hline
    \textbf{[E6]}&
    \textbf{Consistent look and feel of documentation.}
    &
    \textbf{[S1, S2, S3, S5, S6, S8, S10, S15, S20]} &
    \textbf{9/21 (43\%)}\\
    \hline
  \end{tabular}
\end{table*}