\chapter{Introduction}
\label{ch:introduction}
\graphicspath{{mainmatter/introduction/figures/}}

Abstraction layers are the application developer's productivity powerhouse as developers need not continuously consider underlying mechanics. The ubiquitous \glsx{api} enables separation of concerns and reusable component interaction; for example, complex graphics rendering and image manipulation is all achievable via a half-dozen lines of code with appropriate libraries and frameworks, for example OpenCV's \gls{api}.

\Gls{ml}, too, is being abstracted and offered behind \glspl{api}. The 2010s have shown an explosion of cloud-based services providing \textit{web} \glspl{api} typically marketed under an \gls{ai} banner. The \gls{ml} algorithms, data processing pipelines, and infrastructure bringing these techniques to life are also abstracted behind \glspl{api} calls, driven by the motivation to make it easier for developers to blend \gls{ai} into their software.
There is an explosion of interest from application developers (see \cref{fig:introduction:stackoverflow-trends}) that are investigating and exploring how best to infuse recent advances in AI into their software systems. Combined with an ever-increasing buffet of \gls{ai}-based solutions, technologies and products (see \cref{tab:introduction:ai-products}) for developers to choose from, it is evident that we are at the cusp of a new generation of `\gls{ai}-first' software.

Application developers build procedural and functional applications, where code typically evaluates deterministically to produce outcomes. Such software does not rely on probabilistic behaviour, unlike \gls{ai}-first software where, often, \gls{ml} techniques are employed. However, application developers, accustomed to such traditional software engineering paradigms, may not be aware of potential side-effects of those probabilistic techniques. Software that leverages recent advances in \gls{ai}---more specifically data-driven \gls{ml} techniques---will often have a layer of rules that wrap the \gls{ml} components.
%These rule-driven systems typically consume, utilise, and integrate libraries and frameworks, \glsacpl{ide} and other tooling, and cloud-based services such as \gls{aws} \citep{AWS:Home}.
\Gls{ai}-first software is, however, not \textit{solely} procedural-driven and combines large datasets with rules to produce outcomes. Therefore, they are both \textit{data-driven} and procedural-driven. The consequence is that large datasets---that train \gls{ml} models---combined with the algorithmic techniques behind these models result in probabilistic behaviour. Further, since these models can continually learn from \textit{new} data with time, existing probabilistic behaviour can evolve and thus regression testing techniques need to be adjusted as well for new data. 

\afterpage{\begin{landscape}
\begin{table}
\centering
\caption[Categorisation of AI-based products and services]{A broad range of \gls{ai}-based vendors, products, and services is emerging in recent years.  (Adapted from \citep{LoGiudice:2016wf}.)}
\label{tab:introduction:ai-products}
\tablefitlandscape{0.9}{
\begin{tabular}{p{0.2\linewidth}|p{0.25\linewidth}p{0.4\linewidth}}
\toprule
\textbf{Category} &
\textbf{Sample Vendors \& Products} &
\textbf{Typical Use Cases} \\
\midrule

\textbf{Embedded \gls{ai}:}\newline
Expert assistants leverage AI technology embedded in platforms and solutions. &

Amazon: \textit{Alexa}\newline
Apple: \textit{Siri}\newline
Facebook: \textit{Messenger}\newline
Google: \textit{Google Assistant}\newline
Microsoft: \textit{Cortana}\newline
Salesforce: \textit{MetaMind}&

Personal assistants for search, simple inquiry, and growing as expert assistance (composed problems, not just search).\newline 
Available on mobile platforms, devices, the internet of things, and as bots or agents.\newline
Used in voice, image recognition, and various levels of natural language processing sophistication.\bigskip\\

\textbf{\Gls{ai} point solutions:}\newline
Point solutions provide specialised capabilities for natural language processing, vision, speech, and reasoning. &

24[7]: \textit{24[7]}\newline
Admantx: \textit{Admantx}\newline
Affectiva: \textit{Affdex}\newline
Assist: \textit{AssistDigital}\newline
Automated Insights: \textit{Wordsmith}\newline
Beyond Verbal: \textit{Beyond Verbal}\newline
Expert System: \textit{Cogito}\newline
HPE: \textit{Haven OnDemand}\newline
IBM: \textit{Watson Analytics}\newline
Narrative Science: \textit{Quill}\newline
Nuance: \textit{Dragon}\newline
Salesforce: \textit{MetaMind}\newline
Wise.io: \textit{Wise Support}\bigskip&

Semantic text, facial/visual recognition, voice intonation, intelligent narratives.\newline
Various levels of natural language processing, from brief text messaging, chat/conversational messaging, full complex text understanding.\newline
Machine learning, predictive analytics, text analytics/mining, knowledge management and search.\newline
Used as expert advisors, reasoning tools, or in customer service.\\

\textbf{AI platforms:}\newline
Platforms that offer various AI tech, including (deep) machine learning, as tools, \glspl{api}, or services to build solutions.&

CognitiveScale: \textit{Engage, Amplify}\newline
Digital Reasoning: \textit{Synthesys}\newline
Google: \textit{Google Cloud \gls{ml}}\newline
IBM: \textit{Watson Knowledge Studio}\newline
Intel: \textit{Saffron Natural Intelligence}\newline
IPsoft: \textit{Amelia, Apollo, IP Center}\newline
Microsoft: \textit{Cortana Intelligence Suite}\newline
Nuance: \textit{360 platform}\newline
Salesforce: \textit{Einstein}\newline
Wipro: \textit{Holmes}&

\glspl{api}, cloud services, on-premises for developers to build AI solutions. Insights/advice building and rule-based reasoning. Vertical domain advisors (e.g., fraud detection in banking, financial advisors, healthcare). Cognitive services and bots.\\

\bottomrule

\end{tabular}}
\end{table}\end{landscape}}

\begin{figure}[t!]
\centering
\includegraphics[width=.95\linewidth]{stackoverflow-trends2}
\caption[Increasing interest in the developer community of computer vision services]{Increasing interest within the developer community for \glsplx{cvs} is shown via Stack Overflow posts. These trends of \gls{cvs} usage were measured as discussion of posts tagged with the relevant product name.\footnote{`Microsoft Cognitive Services' merges posts tagged with this name, that of `Cortana Intelligence', and `Azure Cognitive Services'.} This graph is based on data from \cref{ch:icse2020}.}
\label{fig:introduction:stackoverflow-trends}
\end{figure}


Developing \gls{ai}-infused applications requires both code \textit{and data}, and an application developer can approach developing from three perspectives, further expanded in \cref{sec:introduction:context}:
\begin{enumerate}
  \item The application developer defines an \gls{ml} model from scratch and trains it from a curated dataset. This approach is laborious in time and demands experience and knowledge of \gls{ml} methods, but the tradeoff is that they have full autonomy in the models they create.
%   The application developer uses a pre-trained ML model (e.g. YOLO or XYZ). This approach removes the time taken to collect data, design and train the ML model; the developers, still need to know where to find these models, evaluate them, and then learn the frameworks within which they operate to use them effectively.
  \item The application developer downloads a pre-trained model (e.g., YOLO \citep{8100173} for computer vision, or GPT-2 \citep{Radford2019} for natural language processing) and `plugs' it into an existing \gls{ml} framework, such as Tensorflow \citep{Abadi:2016vn} or PyTorch \citep{NIPS2019_9015}. This approach removes the time taken to collect data, design and train the \gls{ml} model; the developers, still need to know where to find these models, evaluate them, and then learn the frameworks\footnote{Thus introducing a verbose list of \gls{ml} terminology to her developer vocabulary. See a list of 328 terms provided by Google here: \url{https://developers.google.com/machine-learning/glossary/}. Last accessed 7 December 2018.} within which they operate to use them effectively.
  \item The application developer uses a cloud-based service. It is fast to integrate into their applications, and the \glspl{api} offered abstract the technical know-how behind a web call.
\end{enumerate}
While much research has investigated these first two perspectives (see \cref{ch:background}), the third is yet to be deeply explored, despite the fact that vendors are promoting new offerings encapsulated under this third perspective. As shown in \cref{tab:introduction:ai-products}, vendors are rapidly pushing out new \gls{ml}-based offerings in the form of cloud-based \glspl{api} end-points (\gls{ai} platforms), where the \gls{api} abstraction masks away the underlying mechanics of the models. Developers that use these cloud-based services are presented with documentation providing a narrative (i.e., marketing and in the documentation) that implies integration of these services are just like other cloud services. But does this implication, coupled with abstractions that hide the assumptions made by the \gls{ai}-service providers, lead to developer pain-points and miscomprehension?
If so, how can the service providers improve their documentation to alleviate this?
Do these data-driven services share similarities to the runtime behaviour of traditional cloud services?
And if not, how best can the application developer integrate the data-driven service into their a procedural-driven application to produce \gls{ai}-first software?

% A \citeyear{LoGiudice:2016wf} report by market research company Forrester captured such growth into four key areas \citep{LoGiudice:2016wf}


\begin{table}[p]
\centering
\caption[Differing characteristics of cloud services]{Differing characteristics of intelligent and typical cloud services.}
\label{tab:introduction:characteristics-of-cloud}
\begin{tabular}{@{}ll@{}}
\toprule
  % Heading
  \textbf{Intelligent Cloud Services} &
  \textbf{Typical Cloud Services}
  \\
  \midrule
  Probabilistic &
  Deterministic 
  \\
  Machine Learnt &
  Human Engineered
  \\
  Data-Driven &
  Rule-Driven
  \\
  Black-Box &
  Mostly Transparent
  \\
  \bottomrule
\end{tabular}
\end{table}

\begin{figure}
\centering
\includegraphics[width=0.95\linewidth]{rule-vs-data}
\caption[Differences between data- and procedural-driven cloud services]{The application developer's procedural-driven toolchain is distinct from data-driven toolchain. A developer must consume a typical, data-driven cloud service in a different way than an intelligent data-driven cloud service as they are not the same type of system.}
\label{fig:introduction:rule-vs-data}
\end{figure}

\Cref{fig:introduction:rule-vs-data} provides an illustrative overview between the context clashing of procedural-driven applications and data-driven cloud services, and we contrast characteristics of typical cloud systems and data-driven ones in \cref{tab:introduction:characteristics-of-cloud}.

\afterpage{\begin{callout}
In this thesis, we show that (i) developers do not properly understand the probabilistic data-driven machine-learnt behaviour abstracted behind the end-points, (ii) the `intelligent behaviour' is not fully contained and leaks into the applications that make use of these end-points, and finally (iii) we present how these concerns can be addressed via better documentation and software architecture. that the integration and developer comprehension of cloud services differ from the procedural-driven nature of end-applications.
% Something more precise and punchy here - e.g. intelligent component abstractions are not fully contained and they leak - developers need to be aware of these and we offer specific insights into doc, integration etc. // or We offer X, Y, Z to resolve challenges developers face when using .. Cloud based AI services
% We show how `intelligent' component abstractions are not fully contained and leak into client applications, which developers must be aware of. We offer specific insights into the documentation and integration challenges of such components, namely ways to better improve service documentation and strategies to address leakage issues.
\end{callout}}

\section{Research Context}
\label{sec:introduction:context}

There are a range of integration techniques available to developers, as reflected by Google AI's\footnote{
Google AI was recently rebranded from Google Research, further highlighting how the `\gls{ai}-first' philosophy is increasingly becoming embedded in companies' product lines and research and development teams. Spearheaded through work achieved at Google, Microsoft and Facebook, the emphasis on an \gls{ai}-first attitude we see through Google's 2018 rebranding of \textit{Google Research} to \textit{Google AI} \citep{Howard:2018tz} is evident. A further example includes how Facebook leverage \gls{ai} \textit{at scale} within their infrastructure and platforms \citep{Parekh:2017hx}.
} \textit{\glslong{ml} spectrum} \citep{Ortiz:2017wg,LaForge:2018tm,McGowen:2019vt}. This range is grouped into the three tiers aforementioned, encompassing skills, effort, users, and types of outputs of integration techniques. At one extreme, this approach involves the academic research of developing algorithms and self-sourcing data to achieve intelligence---coined as \gls{byoml} \citep{Ortiz:2017wg,McGowen:2019vt,Jimerson:2017vh}. The other extreme involves off-the-shelf, `friendlier' (abstracted) intelligence with easy-to-use \glspl{api} targeted towards applications developers. The middle-ground involves a mix of the two, with varying levels of automation to assist in development, that turns custom datasets into machine intelligence. 
We illustrate the slightly varied characteristics within this spectrum in \cref{tab:introduction:comparison-of-ml-spectrum} and \cref{fig:introduction:cv-spectrum}.

\def\checkmark{y}
\begin{table}[p]
\centering
\caption[Comparison of the machine learning spectrum]{Comparison of the machine learning spectrum.}
\label{tab:introduction:comparison-of-ml-spectrum}
\begin{tabular}{@{}l|ccccc@{}}
\toprule
  % Heading
  \textbf{Comparator} &
  \textbf{\glsac{byoml}} &
  \textbf{\glsac{ml} F'work} &
  \textbf{Cloud \glsac{ml}} &
  \textbf{Auto-Cloud \glsac{ml}} &
  \textbf{Cloud \glsac{api}} 
  \\ 
  \midrule
  \thinrule
  % Hosting
  \textbf{Hosting} & & & & & \\
  \thinrule
    Locally & \checkmark & \checkmark &  &  &  \\
    Cloud &  &  & \checkmark & \checkmark & \checkmark \\
  \midrule
  \thinrule
  % Output Type
  \textbf{Output} &  &  &  &  &  \\
  \thinrule
    Custom Model & \checkmark & \checkmark & \checkmark & \checkmark &  \\
    \glsac{http} Response &  &  &  &  & \checkmark \\
  \midrule
  \thinrule
  % Autonomy
  \textbf{Autonomy} &  &  &  &  &  \\
  \thinrule
    Low &  &  &  &  & \checkmark \\
    Medium &  &  &  & \checkmark &  \\
    High &  & \checkmark & \checkmark &  &  \\ 
    Highest & \checkmark &  &  &  &  \\
  \midrule
  \thinrule
  % TTM
  \textbf{Time To Market} &  &  &  &  &  \\
  \thinrule
    Medium & \checkmark & \checkmark &  &  &  \\
    High &  &  & \checkmark & \checkmark &  \\
    Highest &  &  &  &  & \checkmark \\
  \midrule
  \thinrule
  % Data Source
  \textbf{Data} &  &  &  &  &  \\ 
  \thinrule
    Self-Sourced & \checkmark & \checkmark & \checkmark & \checkmark &  \\
    Pre-Trained &  & \checkmark &  &  & \checkmark \\
  \midrule
  \thinrule
  % Intended User
  \textbf{Intended User} &  &  &  &  &  \\
  \thinrule  
    Academics & \checkmark & \checkmark &  &  &  \\
    Data Scientist & \checkmark & \checkmark & \checkmark & \checkmark &  \\
    Developers &  &  &  & \checkmark & \checkmark \\
  \bottomrule
\end{tabular}
\end{table}
\begin{figure}[p]
\centering
\includegraphics[width=\linewidth]{cv-spectrum}
\caption[The spectrum of machine learning]{Examples within the \gls{ml} spectrum of computer vision. Colour scales indicates the benefits (green) and drawbacks (red) of each end of the spectrum.}
\label{fig:introduction:cv-spectrum}
\end{figure}

These cloud AI-services are gaining traction within developer circles: we show an increasing trend of \glslong{so} posts mentioning intelligent computer vision services in \cref{fig:introduction:stackoverflow-trends}.\footnote{Query run on 12 October 2018 using StackExchange Data Explorer. Refer to \url{https://data.stackexchange.com/stackoverflow/query/910188} for full query.}
Academia provides varied nomenclature for these services, such as \textit{Cognitive Applications} and \textit{Machine Learning Services} \citep{Hwang:2017tr} or \textit{Machine Learning as a Service} \citep{Ribeiro:2015dz}. 
For the context of this thesis, we will refer to such services under broader term of \textbf{\glspl{iws}},\footnote{This term is an extension inspired by the term `web service', as defined by the World Wide Web Consortium. See \url{https://bit.ly/2CQWJ2Z}, last accessed 19 July 2020.} and diagrammatically express their usage within \cref{fig:introduction:cloud-intelliegnce-service}.

\begin{figure}[h!]
\centering
\includegraphics[width=0.9\linewidth]{cloud-intelliegnce-service}
\caption[Overview of intelligent web services]{Overview of \glsplx{iws}.}
\label{fig:introduction:cloud-intelliegnce-service}
\end{figure}

There are many types of \glspl{iws} available to software developers, offering a range of functions, such as optical character recognition, text-to-speech and speech-to-text transcription, object categorisation, facial analysis and recognition, and natural language processing. The general workflow of using an \gls{iws} is more-or-less the same: a developer accesses an \gls{iws} component via \glsac{rest}/\glsac{soap} \gls{api}(s), which is (typically) available as a cloud-based \gls{paas}.\footnote{We note, however, that a development team may use a similar approach \textit{internally} within a product line or service that may not necessarily reflect a \gls{paas} model.}\footnote{A number of services provide the platform infrastructure to rapidly begin training from custom datasets, such as Google's AutoML (\url{https://cloud.google.com/automl/}, last accessed 7 December 2018). Others provide pre-trained datasets `ready-for-use' in production without the need to train data.} Developers send a given request to analyse a specific piece of data (e.g., an image, body of text, audio file etc.) and receive some intelligence on the data (e.g., object detection, text sentiment, transcription of audio) in addition to an associated \textit{confidence} value that represents the likelihood of that result. This is typically serialised as a \glsac{json}/\glsac{xml} response object. 

%We note the intelligence component masks its `intelligence' through a black-box: in recent years, there is a rise in providing human-level intelligence via crowdsourcing Internet marketplaces such as Amazon Mechanical Turk~\citepweb{MTurk:Home} or ScaleAPI~\citepweb{ScaleAPI:Home}. Thus, an \gls{iws} may be powered by varying degrees of intelligence: human intelligence, machine learning, data mining or intelligence by brute-force.

\begin{callout}
Within this thesis, we scope our investigation to a mature \textup{subset} of \glspl{iws} that provide computer vision intelligence~\citepweb{GoogleCloud:Home,Azure:Home,AWS:Home,Pixlab:Home,IBM:Home,Cloudsight:Home,Clarifai:Home,DeepAI:Home,Imagaa:Home,Talkwaler:Home,Kairos:Home,Cognitec:Home,Affectiva:Home}. For the context of this thesis, we will refer to such services as \textbf{\glspl{cvs}}. 
\end{callout}

\section{Motivation: Current Developer Mindsets'}
\label{sec:introduction:motivation}

\cref{fig:introduction:stackoverflow-trends} shows an increasing trend to the adoption and discussion of \glspl{cis} with developers. As aforementioned, these services are accessible through \glspl{api} and consist of an `intelligence' black box (\cref{fig:introduction:cloud-intelliegnce-service}). When a term `black box' is used, the input (or stimulus) is transformed to its to outputs (or response) without any understanding of the internal architecture by which this transformation occurs; indeed, this well-understood theory arose from the electronic sciences and since adapted to wider applications since the 1950s--60s \citep{Ashby:1957db,Bunge:1963jm} to describe ``systems whose internal mechanisms are not fully open to inspection'' \citep{Ashby:1957db}. 

In the world of machine learning and data mining, where we develop algorithms to make predictions in our datasets or discover patterns within them, these black boxes are inherently probabilistic and stochastic; there is little room for certainty in these results as such insight is purely statistical and associational \citep{Pearl:2018uv} against its training dataset. As an example, a computer vision \gls{cis} returns the \textit{probability} that a particular object (the response) exists in the raw pixels (the stimulus), and thus for a more certain (though not fully certain) distribution of overall confidence returned from the service, a developer must treat the problem stochastically by testing this case hundreds if not thousands of times to find a richer interpretation of the inference made. Developers (at present) do not need to treat their programs in any such stochastic way as traditionally their mindset is that computers will always make certain outcomes. But in the day and age of stochastic and probabilistic systems, this mindset needs to shift.

There are thus therefore three key factors to consider when implementing, testing and developing with a \gls{cis}: (i) the \gls{api} usability, (ii) the nature of stochastic and probabilistic systems, and (iii) how both impact on software quality.

% TODO: Copied from ml inconsistency paper
\subsection{The Impact on Software Quality}
\label{ssec:introduction:motivation:impact}

Do traditional techniques for documenting deterministic \glspl{api} also apply to non-deterministic systems? As \glspl{api} reflect a set of design choices made by their providers intended for use by the developer, does the mindset between the machine learning architect and the novice programmer match? Evaluations of \gls{api} usability advocate for the accuracy, consistency and completeness of \glspl{api} and their documentation \citep{Piccioni:2013em,Robillard:2009uk} written by providers, while providers should consider mismatches between the developer's conceptual knowledge of the \gls{api} its implementation \citep{Ko:2011fb}. However, consistency cannot be guaranteed in probabilistic systems, and the conceptual knowledge of such systems are still treated like black boxes. It is therefore imperative that \gls{cis} providers consider the impact of their \gls{api} usability; if not, poor \gls{api} usability hinders on the internal quality of development practices, slowing developers down to produce the software they need to create.

Moreover, \gls{cis} \glspl{api} are inherently non-deterministic in nature, but developers are still taught with the deterministic mindset that all \gls{api} calls are the same. Simple arithmetic representations (e.g., $2+2=4$) will \textit{always} result in 4; but a multi-layer perceptron neural network performing similar arithmetic representation \citep{Blake:1998vd} gives the probability where the target output (\textit{exactly} 4) and the output inferred (\textit{possibly} 4) matches as a percentage (or as an error where it does not match). That is, instead of an exact output, there is instead a \textit{probabilistic} result: $2+2$ \textit{may} equal 4 with a confidence of $n$. External quality must therefore be considered in the outcome of these systems, such as in the case of thresholding values, to consider whether or not the inference has a high enough confidence to justify its result to end-users.

In order to fully understand this problem, there are multiple dimensions one must consider: the impact of software quality; the fact that these systems underneath are probabilistic and are stochastic; the cognitive biases of determinism in developers; the issue of consistency in \gls{api} usage. While existing literature does extensively explore software quality and \gls{api} usability, these studies have only had emphasis on deterministic systems and thus little work to date has investigated such factors on probabilistic systems that make up the core of computer vision \glspl{cis}. We explore more of these facets in the motivating scenario below.

% TODO: Copied from ml inconsistency paper
\subsection{Motivating Scenario}
\label{ssec:introduction:motivation:scenario}

  How do developers work with a \gls{cis}? How usable are these \glspl{api}, and how well do developers understand the non-deterministic and stochastic nature of a deep-learning cloud-based \gls{api}? To motivate such a scenario, let us introduce a fictional software developer named Pam.

Pam wants to develop a social media photo-sharing mobile app that analyses her and her friends photos. Pam wants the app to categorise photos into scenes (e.g., day vs. night, landscape vs. indoors), generate brief descriptions of each photo, and catalogue photos of her friends as well as common objects (e.g., all photos with her Border Collie dog, all photos taken on a beach on a sunny day).

Rather than building a computer vision engine from scratch, which would take far too much time and effort, Pam thinks she can achieve this using one of the common computer vision \glspl{cis}. Pam comes from a typical software engineering background and has insufficient knowledge of key computer vision terminology and no understanding of the processes behind deep-learning. She ultimately believes all are \glspl{api} alike and internalises a deterministic mindset of them; when she decides on one of the three \glspl{api}, she expects a static result always. As she expects the same for whenever she calls, for example, any substring \gls{api} with the call (or similar) of \texttt{substring("doggy", 0, 2)} and would expect the response \texttt{`dog'} as its output.

To make an assessment of these \glspl{api}, she tries her best to read through the documentation of some computer vision \glspl{api}, but she has no guiding framework to help her choose the right one. Some of the questions that may come to mind include:

\begin{itemize}
  \item What does confidence mean? Aren't these APIs consistent?
  \item Will she need a combination of many computer vision \glspl{api} to solve this task?
  \item How does she know when there is a defect in the response? How can she report it?
  \item How does she know what labels the \gls{api} can pick up, and what labels it can't?
  \item How does she know when the models update? What is the release cycle?
  \item How does it describe her photos and detect the faces?
  \item How can she interpret the results if she disagrees with it to help improve her app?
\end{itemize}

Dazzled by this, she does some brief reading on Wikipedia but is confused by the immense technical detail to take in. She would like some form of guiding framework to assist her and in software engineering terms she can understand.
\section{Research Outcomes}
\label{sec:introduction:hypohtesis}

\todo{AC: ***Rewrote all of \cref{sec:introduction:hypohtesis} for review; \today.***}\\
\itshape
In this thesis, we explore the probabilistic ripple-effect with relation to the development usability of `intelligent' \glspl{api}; specifically, we contextualise within computer vision \glspl{cis}. Our anchoring perspective is software quality---specifically, validation and verification---within such systems and what best practices within the field of software engineering can be applied to assist in operationalisation such systems.
\upshape

The goals of this study aim to provide a snapshot of current developer best practices towards the usage of \glspl{cis} to provide a guiding framework and recommendations for software developers and \glspl{cis} providers alike. Based on the motivating case studies in \cref{sec:introduction:motivation}, we articulate three Research Hypotheses (RH1--3) below and eight Research Questions (RQs) based on  both empirical and non-empirical software engineering methodology \citep{Shull:2007vh,Simon:1996uw}.

\begin{titled-frame}{\underline{RH1}: \textit{Existing \glspl{cis} present insufficient \gls{api} documentation for general use.} }
\vspace{-12pt}
\paragraph{Research Hypothesis}
\gls{api} documentation of intelligent services are inadequate and insufficient given the disparity of mindsets between the software engineer and data scientist. Chiefly, software engineers---all with varied experience of using AI-based development tooling, if any---may have very limited general understanding of the `magic' that occurs behind these probabilistic `intelligent' \glspl{api}. We do not know what key aspects of the documentation matter to them, nor what they do or do not understand of the existing documentation.

\paragraph{Research Goal}
To improve the documentation of existing \gls{cis} providers, specifically of computer vision \glspl{api}.

\paragraph{Research Questions}
\begin{enumerate}[label=\textbf{RQ1.\arabic*.}, ref=RQ1.\arabic*, leftmargin=3.5\parindent, rightmargin=1\parindent]
  \item What practices are in use for intelligent services' \gls{api} documentation? 
  % KQ: Exploratory question to understand phenomena: Description and Classification question.
  \label{rqs:apidoc:what-is-in-use}
  
  \item How do developers currently understand and interpret the documentation given a lack of formal training in artificial intelligence? That is, what do they understand and not understand, and what key aspects of the \gls{api} documentation matter do developers as they see it?
  % KQ: Base-rate question to understand normal patterns of the occurrence of the phenomena: Descriptive-Process questions.
  \label{rqs:apidoc:how-do-devs-understand-it}
  
  \item What additional information or attributes need to be included in the \gls{api} documentation?
  % DQ: Design question
  \label{rqs:apidoc:what-additional-information-needed}
\end{enumerate}

\paragraph{Research Contribution} An intelligent service \gls{api} documentation quality assessment framework to evaluate how well the service has been documented for software engineers to use.

\paragraph{Research Method}

Problem identification and discovery to validate our hypothesis in the \textit{general context} of \gls{api} usage will begin with a background to help inform what prior works have been done in the \gls{api} documentation space. We will follow this with repository and question mining, i.e., searching on developer communities such as Quora, Stack Overflow, and GitHub Issues to find out what developers complain about and mine this knowledge into a framework.

We then will conduct an internal pre-controlled survey within our research group (we refer to as the `pilot' survey study) and will use findings from the background and mining to help inform us of the kinds of questions to ask. 

Findings from the pilot survey to help inform a wider structured survey and unstructured interview, where we will recruit external software engineers in industry through contacts of our research group. A quantitative (survey) and qualitative (interview) analysis will help begin to shape our research outcome of an API documentation quality assessment framework and help stabilise a general understanding of how developers use the existing \glspl{api}.
\end{titled-frame}

\begin{titled-frame}{\underline{RH2}: \textit{Existing \glspl{cis} present insufficient metadata for context-specificity.} }
\vspace{-12pt}
\paragraph{Research Hypothesis}
Intelligent service \glspl{api} respond with insufficient information for developers to operationalise the service into a business-driven application and, thus, additional metadata is needed to assist developers. Such metadata is likely to be added to the response objects of the \gls{api}.

\paragraph{Research Goal}
To improve the quality of \textit{context-specific response data} from the \gls{api} endpoints of intelligent services.

\paragraph{Research Questions}
\begin{enumerate}[label=\textbf{RQ2.\arabic*.}, ref=RQ2.\arabic*, leftmargin=3.5\parindent, rightmargin=1\parindent]
  \item What are current problems due to lack of return metadata?
  % KQ; Exploratory question to understand phenomena: Description and Classification question
  \label{rqs:metadata:what-problems-due-to-lack-of-metadata}
  
  
  \item What kind of metadata do developers want? Why do they want this metadata?
  % KQ; Base-rate question to understand normal patterns of the occurrence of the phenomena: Descriptive-Process questions.
  \label{rqs:metadata:what-metadata-do-devs-want-and-why}
  
  \item Does additional metadata assist developers in developing applications that use intelligent services of varying contexts, and if so, how?
  % KQ; relationship question to understand the correlation between two phenomena
  \label{rqs:metadata:how-does-metadata-assist-devs}
\end{enumerate}

\paragraph{Research Contributions} A list of metadata key-value-pairs that assist developers in using these \glspl{api} during the development of software that consume these services. In essence, improvements to the framework of Research Outcome 1: ``\textit{An intelligent service \gls{api} documentation \underline{\upshape and metadata} quality assessment framework}''.

\paragraph{Research Method} To confirm findings of the method within RH1 is genuine, we shift from reviewing the documentation from a general stance to a specialised (context-specific) stance in the use of these \glspl{api}.

Thus, we will use context-specific action research to develop basic `prototypes' of varying contexts to help identify where any potential gaps are in the findings of RH1.
To validate the findings of developer opinion in the surveys and interviews of RH1 are indeed genuine, this helps ensure that there is nothing missing by adding in further context to such opinions.
\end{titled-frame}

\begin{titled-frame}{\underline{RH3}: \textit{RH1 and RH2 improve quality,  productivity or developer informativeness.} }
\vspace{-12pt}
\paragraph{Research Hypothesis}
The implication of hypotheses 1 and 2 suggest that improving both the documentation and providing further metadata will improve product quality (internal or external), and/or developer productivity and/or developer education in developing software with intelligent components.

\paragraph{Research Goal}
 To confirm if improvements to \gls{api} documentation and response metadata  are reflected as improvements to product quality, developer productivity and/or developer education.
 
 \paragraph{Research Questions}
\begin{enumerate}[label=\textbf{RQ3.\arabic*.}, ref=RQ3.\arabic*, leftmargin=3.5\parindent, rightmargin=1\parindent]
  \item  What metrics are improved when the intelligent service documentation or metadata is improved?
  % KQ; relationship question to understand the correlation between two phenomena
  \label{rqs:implications:what-metrics-improve}
  
  \item With respect to \ref{rqs:implications:what-metrics-improve}, the three aspects are explored:
  % KQ; relationship question to understand the correlation between two phenomena
  \begin{enumerate}
  \item Are improvements reflected in product quality (i.e., improve avoiding common pitfalls; external quality)?
  \item Are improvements reflected in developer productivity (e.g., faster, better, fewer bugs; internal quality)?
  \item Are improvements reflected as a subjective `feel-good' factor for the developer (e.g., is the developer better informed or more confidence in what they do)?
  \end{enumerate}
  \label{rqs:implications:aspects}  
\end{enumerate}

\paragraph{Research Contribution}
A concrete sample solution or framework that improves such metrics, thereby confirming that our documentation and metadata quality assessment framework improves these facets.

\paragraph{Research Method}

To confirm that the framework is valid, we will provide a fictitious \gls{api} that is documented with the additional metadata and information organised using our framework.

We then ask 20 developers to complete five tasks under an observational, comparative controlled study, 10 of which will (a) develop with the new framework, and the other 10 will (b) develop with the as-is/existing documentation. From this, we compare if the framework makes improvements by capturing metrics and recording the observational sessions for qualitative and quantitative analysis.
\end{titled-frame}

Ultimately, we seek to understand the conceptual understanding of software engineers who operationalise stochastic and probabilistic systems, and furthermore understand knowledge representation with these systems' \gls{api} documentation. Our motivation is to provide insight into current practices and compare the best practices with actual practise. We strive for this to  provide developers with a guiding framework on how to best operationalise these systems via the form of some checklist or tool they can use to ensure optimal software quality.

It is anticipated that the findings from this study in the computer vision \glspl{cis} space will be generalisable to other areas, such as time-series information, natural language processing and others.

%  Paper 2:
%* RQ1. How do software engineers evaluate (knowledge representation) machine learning APIs for use in an application?
%   * Motivation: to provide insights into the current practice
%   * Method: Survey
%* RQ2. Do software engineers follow best practices when evaluating machine learning APIs?
%   * Motivation: to compare best practice with actual practice
%   * Survey
\section{Thesis Organisation}
\label{sec:introduction:organisation}

We organise the thesis into four parts. \textbf{\Cref{part:preface}~(\textit{The Preface})} includes introductory, background and methodology chapters. This is a \textit{PhD by Publication}, and \textbf{\cref{part:publications}~(\textit{Publications})} comprises of eight publications resulting from this work over \cref{ch:icsme2019,ch:tse2020,ch:icse2020,ch:fse-demo2020,ch:fse2020,ch:icwe2019,ch:semotion2021,ch:caise2021}; publications are included verbatim except for terminology and formatting changes to better fit the suitability of a coherent thesis. \textbf{\Cref{part:postface}~(\textit{The Postface})} includes the conclusion and future works chapter, as well as a list of academic studies and online artefacts referenced within the thesis. \textbf{\Cref{part:appendices}~(\textit{Appendices})} includes all supplementary material, including mandatory authorship statements and ethics approval. Details of each chapter following this introductory chapter are provided in the following section.

\subsection{\Cref{part:preface}: Preface}

\subsubsection{\cref{ch:background}: Background} This chapter provides an overview of prior studies broadly around three key pillars: the development of an \gls{iws}, the usage of an \gls{iws}, and the nature of an \gls{iws}. We use the three perspectives of software quality (particularly, reliability), probabilistic and non-deterministic systems, and explanation and communication theory to describe prior work.

\subsubsection{\cref{ch:research-methodology}: Research Methodology} This chapter provides a summative review of research methods and philosophical stances relevant to software engineering. We illustrate that the methods used within our publications are sound via an analysis of the methodologies used in seminal works referenced in this thesis.

\subsection{\Cref{part:publications}: Publications}

\subsubsection{\cref{ch:icsme2019}: Exploring the nature of \glspl{cvs}} This chapter was presented at the 2019 \textbf{International Conference on Software Maintenance and Evolution~(ICSME)}~\citep{Cummaudo:2019icsme}. We describe an 11-month longitudinal experiment assessing the behavioural (run-time) issues of three popular \glspl{cvs}: Google Cloud Vision~\citepweb{GoogleCloud:Home}, Amazon Rekognition~\citepweb{AWS:Home} and Azure Computer Vision~\citepweb{Azure:Home}. By using three different data sets---two of which we curate as additional contributions---we demonstrate how the services are inconsistent amongst each other and within themselves. This study answers \ref{rq:nature}: Despite presenting conceptually-similar functionality, each service behaves and produces slightly varied (inconsistent) results and demonstrates non-deterministic runtime behaviour. We discuss potential evolution risks to consumers of such services as the services provide non-static outputs for the same inputs, thereby having significant impact to the robustness of consuming applications. Further details in the study include a brief assessment into the lack of sufficient detail of these concerns in their documentation.

\subsubsection{\cref{ch:icse2020}: Understanding developer struggles when using \glspl{cvs}} This chapter was presented at the \textbf{2020 International Conference on Software Engineering~(ICSE)}~\citep{Cummaudo:2020icse}. We conduct a mining study of 1,425 \glslong{so} questions that provide indications of the types frustrations that developers face when integrating \glspl{cvs} into their applications. To gather what their pain-points are, we use two classification taxonomies that also use \glslong{so} to understand generalised and documentation-specific pain-points in mature software engineering domains. This study answers \ref{rq:devs} in detail and provides a validation to our motivation of \ref{rq:docs}: we validate that the \textit{completeness} of current \gls{cvs} \gls{api} documentation is a main concern for developers and there is insufficient explanation into the errors and limitations of the service. We find that the documentation does not adequately cover all aspects of the technical domain. In terms of integrating with the service, developers struggle most with simple errors and ways in which to use the \glspl{api}; this is in stark contrast to mature software domains. Our interpretation is that developers fail to understand the \gls{iws} lifecycle and the `whole' system that wraps such services. We also interpret that developers have a shallower understanding of the core issues within \glspl{cvs} (likely due to the nuances of \gls{ml} as suggested in a discussion in the paper), which warrants an avenue for future work in software engineering education.

\afterpage{
\begin{landscape}
\begin{table}
  \centering
  \caption[List of publications resulting from this thesis]{List of publications resulting from this thesis, separated by phenomena exploration (above) and solution design (below).}
  \label{tab:introduction:structure:list-of-pubs}
  \tablefitlandscape{0.85}{\begin{tabular}{rp{0.335\linewidth}ccc|cc}
    \toprule
    \textbf{Ref.} &
    \textbf{Venue} &
    \textbf{Acronym} &
    \textbf{Rank\tablefootnote{Conference publications ranking measured using the CORE Conference Ranks (\url{http://www.core.edu.au/conference-portal}) and Journal publications rankings using the Scimago Ranking (\url{https://www.scimagojr.com/}). Rankings retrieved January 2020.}} &
    \textbf{Published\tablefootnote{Date of publication, if applicable.}} &
    \textbf{Chapter} &
    \textbf{RQs}\\
    \midrule
    
    % ICSME
    \citep{Cummaudo:2019icsme} &
    35\textsuperscript{th} International Conference on Software Maintenance and Evolution &
    ICSME &
    A &
    05 Dec 2019 &
    \cref{ch:icsme2019} &
    \ref{rq:nature} \\

    % ESEM
    \citep{Cummaudo:2019esem} &
    13\textsuperscript{th} International Symposium on Empirical Software Engineering and Measurement &
    ESEM &
    A &
    17 Oct 2019 &
    Excluded\tablefootnote{The extended version of this conference proceeding is provided in \cref{ch:tse2020}.} &
    \ref{rq:docs:complete} \\
    
    % ICSE
    \citep{Cummaudo:2020icse}&
    42\textsuperscript{nd} International Conference on Software Engineering&
    ICSE &
    A* &
    \textit{In Press}&
    \cref{ch:icse2020}&
    \ref{rq:devs} \\
    
    % SEmotion
    \citep{Cummaudo:2021semotion}&
    6\textsuperscript{th} International Workshop on Emotion Awareness in Software Engineering\tablefootnote{An ICSE 2021 workshop.}&
    SEmotion&
    A* &
    \textit{In Progress}&
    \cref{ch:semotion2021}&
    \ref{rq:devs:frustration}\\
    
    % SEIP
    \citep{Graetsch:2021caise}&
    33\textsuperscript{rd} International Conference on Advanced Information Systems Engineering&
    CAiSE&
    A &
    \textit{In Progress}&
    \cref{ch:caise2021}&
    \ref{rq:devs:frustration}\\
    
    \midrule
    
    % TSE
    \citep{Cummaudo:2020tse}&
    IEEE Transactions on Software Engineering & 
    TSE &  
    Q1&
    \textit{In Review}& 
    \cref{ch:tse2020} &
    \ref{rq:docs} \\
    
    % ICWE
    \citep{Ohtake:2019vi} & 
    13\textsuperscript{th} International Conference on Web Engineering&
    ICWE&
    B&
    26 Apr 2019 &
    \cref{ch:icwe2019} &
    \ref{rq:design} \\
    
    % ICSE(d)
    \citep{Cummaudo:2020fse-demo}&
    28\textsuperscript{th} Joint European Software Engineering Conference and Symposium on the Foundations of Software Engineering&
    FSE(d)\tablefootnote{We abbreviate this with an added `d' (for the demonstrations track) to distinguish this paper from our full FSE 2020 paper.} &
    A* &
    \textit{In Press}&

    \cref{ch:fse-demo2020} &    
    \ref{rq:design} \\
     
    % FSE
    \citep{Cummaudo:2020fse}&
    28\textsuperscript{th} Joint European Software Engineering Conference and Symposium on the Foundations of Software Engineering&
    FSE&
    A*&
    \textit{In Press} &
    \cref{ch:fse2020} &
    \ref{rq:design} \\

    \bottomrule
  \end{tabular}}  
\end{table}
\end{landscape}
}

\subsubsection{\cref{ch:semotion2021}: Ranking \gls{cvs} pain-points by frustration} This chapter has been published as a technical report pre-print on arXiv and an extended version is \textbf{in review} for submission to the \textbf{2021 International Workshop on Emotion Awareness in Software Engineering (SEmotion)}~\citep{Cummaudo:2021semotion}. In this work, we use our dataset consisting of the 1,425 Stack Overflow questions from \citep{Cummaudo:2020icse} to interpret the breakdown of emotions developers express per classification of pain-points conducted in \cref{ch:icse2020}. We find that the distribution of various emotions differ per question type, and developers are most frustrated when the expectations of a \gls{cvs} does not match the reality of what these services actually provide, which shapes our answer for \ref{rq:devs:frustration} and thus \ref{rq:devs}.

\subsubsection{\cref{ch:caise2021}: Lessons in applying pre-trained models to Stack Overflow} This chapter is \textbf{in review} the \textbf{2021 International Conference on Advanced Information Systems Engineering (CAiSE)}~\citep{Graetsch:2021caise}. This work presents a deeper investigation into the classification model used within \cref{ch:semotion2021} to better interpret the automation effort we conducted, thereby highlighting valuable lessons we learnt from performing this exercise. Specifically, we find that the classification model we used in this exercise presented substantial data imbalance, which presented unexpected results (namely, a high level of posts that showed the emotion, `love'). We identify how novel documentation tooling such as model cards \citep{Mitchell:2018in} or datasheets \citep{Gebru:2018wh} could have identified risks to our study earlier, and make suggestions needed into future documentation efforts. This work presents complementary  results to \ref{rq:docs} to help propose which documentation elements \gls{ml} models (and thus \glspl{iws}) should provide before diving `straight in'.

\subsubsection{\cref{ch:tse2020}: Investigating improvements to \gls{cvs} \gls{api} documentation} This chapter was accepted as a paper at the \textbf{2019 International Symposium on Empirical Software Engineering and Measurement~(ESEM)} \citep{Cummaudo:2020icse}. To understand where to improve \gls{cvs} documentation, we first need to investigate \textit{what} makes a good \gls{api} document. This short paper initially answered one aspect of \ref{rq:docs:complete}: the extent by which \textit{academic literature} has studied various \gls{api} documentation artefacts. By conducting an \glslong{sms} resulting in 21 primary studies, we systematically develop a taxonomy that combines documentation artefacts studied in scattered work into a structured framework of 5 dimensions and 34 weighted categorisations. We then extend this work by triangulating the taxonomy with opinions from developers using a survey to assess the efficacy of these artefacts (thereby answering the second aspect of \ref{rq:docs:complete}). From this, we assess the how well \gls{cvs} providers document their \glspl{api} via a heuristic validation of the taxonomy, using the three services from the ICSME publication to make recommendations where documentation should be more complete, thereby answering \ref{rq:docs:missing} (and thus \ref{rq:docs}). The extended version of this chapter has been submitted to the \textbf{IEEE Transactions on Software Engineering (TSE)} in~\citep{Cummaudo:2020tse} and we are in the process of finalising a minor review.

\subsubsection{\cref{ch:icwe2019}: Merging responses of multiple \glspl{cvs}} This chapter was presented at the \textbf{2019 International Conference on Web Engineering~(ICWE)}~\citep{Ohtake:2019vi}. Early exploration of \glspl{cvs} showed that multiple services use vastly different ontologies for the same input. As an initial strategy to improve the reliability of these services, we explored if merging multiple responses using WordNet \citep{WordNetMiller1995} and a novel label merging algorithm based on the proportional representation approach used in political voting could make any improvements. While this approach resulted in a modest improvement to reliability, it did not consider to the evolution issues or developer pain-points we later identified.

\subsubsection{\cref{ch:fse-demo2020}: Developing a confidence thresholding tool} This chapter was presented at the demonstrations track of the \textbf{2020 Joint European Software Engineering Conference and Symposium on the Foundations of Software Engineering (ESEC/FSE)}~\citep{Cummaudo:2020fse-demo}. When integrating with a \gls{cvs}, developers need to select an appropriate confidence threshold suited to their use case and determine whether a decision should be made. An issue, however, is that these \glspl{cvs} are not calibrated to the specific problem-domain datasets and it is difficult for software developers to determine an appropriate confidence threshold on their problem domain. This tool presents a workflow and supporting tool for application developers to select decision thresholds suited to their domain that---unlike existing tooling---is designed to be used in pre-development, pre-release and production. This tooling forms part of a solution to \ref{rq:design} for developers to maintain robustness and reliability in their systems.

\subsubsection{\cref{ch:fse2020}: Developing a \gls{cvs} integration architecture} This chapter was presented at the \textbf{2020 Joint European Software Engineering Conference and Symposium on the Foundations of Software Engineering (ESEC/FSE)}~\citep{Cummaudo:2020fse}. Based on the findings, we propose a set of new service error codes for describing the empirically observed error conditions of \gls{iws} based on our findings in \cref{ch:icsme2019}. To achieve this, we propose a proxy server intermediary that lies between a client application and a \gls{iws}; the proxy server tactic is designed to return these error codes when substantial evolution occurs against a benchmark dataset that represents the application domain context (similar to that proposed in \cref{ch:fse-demo2020}). A technical evaluation of our implementation of this architecture identifies 1,054 cases of substantial evolution in confidence values and 2,461 cases of evolution in the response label sets when 331 images were sent to a \gls{cvs}.

\subsection{\Cref{part:postface}: Postface}

In \Cref{ch:conclusions}, we review the contributions made in this thesis and the relevance and significance to identifying and resolving key issues when application developers integrate with \gls{cvs}. We evaluate these outcomes with reference to the research goals, and discuss threats to validity of the work. Lastly, we discuss the various avenues of research arising from this work. References from literature and a list of online artefacts are provided after this concluding chapter.

\subsection{\Cref{part:appendices}: Appendices}

\Cref{ch:additional-materials} thru \cref{ch:ethics} are appendices. \Cref{ch:additional-materials} provides additional material referenced within this thesis but not provided in the body. The source code for the reference architecture described in \cref{ch:fse2020} is reproduced in \cref{ch:reference-architecture-code}. The supplementary materials published with \cref{ch:tse2020} are reproduced in \cref{ch:tse-supplementary-materials}, which also describes the list of primary sources arising in the \glslong{sms} we conducted.
 We provide mandatory coauthor declaration forms describing the contribution breakdown for each publication within \cref{ch:authorship-statements}. \Cref{ch:ethics} contains copies of the ethics clearance for various experiments within this thesis. 

\section{Research Contributions}
\label{sec:introduction:research-contributions}

The outcomes of answering the four primary research questions elaborated in \cref{sec:introduction:goals} shapes three primary contributions this thesis offers to software engineering knowledge:

\begin{itemize}
  \item An \textbf{improved understanding in the landscape of \glspl{cvs}}, with respect to their runtime behaviour and evolutionary profiles. 
  \item A novel \textbf{service integration architecture} that helps developers with integrating their applications with \glspl{cvs}.
  \item A \textbf{key list of attributes that should be documented}, to assist \gls{cvs} providers to better document their services.
\end{itemize}

In this section, we detail how each publication forms a coherent body of work and how each publication relates to the primary contributions made.

After our exploratory analysis on the nature of \glspl{cvs} (\cref{ch:icsme2019}), we proposed two sets of recommendations targeted towards two stakeholders: (i) the service \textit{consumers} (i.e., application developers) and (ii) the service \textit{providers}. Our subsequent publications arose as a two-fold investigation to develop two strategies in which developers and providers can, respectively, (i) better integrate these intelligent components into their applications, and (ii) how these services can be better documented. \Cref{tab:introduction:structure:list-of-pubs} provides a tabulated form of the publications and research questions addressed within this thesis; for ease of reference, we refer to the publications in within this section in their abbreviated form as listed in \cref{tab:introduction:structure:list-of-pubs}. We also provide abbreviations for easier reference in this section. A high-level overview of the cohesiveness of our publications is provided in \cref{fig:introduction:structure:publications-overview}.

\begin{figure}[hbt]
  \includegraphics[width=\linewidth]{publications-overview}
  \caption[Overview publication coherency]{Activity diagram of the coherency of our publications, how our research was conducted, and relevant connections between publications. Our two-phase structure initial phenomena exploration and a proposed solutions to issues identified from the exploration. We map the contributions within each publication to the three primary contributions of the thesis. Acronyms of each publication are provided in \cref{tab:introduction:structure:list-of-pubs}.}
  \label{fig:introduction:structure:publications-overview}
\end{figure}

\subsection{Contribution 1: Landscape Analysis \& Preliminary Solutions}

The first two bodies of work in this paper are the ICSME and ICWE papers. These two works investigated a landscape analysis \glspl{cvs} from two perspectives: firstly, we conducted a longitudinal study to better understand the attributes associated with these services (ICSME)---particularly their evolution and behavioural profiles, and their potential impacts to software reliability---and tackled a preliminary solution facade to `merge' responses of the services together (ICWE). 

The ICSME paper confirmed our hypotheses that the services have a non-deterministic behavioural profile, and that the evolution occurring within the \gls{ml} models powering these services are not sufficiently communicated to software engineers. This therefore led to follow up investigation into how developers perceive these services, and thereby determine if they are frustrated due to this lack of communication. 

Our ICWE paper explored one aspect identified from the ICSME paper that we identified early on: that different services use different vocabularies to describe semantically similar objects but in different ways (e.g., `border collie' vs. `collie'), despite offering functionally similar capabilities. We attempted to merge the response labels from these services using a proportional representation approach, and upon comparison with more naive merge approaches, we improved label-merge performance by an F-measure of 0.015. However, while this was an interesting outcome for a preliminary solution design, investigation from our following work suggested that standardising ontologies between service providers becomes challenging and normalising the entire ontological hierarchy of response labels would need to fall under the responsibility of a certain body (that does not exist). Further, we did not find sufficient evidence that developers would frequently switch between service providers. Therefore, we opted for a shielded relay architecture in our later design work.  

\subsection{Contribution 2: Improving Documentation Attributes}

As mentioned, our ICSME paper found that evolutionary and non-deterministic behavioural profile of are not adequately documented in pre-trained \gls{ml} model \glspl{api} documentation, and further developers find this frustrating (\cref{ch:semotion2021}) and potential issues can arise as a result (\cref{ch:caise2021}). A recommendation concluding from this work was that service providers should improve their documentation, however there lacked a strategy by which they could do this, and our hypotheses that developers were actually frustrated by this lack of communication was yet to be tested. This led to two follow-up further investigations as presented in our ICSE and ESEM papers.

One aspect of our ICSE paper was to confirm whether developers are actually frustrated with the service's limited \gls{api} documentation. By mining \glslong{so} posts with reference to documentation issues, we adopted a \citeyear{Aghajani:2019bo} documentation-related taxonomy by \citet{Aghajani:2018et} to classify posts, and found that 47.87\% of posts classified fell under the `completeness' dimension of \citeauthor{Aghajani:2018et}'s taxonomy. This interpretation, therefore, warranted the recommendation proposed in the ICSME paper to improve service documentation. 

However, though improvements to more complete documentation was justified from the ICSE paper, we needed to explore exactly \textit{what} makes a `complete' \gls{api} document. By conducting a \glslong{sms} resulting in 4,501 results, we curated 21 primary studies that outline the facets of \gls{api} documentation knowledge. From these studies, we distilled a documentation framework describing a prioritised order of the documentation assets \gls{api}'s should document that is described in our ESEM short paper. After receiving community feedback, we extended this short paper with a follow-up experiment submitted to TSE. By conducting a survey with developers, we assessed our \gls{api} documentation taxonomy's efficacy with practitioner opinions, thereby producing a weighted taxonomy against \textit{both} literature and developer sources. Lastly, we triangulated both weightings against a heuristic evaluation against common \gls{cvs} providers' documentation. This allowed us to deduce which specific areas in existing \gls{cvs} providers' \gls{api} documentation needed improvement, which was a primary contribution from our TSE article.

\subsection{Contribution 3: Service Integration Architecture}

Two recommendations from our ICSME study encouraged developers to test their applications with a representative ontology for their problem domain and to incorporate a specialised testing and monitoring techniques into their workflow. Strategies on \textit{how} to achieve this were explored in later studies.  Following a similar approach to our solution of improved \gls{api} documentation, we validated the substantiveness of our recommendations using our mining study of \glslong{so} (our ICSE paper) to help inform us of generalised issues developers face whilst integrating \glspl{cvs} into their applications. To achieve this, we used a \glslong{so} post classification taxonomy proposed by \citet{Beyer:2018fm} into seven categories, where 28.9\% and 20.37\% of posts asked issues regarding how to use the \gls{cvs} \gls{api} and conceptual issues behind \glspl{cvs}, respectively. Developers presented an insufficient understanding of the non-deterministic runtime behaviour, functional capability, and limitations of these services and are not aware of key computer vision terminology. When contrasted to more conventional domains such as mobile-app development, the spread of these issues vary substantially.

We proposed two technical solutions in our two FSE papers to help alleviate this issue. Firstly, our FSE demonstrations paper---FSE(d) for short---provides a workflow for developers to better select an appropriate confidence threshold, and thus decision boundary, calibrated for their particular use case. In our ESEC/FSE paper, we provide a reference architecture for developers to guard against the non-deterministic issues that may `leak' into their applications. This architecture tactic proposes a client-server intermediary proxy server, similar to the style proposed in our ICWE paper. However, unlike the ICWE paper that uses proportional representation approach to modify multiple sources, our FSE paper proposes a guarded relay, whereby a single service is used, and the proxy server maintains a lifecycle to monitor evolution issues identified in ICSME and should be benchmarked against the developer's dataset (i.e., against the particular application domain) as suggested in FSE(d). For robust component composition, this architecture tactic handles four key requirements: (i) it clearly defines erroneous conditions that occur when evolution occurs in \glspl{cvs}; (ii) it notifies of behavioural changes in the service; (iii) it monitors the service for change and substantial impact this may have to the client application; and (iv) is flexible enough to be implemented and adaptable to any client application or specific intelligent service to facilitate reuse. Both FSE papers serve as two primary contributions to \ref{rq:design}.

% RQ3.3 -- range is diff == existing strategies are not enough!

