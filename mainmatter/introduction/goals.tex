\section{Research Outcomes}
\label{sec:introduction:hypohtesis}

\itshape
In this work, we present a framework to improve the documentation quality of \glsplx{cvcis} and their \glspl{api}. We demonstrate that developers currently lack the understanding of how these services function and our framework mitigates this by presenting a solution to improve the documentation quality of the \glspl{api} and improve the existing techniques used to integrate these services into developer's end-applications.

%In this thesis, we explore the probabilistic ripple-effect with relation to the development usability of `intelligent' \glspl{api}; specifically, we contextualise within \glspl{cvcis}. Our anchoring perspective is software quality---specifically, validation and verification---within such systems and what best practices within the field of software engineering can be applied to assist in operationalisation such systems.
\upshape

The goals of this study aim to provide a snapshot of current developer best practices towards the usage of \glspl{cis} to provide a guiding framework and recommendations for software developers and \glspl{cis} providers alike. Our anchoring perspective is software quality---specifically, validation and verification---within such systems and what best practices within the field of software engineering can be applied to assist in consumption of \glspl{cis}.
 Based on the motivating case studies in \cref{sec:introduction:motivation}, we articulate three Research Hypotheses (RH1--3) below and seven Research Questions (RQs) based on  both empirical and non-empirical software engineering methodology \citep{Shull:2007vh,Simon:1996uw}.

\newcommand{\rh}[1]{\hyperref[rh#1]{RH#1}}

\begin{titled-frame}{\underline{RH1}: \textit{Existing \glspl{cis} present insufficient \gls{api} documentation for general use.} }
\phantomsection
\label{rh1}
\vspace{-12pt}
\paragraph{Research Hypothesis}
\gls{api} documentation of intelligent services are inadequate and insufficient given the disparity of mindsets between the application developers and \gls{cis} providers. Chiefly, application developers have limited general understanding of the `magic' that occurs behind these probabilistic `intelligent' \glspl{api}. We do not know what key aspects of the documentation matter to them, nor what they do or do not understand of the existing documentation.

\paragraph{Research Goal}
To improve the effectiveness of the documentation in existing \gls{cis} providers, specifically of \gls{cvcis} \glspl{api}.

\paragraph{Research Questions}
\begin{enumerate}[label=\textbf{RQ1.\arabic*.}, ref=RQ1.\arabic*, leftmargin=3.5\parindent, rightmargin=1\parindent]
  \item What practices are in use for intelligent services' \gls{api} documentation? 
  % KQ: Exploratory question to understand phenomena: Description and Classification question.
  \label{rqs:apidoc:what-is-in-use}
  
  \item How do developers currently understand and interpret the documentation given a lack of formal training in artificial intelligence? That is, what do they understand and not understand, and what key aspects of the \gls{api} documentation matter do developers as they see it?
  % KQ: Base-rate question to understand normal patterns of the occurrence of the phenomena: Descriptive-Process questions.
  \label{rqs:apidoc:how-do-devs-understand-it}
  
  \item What additional information or attributes would developers prefer to be included in the \gls{api} documentation?
  % DQ: Design question
  \label{rqs:apidoc:what-additional-information-needed}
\end{enumerate}

\paragraph{Research Contribution} An intelligent service \gls{api} documentation quality assessment framework to evaluate how well the service has been documented for software engineers to use.

\end{titled-frame}

\begin{titled-frame}{\underline{RH2}: \textit{Existing \glspl{cis} present insufficient metadata for context-specificity.} }
\phantomsection
\label{rh2}
\vspace{-12pt}
\paragraph{Research Hypothesis}
Intelligent service \glspl{api} respond with insufficient information for developers to operationalise the service into a business-driven application and, thus, additional metadata is needed to assist developers. Such metadata is likely to be needed as part of the response objects of the \gls{api}.

\paragraph{Research Goal}
To improve the quality of \textit{context-specific response data} from the \gls{api} endpoints of intelligent services.

\paragraph{Research Questions}
\begin{enumerate}[label=\textbf{RQ2.\arabic*.}, ref=RQ2.\arabic*, leftmargin=3.5\parindent, rightmargin=1\parindent]
  \item What are current problems due to lack of return metadata?
  % KQ; Exploratory question to understand phenomena: Description and Classification question
  \label{rqs:metadata:what-problems-du                                                                e-to-lack-of-metadata}
  
  
  \item What additional metadata do developers desire to achieve implementing context-specific applications?
  % DQ: design question
  \label{rqs:metadata:what-metadata-do-devs-want-and-why}
\end{enumerate}

\paragraph{Research Contributions} A list of metadata key-value-pairs that assist developers in using these \glspl{api} during the development of software that consume these services. In essence, improvements to the framework of Research Outcome 1: ``\textit{An intelligent service \gls{api} documentation \underline{\upshape and metadata} quality assessment framework}''.

\end{titled-frame}

\begin{titled-frame}{\underline{RH3}: \textit{\rh{1} and \rh{2} improve quality,  productivity or developer informativeness.}}
\phantomsection
\label{rh3}
\vspace{-12pt}
\paragraph{Research Hypothesis}
The implication of hypotheses 1 and 2 suggest that improving both the documentation and providing further metadata will improve product quality (internal or external), and/or developer productivity and/or developer education in developing software with intelligent components.

\paragraph{Research Goal}
 To confirm if improvements to \gls{api} documentation and response metadata  are reflected as improvements to product quality, developer productivity and/or developer education.

\paragraph{Research Questions}
\begin{enumerate}[label=\textbf{RQ3.\arabic*.}, ref=RQ3.\arabic*, leftmargin=3.5\parindent, rightmargin=1\parindent]
  \item  Does an improvement of documentation or metadata correlate to an improvement in software quality, developer productivity and/or developer informativeness?
  % KQ; relationship question to understand the correlation between two phenomena
  \label{rqs:implications:do-metrics-improve}
  
  \item With respect to \ref{rqs:implications:do-metrics-improve}, the three aspects are explored:
  % KQ; causality questions...
  \begin{enumerate}
  \item Does the improvement cause increased product quality, as measured through improved external quality metrics?
  \item Does the improvement cause increased developer productivity, as measured through improved internal quality metrics?
  \item Does the improvement cause increased developer informativeness or increased confidence in developing \gls{cis}-powered applications?
  \end{enumerate}
  \label{rqs:implications:aspects}  
\end{enumerate}

\paragraph{Research Contribution}
A concrete sample solution or framework that improves such metrics, thereby confirming that our documentation and metadata quality assessment framework improves these facets.
\end{titled-frame}

Ultimately, we seek to understand the conceptual understanding of software engineers who operationalise stochastic and probabilistic systems, and furthermore understand knowledge representation with these systems' \gls{api} documentation. Our motivation is to provide insight into current practices and compare the best practices with actual practise. We strive for this to  provide developers with a guiding framework on how to best operationalise these systems via the form of some checklist or tool they can use to ensure optimal software quality.

It is anticipated that the findings from this study in the \glspl{cvcis} space will be generalisable to other areas, such as time-series information, natural language processing and others.

%  Paper 2:
%* RQ1. How do software engineers evaluate (knowledge representation) machine learning APIs for use in an application?
%   * Motivation: to provide insights into the current practice
%   * Method: Survey
%* RQ2. Do software engineers follow best practices when evaluating machine learning APIs?
%   * Motivation: to compare best practice with actual practice
%   * Survey