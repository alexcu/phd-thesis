\section{Research Goals}
\label{sec:introduction:goals}

% Primary/Secondary
\def\NumPrimaryRQs{four}
\def\NumSecondaryRQs{seven}

% Category
\def\NumEmpiricalRQs{nine}
\def\NumNonEmpiricalRQs{two}

% Subcategory
\def\NumExploratoryRQs{seven}
\def\NumBaseRateRQs{two}

% Question Types
\def\NumDescriptionAndClassificationRQs{five}
\def\NumExistenceRQs{one}
\def\NumCompositionRQs{one}
\def\NumDescriptiveComparativeRQs{one}
\def\NumFrequencyDistributionRQs{two}

% RQ1
\def\RQOneTextLandscapeAnalysis{What is the nature of cloud-based \glspl{cvs}?}
\def\RQOneTextLandscapeAnalysisRuntime{What is their runtime behaviour?}
\def\RQOneTextLandscapeAnalysisEvolution{What is their evolution profile?}

% RQ2
\def\RQTwoTextDocumentation{Are \gls{cvs} \glsacpl{api} sufficiently documented?}
\def\RQTwoTextDocumentationWhatIsCompleteDocs{What \gls{api} documentation artefacts compromise a `complete' API document, according to both literature and practitioners?}
\def\RQTwoTextDocumentationMissingAttributes{What additional information or attributes do application developers need in \gls{cvs} \gls{api} documentation to make it more complete?}

% RQ3
\def\RQThreeTextDevMiscomprehension{Are \glspl{cvs} more misunderstood than conventional software engineering domains?}
\def\RQThreeTextDevMiscomprehensionIssueTypes{What types of issues do application developers face most when using \glspl{cvs}, as expressed as questions on \glslong{so}?}
\def\RQThreeTextDevMiscomprehensionFrustration{Which of these issues are application developers most frustrated with?}
\def\RQThreeTextDevMiscomprehensionVsConventional{Is the distribution \gls{cvs} pain-points different to established software engineering domains, such as mobile or web development?}

% RQ4
\def\RQFourDesign{What strategies can developers employ to integrate their applications with \glspl{cvs} while preserving robustness and reliability?}

This thesis aims to investigate and better understand the nature of cloud-based \glsplx{cvs}\footnote{As these services are proprietary, we are unable to conduct source code or model analysis, and hence are not used in the investigation of this thesis.} as a concrete exemplar of \glsplx{iws}.
We identify the maturity, viability and risks of \glspl{cvs} through the anchoring perspective of \textit{reliability} that affects the internal and external quality of software. We adopt the McCall \citep{McCall:1977uy} and Boehm \citep{Boehm:1978vv} interpretations of reliability via the sub-characteristics of a service's \textit{consistency} and \textit{robustness} (or fault/error tolerance), and the \textit{completeness}\footnote{We treat the \gls{api} documentation of a \gls{cvs} as a first-class citizen.} of its documentation. (A detailed discussion is further provided in \cref{sec:background:software-quality}.)
This thesis explores and contributes towards \textit{four} key facets regarding reliability in \gls{cvs} usage and the completeness of its associated documentation. We formulate four primary research questions (RQs), based on both empirical and non-empirical software engineering methodology \citep{Simon:1996uw}, further discussed in \cref{ch:research-methodology}.

Firstly, we investigate adverse implications that arise when using \glspl{cvs} that affects consistency and robustness~(\textbf{\cref{ch:icsme2019}}). We show how \glspl{cvs} have a non-deterministic runtime behaviour and evolve with unintended and non-trivial consequences to developers. We demonstrate that these services have inconsistent behaviour despite offering the same functionality and pose evolution risk that effects robustness of consuming applications when responses change given the same (consistent) inputs. 

% TODO: Ensure this stays on the same page
\clearpage
Formally, we structure the following RQs:

\begin{leftbar}
\begin{enumerate}[label=\faQuestionCircle~~\textbf{RQ\arabic*.}, ref=RQ\arabic*, leftmargin=2.5\parindent, rightmargin=1\parindent]
    \item \textbf{\RQOneTextLandscapeAnalysis{}}\label{rq:nature}
    \begin{enumerate}[label=\textit{RQ1.\arabic*.}, ref=RQ1.\arabic*]
      \item \RQOneTextLandscapeAnalysisRuntime{}\label{rq:nature:runtime}
      \item \RQOneTextLandscapeAnalysisEvolution{}\label{rq:nature:evolution}
    \end{enumerate}
\end{enumerate}
\end{leftbar}

Secondly, we investigate the reliability of the documentation these services offer through the lenses of its completeness. We collate prior knowledge of good \gls{api} documentation and assess the efficacy of such knowledge against practitioners~(\textbf{\cref{ch:tse2020}}). We show that these service's behaviour and evolution is not reliably documented adequately against this knowledge. Formally, we develop the following RQs:

\begin{leftbar}
\begin{enumerate}[label=\faQuestionCircle~~\textbf{RQ\arabic*.}, ref=RQ\arabic*, leftmargin=2.5\parindent, rightmargin=1\parindent,start=2]
    \item \textbf{\RQTwoTextDocumentation{}}\label{rq:docs}
    \begin{enumerate}[label=\textit{RQ2.\arabic*.}, ref=RQ2.\arabic*]
      \item \RQTwoTextDocumentationWhatIsCompleteDocs{} \label{rq:docs:complete}
      \item \RQTwoTextDocumentationMissingAttributes{} \label{rq:docs:missing}
    \end{enumerate}
\end{enumerate}
\end{leftbar}

Thirdly, we investigate how software developers approach using these services and directly assess developer pain-points resulting from the nature of \glspl{cvs} and their documentation~(\textbf{\cref{ch:icse2020}}). We show that there is a statistically significant difference in these complaints when contrasted against more established software engineering domains (such as web or mobile development) as expressed as questions asked on \glslong{so}. We provide a number of exploratory avenues for researchers, educators, software engineers and \gls{iws} providers to alleviate these complaints based on this analysis. Further, using a data set consisting of 1,245 \glslong{so} questions, we explore the emotional state of developers to understand which aspects (i.e., pain-points) developers are most frustrated with~(\textbf{\cref{ch:semotion2021}}) and the types of traps developers can fall into when substantial documentation is not provided for specific \gls{ml} models~(\textbf{\cref{ch:caise2021}}). We formulate the following RQs:

\begin{leftbar}
\begin{enumerate}[label=\faQuestionCircle~~\textbf{RQ\arabic*.}, ref=RQ\arabic*, leftmargin=2.5\parindent, rightmargin=1\parindent,start=3]
  \item \textbf{\RQThreeTextDevMiscomprehension{}}\label{rq:devs}
  \begin{enumerate}[label=\textit{RQ3.\arabic*.}, ref=RQ3.\arabic*]
    \item \RQThreeTextDevMiscomprehensionIssueTypes{} \label{rq:devs:issues}
    \item \RQThreeTextDevMiscomprehensionFrustration{} \label{rq:devs:frustration}
    \item \RQThreeTextDevMiscomprehensionVsConventional{} \label{rq:devs:vs-traditional}
  \end{enumerate}
\end{enumerate}
\end{leftbar}


Lastly, we explore several strategies to help improve \glspl{cvs} reliability. Firstly, we investigate whether merging the responses of \textit{multiple} \glspl{cvs} can improve their reliability and propose a novel algorithm---based on the proportional representation method used in electoral systems---to merge labels and associated confidence values from three providers (\textbf{\cref{ch:icwe2019}}). Secondly, we develop an integration architecture style (or facade) to guard against \gls{cvs} evolution, and synthesise an integration workflow that addresses the concerns raised by developers in addition to embedding `complete' documentation artefacts into the workflow's design (\textbf{\cref{ch:fse-demo2020,ch:fse2020}}). Our final RQ is:

\begin{leftbar}
\begin{enumerate}[label=\faQuestionCircle~~\textbf{RQ\arabic*.}, ref=RQ\arabic*, leftmargin=2.5\parindent, rightmargin=1\parindent,start=4]
  \item \textbf{\RQFourDesign{}}\label{rq:design}
\end{enumerate}
\end{leftbar}

\section{Research Methodology}

This thesis employs a mixed-methods approach using the concurrent triangulation strategy \citep{Jick:1979el,Bratthall2002}. The research presented consists of both empirical and non-empirical research design. This section provides a high-level overview of the research methodology within this thesis. Further details are provided in  \cref{sec:introduction:research-contributions,ch:research-methodology}.

Firstly, \ref{rq:nature}--\ref{rq:devs} are all empirical, knowledge-based questions \citep{Easterbrook:2007ws,Meltzoff:1998wg} that aim to provide the software engineering community with a greater understanding of the phenomena surrounding \glspl{cvs} from three perspectives: the nature of the services themselves, how developers perceive these services and how service providers can improve these services.  We answer \ref{rq:nature} using a longitudinal experiment that assesses both the services' responses and associated documentation (complementing \ref{rq:docs:missing}). We adopt qualitative and quantitative data collection; specifically (i) structured observations to quantitatively analyse the results over time, and (ii) documentary research methods to inspect service documentation.
Secondly, we perform systematic mapping study following the guidelines of \citet{Kitchenham:2007dd} and \citet{Petersen:2008td} to better understand how \gls{api} documentation of these services can be improved (i.e., more complete), which targets \ref{rq:docs}. Based on the findings from this study, we use a systematic taxonomy development methodology specifically targeted toward software engineering \citep{Usman:2017hn} that structures scattered \gls{api} documentation knowledge into a taxonomy. We then validate this taxonomy against practitioners using survey research, using a survey instrument inspired by \citeauthor{Brooke:1996ua}'s well-established \glslong{sus} \citep{Brooke:1996ua}  and contextualising it within \gls{api} documentation utility, which answers \ref{rq:devs:vs-traditional}. To answer \ref{rq:docs:missing}, we perform an empirical application of the taxonomy to three \glspl{cvs}, and therefore assess where improvements can be made.
Thirdly, we adopt field survey research using repository mining of developer discussion forums (i.e., Stack Overflow) to answer \ref{rq:devs}, and classify these using both manual and automated techniques.

The second aspect of our research design involves non-empirical research, which explores a design-based question \citep{Simon:1996uw} to answer \ref{rq:design}. As the answers to our first three RQs establish a greater understanding of the nature behind \glspl{cvs} from various perspectives, the strategies we design in \ref{rq:design} aims at designing more reliable integration methods so that developers can better use these cloud-based services in their applications.