\section{Research Goals}

\itshape
In this thesis, we explore the effect stochastic and probabilistic systems play on the usability of \glspl{api} with respect to computer vision \glspl{cis}. Our perspective is software quality---specifically, validation and verification---within such systems and what best practices within the field of software engineering can be applied to assist in operationalisation such systems.
\upshape

The goals of this study aim to provide a snapshot of current developer best practices towards the usage of \glspl{cis} to provide a guiding framework and recommendations for software developers and \glspl{cis} providers alike. We propose two major bodies of work.

\subsubsection*{Goal 1: \textit{Understand the developer's mindset towards selecting a computer vision \gls{cis}}.}

\subsubsection*{Goal 2: \textit{Determine what quality factors affect software built using computer vision \glspl{cis}}.}

\subsubsection*{Goal 3: \textit{Provide an evaluation framework developers wishing to use computer vision \glspl{cis}.}}

Chiefly, we can specify the following high-level research questions:

\begin{enumerate}[label=\textbf{RQ\arabic*}., leftmargin=4\parindent]
  \item How do software engineers understand and evaluate computer vision \glspl{cis} for use in both generic and specific applications?
  \item Do software engineers follow best practices when evaluating computer vision \glspl{cis}? How does this compare to actual practice?
  \item What is needed to improve the state-of-the-art of computer vision \glspl{cis} in terms of \gls{api} documentation?
  \item What aspects of validation and verification can be improved in the field of computer vision \glspl{cis}?
\end{enumerate}

Ultimately, we seek to understand the conceptual understanding of software engineers who operationalise stochastic and probabilistic systems, and furthermore understand knowledge representation with these systems' \gls{api} documentation. Our motivation is to provide insight into current practices and compare the best practices with actual practise. We strive for this to  provide developers with a guiding framework on how to best operationalise these systems via the form of some checklist or tool they can use to ensure optimal software quality.

It is anticipated that the findings from this study in the computer vision \glspl{cis} space will be generalisable to other areas, such as time-series information, natural language processing and others.

%  Paper 2:
%* RQ1. How do software engineers evaluate (knowledge representation) machine learning APIs for use in an application?
%   * Motivation: to provide insights into the current practice
%   * Method: Survey
%* RQ2. Do software engineers follow best practices when evaluating machine learning APIs?
%   * Motivation: to compare best practice with actual practice
%   * Survey