\section{Research Outcomes}
\label{sec:introduction:hypohtesis}

The goals of this study aim to provide a snapshot of current developer practices towards the usage of \glsplx{iws}, specifically the subset that provides \glslong{cv} intelligence referred to as \glsplx{cvs}. Our anchoring perspective is software quality---specifically, validation and verification---and the quality attributes of effectiveness and reliability, where we explore  what best practices within the field of software engineering can be applied to assist in consumption of \glspl{iws}.

\ifdefined\review\else\begin{framed}\fi{
\itshape
\noindent
\textbf{This thesis investigates:} (i)~specific characteristics of \glspl{cvs} that may adversely affect internal and external quality of applications consuming them; (ii)~the nature of \gls{api} documentation knowledge, what should be documented, and which documentation attributes \gls{iws} can better express in their own \gls{api} documentation; (iii)~pain-points developers face when developing with \gls{cvs}; and (iv)~several solutions to help mitigate these issues.
}\ifdefined\review\else\end{framed}\fi

Based on the motivating case studies in \cref{sec:introduction:motivation}, we articulate three Research Hypotheses (RH1--3) below and seven Research Questions (RQs) based on  both empirical and non-empirical software engineering methodology \citep{Shull:2007vh,Simon:1996uw}, further discussed in \cref{ch:research-methodology}.

\newcommand{\rh}[1]{\hyperref[rh#1]{RH#1}}
\newcommand{\qualattr}[1]{{\upshape\sffamily\itshape\scshape~#1~}}

\ifdefined\review\else\begin{titled-frame}\fi
{$\blacksquare$~\bfseries \underline{RH1}: \textit{Existing \glspl{cvs} present \qualattr{ineffective} \gls{api} documentation.}}
\phantomsection
\label{rh1}
\vspace{-12pt}
\paragraph{Research Hypothesis}
\gls{api} documentation of \glspl{cvs} are inadequate and insufficient given the disparity of mindsets between the application developers and \gls{cvs} providers. Chiefly, application developers have limited general understanding of the `magic' that occurs behind these probabilistic `intelligent' \glspl{api}. We do not know what key aspects of the documentation matter to them, nor what they do or do not understand of the existing documentation. Applying rigorous \gls{se} documentation knowledge approaches, as defined from existing literature and aligned to the practical needs of software engineers, can help improve existing \glspl{cvs} documentation knowledge.

\paragraph{Research Goal}
To improve the \qualattr{effectiveness} of the documentation in existing \gls{iws} providers, specifically of \gls{cvs} \glspl{api}, by using existing \gls{se} API documentation knowledge aligned to the needs of developers.

\paragraph{Research Questions}
\begin{enumerate}[label=\textbf{RQ1.\arabic*.}, ref=RQ1.\arabic*, leftmargin=3.5\parindent, rightmargin=1\parindent]
  \item What \textit{guidelines} do a range of \gls{se} literature into \gls{api} documentation knowledge suggest \gls{api} providers document, and how is it best studied? \todo{This is answered by the ESEM paper}
  \label{rqs:apidoc:what-is-in-use}% KQ: Exploratory question to understand phenomena: Description and Classification question.
  
  \item How \textit{well} do \gls{cvs} providers follow these guidelines in documenting their own \glspl{api}, and what additional information or attributes would developers prefer to be included in the \gls{cvs} \gls{api} documentation? \todo{This is answered by my assessment against the three CV services against the ESEM taxonomy AND triangulated against the survey that finds out what developers find practically important---HOWEVER this is NOT YET PUBLISHED or WRITTEN UP (i.e., 20\% extension to my ESEM paper as suggested).}
  \label{rqs:apidoc:what-additional-information-needed}% DQ: Design question
 
  \item How do developers currently \textit{comprehend} \gls{cvs} given a lack of formal training in \gls{ai}? What are their pain-points arise as a result and what key aspects of the \gls{api} documentation matter do developers as they see it? \todo{This is answered by the ICSE paper, based on an analysis of Stack Overflow complaints.}% KQ: Base-rate question to understand normal patterns of the occurrence of the phenomena: Descriptive-Process questions.
  \label{rqs:apidoc:how-do-devs-understand-it}
\end{enumerate}

\paragraph{Research Contribution} An intelligent service \gls{api} documentation quality assessment framework to evaluate how well existing \glspl{cvs} have been documented for software engineers to use, and where improvements can be made.
\ifdefined\review\else\end{titled-frame}\fi

\ifdefined\review\else\begin{titled-frame}\fi
{$\blacksquare$~\bfseries \underline{RH2}: \textit{Existing \glspl{cvs} are \qualattr{unreliable} for context-specific use.} }
\phantomsection
\label{rh2}
\vspace{-12pt}
\paragraph{Research Hypothesis}
\glspl{cvs} are non-deterministic, evolving, and inconsistent. They are not yet ready to be operationalised into large-scale business-driven applications and, thus, additional protections, guards and development workflows must be considered during pre-development, development, testing and production to protect applications from such concerns.

\paragraph{Research Goal}
To improve the \qualattr{reliability} of applications built using \gls{api} endpoints of \glspl{iws} by guarding for domain-specific considerations of the application.

\paragraph{Research Questions}
\begin{enumerate}[label=\textbf{RQ2.\arabic*.}, ref=RQ2.\arabic*, leftmargin=3.5\parindent, rightmargin=1\parindent]
  \item What current reliability problems present in \glspl{cvs}? \todo{This is answered with ICSME paper}
  % KQ; Exploratory question to understand phenomena: Description and Classification question
  \label{rqs:metadata:what-problems-du                                                                e-to-lack-of-metadata}
    
  \item What additional tooling and workflows can be implemented when designing context-specific applications against \glspl{cvs}? \todo{This is answered with Threshy + ASE paper (and possibly Tomohiro's work).}
  \label{rqs:metadata:what-metadata-do-devs-want-and-why}  % DQ: design question
\end{enumerate}

\paragraph{Research Contributions} A new architectural design that can benchmark a service against a specific, application-specific dataset and then identify if substantial evolution has occurred, thus mitigating the \glspl{cvs} from having unintended side effects when the application is deployed in production.
\ifdefined\review\else\end{titled-frame}\fi

\noindent
\todo{Considering removing RH3... I do not have publications to support this}
\color{red}

\ifdefined\review\else\begin{titled-frame}\fi
{$\blacksquare$~\bfseries \underline{RH3}: \textit{\rh{1} and \rh{2} improve quality,  productivity or developer informativeness.}}
\phantomsection
\label{rh3}
\vspace{-12pt}
\paragraph{Research Hypothesis}
The implication of hypotheses 1 and 2 suggest that improving both the documentation and providing further metadata will improve product quality (internal or external), and/or developer productivity and/or developer education in developing software with intelligent components.

\paragraph{Research Goal}
 To confirm if improvements to \gls{api} documentation and response metadata  are reflected as improvements to product quality, developer productivity and/or developer education.

\paragraph{Research Questions}
\begin{enumerate}[label=\textbf{RQ3.\arabic*.}, ref=RQ3.\arabic*, leftmargin=3.5\parindent, rightmargin=1\parindent]
  \item  Does an improvement of documentation or metadata correlate to an improvement in software quality, developer productivity and/or developer informativeness?
  % KQ; relationship question to understand the correlation between two phenomena
  \label{rqs:implications:do-metrics-improve}
  
  \item With respect to \ref{rqs:implications:do-metrics-improve}, the three aspects are explored:
  % KQ; causality questions...
  \begin{enumerate}
  \item Does the improvement cause increased product quality, as measured through improved external quality metrics?
  \item Does the improvement cause increased developer productivity, as measured through improved internal quality metrics?
  \item Does the improvement cause increased developer informativeness or increased confidence in developing \gls{iws}-powered applications?
  \end{enumerate}
  \label{rqs:implications:aspects}  
\end{enumerate}

\paragraph{Research Contribution}
A concrete sample solution or framework that improves such metrics, thereby confirming that our documentation and metadata quality assessment framework improves these facets.
\ifdefined\review\else\end{titled-frame}\fi
\color{black}

\section{Thesis Contributions}

This thesis offers four major contributions:

\begin{enumerate}[label=(\roman*),leftmargin=2\parindent]
  \item an improved systems integration strategy to better incorporate off-the-shelf \gls{ai} components into end-applications;
  \item general recommendations for application developers and \glspl{iws} providers to alleviate developer productivity issues when working with \glspl{cvs};
  \item an architectural facade strategy designed to guard against potential evolution risk within these services; and,
  \item support tooling and software development workflows to assist in alleviating issues arising from poor documentation.
\end{enumerate}

%\section{Concluding Remarks}
%
%Ultimately, we seek to understand the conceptual understanding of software engineers who operationalise stochastic and probabilistic systems, and furthermore understand knowledge representation with these systems' \gls{api} documentation. Our motivation is to provide insight into current practices and compare the best practices with actual practise. We strive for this to  provide developers with a guiding framework on how to best operationalise these systems via the form of some checklist or tool they can use to ensure optimal software quality.
%
%It is anticipated that the findings from this study in the \glspl{cvs} space will be generalisable to other areas, such as time-series information, natural language processing and others.

%  Paper 2:
%* RQ1. How do software engineers evaluate (knowledge representation) machine learning APIs for use in an application?
%   * Motivation: to provide insights into the current practice
%   * Method: Survey
%* RQ2. Do software engineers follow best practices when evaluating machine learning APIs?
%   * Motivation: to compare best practice with actual practice
%   * Survey