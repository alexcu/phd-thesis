\section{Research Goals}
\label{sec:introduction:hypohtesis}

\Cref{ssec:introduction:motivation:scenario} describes two instances where an \gls{iws} can have either minor or substantiative impact when the software engineers developing with them misunderstand (or unappreciate) the nuances of \gls{ml}. This hinders the developer experience, productivity, and understanding/appreciation of \gls{ai}-based components. Thus, both the internal and external quality of applications using these components may have real-world consequences.

The goals of this study aim to provide a snapshot of current developer practices towards the usage of \glsplx{iws}, specifically the subset that provides \glslong{cv} intelligence referred to as \glsplx{cvs}. Our anchoring perspective is software quality---specifically, validation and verification---and the quality attributes of effectiveness and reliability, where we explore  how best practices within the field of software engineering apply in the consumption of \glspl{iws}.

\begin{callout}
Within the context of \gls{cvs}, this thesis investigates: (i)~specific characteristics that influence the internal and external quality of applications consuming \glspl{iws}; (ii)~the nature of \gls{api} documentation knowledge and how best to document \gls{cvs}; (iii)~pain-points developers face when developing with \gls{cvs}.
\end{callout}

Based on the motivating case studies in \cref{sec:introduction:motivation}, we articulate three Research Hypotheses (RH1--3) below and seven Research Questions (RQs) based on  both empirical and non-empirical software engineering methodology \citep{Shull:2007vh,Simon:1996uw}, further discussed in \cref{ch:research-methodology}.


\newcommand{\rh}[1]{\hyperref[rh#1]{RH#1}}
\subsection[Research Hypothesis 1]{\underline{RH1}: \glspl{iws} indirectly influences software quality.}
\label{rh1}

\paragraph{Null Hypothesis}
Large-scale, domain-specific, production systems maintain good reliability (further discussed in \cref{ch:background}), an attribute of good quality software. Reliable software is consistent, fault-tolerant and deterministic. \Glspl{iws} are, therefore, reliable and ready for large-scale use.

\paragraph{Research Claim}
Additional protections, guards and development workflows during pre-development, development, testing and production protect applications maintain internal and external reliability.

\paragraph{Research Goal}
To improve the reliability of applications built using \gls{api} endpoints of \glspl{iws} by guarding for domain-specific considerations of the application. We aim to achieve this in the context of \gls{cvs}.

\paragraph{Research Questions}
\begin{enumerate}[label=\textbf{RQ2.\arabic*.}, ref=RQ1.\arabic*, leftmargin=3.5\parindent, rightmargin=1\parindent]
  \item Are \glspl{cvs} consistent between services given the same domain-specific subject matter?
  \item What long-term attributes do \gls{cvs} present and do these attributes impact software quality?
\end{enumerate}

\paragraph{Research Contributions} A novel architectural design and development workflow designed for consuming \glspl{cvs} to help monitor and guard against unintended post-production side effects.

\subsection[Research Hypothesis 2]{\underline{RH2}: \glspl{iws} indirectly influences developer experience}
\label{rh2}


% Null hypothesis on DevX and unintended side effects, productivity etc.
%  \item What concerns do developers face when using \gls{cvs}, and how does this compare to more established \gls{se} fields?

\paragraph{Hypothesis Description}
\gls{api} documentation of \glspl{cvs} present problems given the disparity of mindsets between the application developers (deterministic) and \gls{cvs} providers (nondeterministic). Chiefly, application developers have limited general understanding of the `magic' that occurs behind these probabilistic `intelligent' \glspl{api}. We do not know what key aspects of the documentation matter to them, nor what they do or do not understand of the existing documentation. Applying rigorous \gls{se} documentation knowledge approaches, as defined from existing literature and aligned to the practical needs of software engineers, can help improve existing \glspl{cvs} documentation knowledge.

\paragraph{Research Goal}
To improve the \qualattr{effectiveness} of the documentation in existing \gls{iws} providers, specifically of \gls{cvs} \glspl{api}, by using existing \gls{se} API documentation knowledge aligned to the needs of developers.

\paragraph{Research Questions}
\begin{enumerate}[label=\textbf{RQ1.\arabic*.}, ref=RQ1.\arabic*, leftmargin=3.5\parindent, rightmargin=1\parindent]
  \item What \textit{guidelines} do a range of \gls{se} literature into \gls{api} documentation knowledge suggest \gls{api} providers document, and how is it best studied? \todo{This is answered by the ESEM paper}
  \label{rqs:apidoc:what-is-in-use}% KQ: Exploratory question to understand phenomena: Description and Classification question.
  
  \item How \textit{well} do \gls{cvs} providers follow these guidelines in documenting their own \glspl{api}, and what additional information or attributes would developers prefer to be included in the \gls{cvs} \gls{api} documentation? \todo{This is answered by my assessment against the three CV services against the ESEM taxonomy AND triangulated against the survey that finds out what developers find practically important---HOWEVER this is NOT YET PUBLISHED or WRITTEN UP (i.e., 20\% extension to my ESEM paper as suggested).}
  \label{rqs:apidoc:what-additional-information-needed}% DQ: Design question
 
  \item How do developers currently \textit{comprehend} \gls{cvs} given a lack of formal training in \gls{ai}? What are their pain-points arise as a result and what key aspects of the \gls{api} documentation matter do developers as they see it? \todo{This is answered by the ICSE paper, based on an analysis of Stack Overflow complaints.}% KQ: Base-rate question to understand normal patterns of the occurrence of the phenomena: Descriptive-Process questions.
  \label{rqs:apidoc:how-do-devs-understand-it}
\end{enumerate}

\paragraph{Research Contribution} An intelligent service \gls{api} documentation quality assessment framework to evaluate how well existing \glspl{cvs} have been documented for software engineers to use, and where improvements can be made.

\section{Thesis Contributions}

This thesis offers four major contributions:

\begin{enumerate}[label=(\roman*),leftmargin=2\parindent]
  \item an improved systems integration strategy to better incorporate off-the-shelf \gls{ai} components into end-applications;
  \item general recommendations for application developers and \glspl{iws} providers to alleviate developer productivity issues when working with \glspl{cvs};
  \item an architectural facade strategy designed to guard against potential evolution risk within these services; and,
  \item support tooling and software development workflows to assist in alleviating issues arising from poor documentation.
\end{enumerate}

%\section{Concluding Remarks}
%
%Ultimately, we seek to understand the conceptual understanding of software engineers who operationalise stochastic and probabilistic systems, and furthermore understand knowledge representation with these systems' \gls{api} documentation. Our motivation is to provide insight into current practices and compare the best practices with actual practise. We strive for this to  provide developers with a guiding framework on how to best operationalise these systems via the form of some checklist or tool they can use to ensure optimal software quality.
%
%It is anticipated that the findings from this study in the \glspl{cvs} space will be generalisable to other areas, such as time-series information, natural language processing and others.

%  Paper 2:
%* RQ1. How do software engineers evaluate (knowledge representation) machine learning APIs for use in an application?
%   * Motivation: to provide insights into the current practice
%   * Method: Survey
%* RQ2. Do software engineers follow best practices when evaluating machine learning APIs?
%   * Motivation: to compare best practice with actual practice
%   * Survey