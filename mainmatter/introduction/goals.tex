\section{Research Goals}
\label{sec:introduction:hypohtesis}

The goals of this study aim to provide a snapshot of current developer practices towards the usage of \glsplx{iws}, specifically the subset that provides \glslong{cv} intelligence referred to as \glsplx{cvs}. We identify the maturity, viability and risks of \glsplx{iws} through an anchoring perspective of internal and external software quality. 
As these services are proprietary, we are unable to conduct source code or model analysis, and hence are not used in the investigation of this thesis.
We explore how best practices within the field of \gls{se} apply in the consumption of \glspl{iws}.

\begin{callout}
Within the context of \gls{cvs}, this thesis investigates: (i)~dynamic runtime behaviour characteristics that influence internal and external quality of applications consuming \glspl{iws}; (ii)~the nature of \gls{api} documentation as provided by \glspl{cvs}; (iii)~developer struggles and perceptions of \glspl{cvs} as contrasted to mature \gls{se} domains.
\end{callout}

Based on the motivating case studies in \cref{sec:introduction:motivation}, we articulate two Research Hypotheses (RH1--2) below and nine Research Questions (RQs) based on  both empirical and non-empirical \gls{se} methodology \citep{Shull:2007vh,Simon:1996uw}, further discussed in \cref{ch:research-methodology}.

\newcommand{\rh}[1]{\hyperref[rh#1]{RH#1}}
\subsection[Research Hypothesis 1]{\underline{RH1}: \gls{iws} runtime behaviour influences external quality}
\label{rh1}

\paragraph{Null Hypothesis}
Large-scale, domain-specific, production systems maintain good reliability (further discussed in \cref{sec:background:software-quality}), an attribute of good quality software. Reliable software is consistent, fault-tolerant and deterministic. \Glspl{iws} are, therefore, reliable and ready for large-scale use.

\paragraph{Research Claim}
Additional protections, guards and development workflows during pre-development, development, testing and production protect applications maintain internal and external reliability.

\paragraph{Research Goal}
To improve the reliability of applications built using \gls{api} endpoints of \glspl{iws} by guarding for domain-specific considerations of the application. We aim to achieve this in the context of \gls{cvs}.

\paragraph{Research Questions}
\begin{enumerate}[label=\textbf{RQ1.\arabic*.}, ref=RQ1.\arabic*, leftmargin=3.5\parindent, rightmargin=1\parindent]
  \item \tonote{ICSME-RQ1} Do \glspl{cvs} provide consistent responses amongst services?
  \item \tonote{ICSME-RQ2} What long-term runtime behavioural issues arise when using \glspl{cvs}?
\end{enumerate}

\subsection[Research Hypothesis 2]{\underline{RH2}: Compression of \glspl{iws} influence internal quality}
\label{rh2}


% Null hypothesis on DevX and unintended side effects, productivity etc.
%  \item What concerns do developers face when using \gls{cvs}, and how does this compare to more established \gls{se} fields?

\paragraph{Null Hypothesis}
Good \gls{api} documentation and usability assists developers to write software more productively (see \cref{sec:background:api}). The communication of nondeterministic behaviour is sufficiently described in \gls{cvs} \glspl{api}, as reflected in developer community discussion.

% Chiefly, application developers have limited general understanding of the `magic' that occurs behind these probabilistic `intelligent' \glspl{api}. We do not know what key aspects of the documentation matter to them, nor what they do or do not understand of the existing documentation. Applying rigorous \gls{se} documentation knowledge approaches, as defined from existing literature and aligned to the practical needs of software engineers, can help improve existing \glspl{cvs} documentation knowledge.

\paragraph{Research Claim}
Pain-points on community discussion forums---such as \glslong{so}---reflect the clash of mindsets between application developers and  \gls{cvs} providers, and both poor documentation and education in the area results in lost productivity and poor internal quality.

\paragraph{Research Goal}
To improve the effectiveness of the usability and documentation of \gls{iws} \glspl{api}---specifically of \gls{cvs} \glspl{api}---by using existing \gls{se} \gls{api} documentation knowledge aligned to the needs of developers, and identifying how developers complaints differ to more established fields. 

\paragraph{Research Questions}
\begin{enumerate}[label=\textbf{RQ1.\arabic*.}, ref=RQ2.\arabic*, leftmargin=3.5\parindent, rightmargin=1\parindent]

  \item \tonote{ESEM-RQ2} How is \gls{api} documentation studied?
  \item \tonote{ESEM-RQ1} What knowledge guidelines do \gls{api} documentation studies contribute?
  \item \tonote{ESEMx-RQ1} To what extent do practitioners agree with the usability of this knowledge?
  \item \tonote{ESEMx-RQ2} To what extent do \gls{cvs} providers follow these guidelines in documenting their \glspl{api} \label{rqs:apidoc:what-is-in-use}% KQ: Exploratory question to understand phenomena: Description and Classification question.
  \item \tonote{ESEMx-RQ3} What additional information or attributes do developers need in \gls{cvs} \gls{api} documentation?
  \label{rqs:apidoc:what-additional-information-needed}% DQ: Design question
  \item \tonote{ICSE-RQ1} How do developers mis-comprehend \glspl{cvs}, as expressed as pain-points on \glslong{so}?% KQ: Base-rate question to understand normal patterns of the occurrence of the phenomena: Descriptive-Process questions.,
  \item \tonote{ICSE-RQ2} Are the distribution of these complaints different to established \gls{se} fields?
%  \label{rqs:apidoc:how-do-devs-understand-it}
\end{enumerate}



\section{Thesis Contributions}

This thesis offers six major contributions:

\begin{enumerate}[label=(\roman*),leftmargin=2\parindent]
  \item \tonote{ASE} a novel architectural facade to guard against unintended post-production side effects of \glspl{iws}; 
  \item \tonote{ASE} a reference implementation of the architecture within the context of \glspl{cvs};
  \item \tonote{ICSE-Demo} a tool to help fine-tune confidence thresholds against domain-specific datasets when using \glspl{iws}; 
  \item \tonote{ASE+ICSE-Demo} a software development workflow designed to evaluate and consume \glspl{cvs} in pre-development, development, testing and production;
  \item \tonote{ESEM} a documentation quality assessment taxonomy, based on \todo{number of primary studies} studies, to evaluate the quality of documentation of \glspl{api};
  \item \tonote{ESEMx+ICSE} knowledge for both educators and practitioners to improve understanding  \glspl{api} within the context of \glspl{cvs}.
\end{enumerate}

%\section{Concluding Remarks}
%
%Ultimately, we seek to understand the conceptual understanding of software engineers who operationalise stochastic and probabilistic systems, and furthermore understand knowledge representation with these systems' \gls{api} documentation. Our motivation is to provide insight into current practices and compare the best practices with actual practise. We strive for this to  provide developers with a guiding framework on how to best operationalise these systems via the form of some checklist or tool they can use to ensure optimal software quality.
%
%It is anticipated that the findings from this study in the \glspl{cvs} space will be generalisable to other areas, such as time-series information, natural language processing and others.

%  Paper 2:
%* RQ1. How do software engineers evaluate (knowledge representation) machine learning APIs for use in an application?
%   * Motivation: to provide insights into the current practice
%   * Method: Survey
%* RQ2. Do software engineers follow best practices when evaluating machine learning APIs?
%   * Motivation: to compare best practice with actual practice
%   * Survey