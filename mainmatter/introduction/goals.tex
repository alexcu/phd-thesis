\section{Research Goals}
\label{sec:introduction:hypohtesis}

\subsection{Research Overview}


\todo{ is it JUST CVS, or they one exemplar of intelligent APIs/services?? }
This thesis aims to investigate and better understand the nature of cloud-based \glsplx{cvs}.\footnote{As these services are proprietary, we are unable to conduct source code or model analysis, and hence are not used in the investigation of this thesis.} 
We identify the maturity, viability and risks of \glspl{cvs} through the anchoring perspective of \textit{reliability} that affects the internal and external quality of software. We adopt the McCall \citep{McCall:1977uy} and Boehm \citep{Boehm:1978vv} interpretations of reliability via the sub-characteristics of a service's \textit{consistency} and \textit{robustness} (or fault/error tolerance), and the \textit{completeness}\footnote{We treat the \gls{api} documentation of a \gls{cvs} as a first-class citizen.} of its documentation. (A detailed discussion is further provided in \cref{sec:background:software-quality}.)
This thesis explores and contributes towards \textit{four} key facets regarding reliability in \gls{cvs} usage and the completeness of its associated documentation. We formulate four primary research questions (RQs) with six sub-RQs, based on both empirical and non-empirical \glslong{se} methodology \citep{Shull:2007vh,Simon:1996uw}, further discussed in \cref{ch:research-methodology}.

\subsection{Research Questions}

Firstly, we investigate adverse implications that arise when using \glspl{cvs} that affects consistency and robustness~(\textbf{\cref{ch:icsme2019}}). We show how \glspl{cvs} have a non-deterministic runtime behaviour and evolve with unintended and non-trivial consequences to developers. We demonstrate that these services have inconsistent behaviour despite offering the same functionality and pose evolution risk that effects robustness of consuming applications when responses change given the same (consistent) inputs. Thus, we conclude how the nature of these services (at present) are not fully robust, consistent, and thus not reliable. Formally, we structure the following RQs:

\begin{leftbar}
\begin{enumerate}[label=\faQuestionCircle~~\textbf{RQ\arabic*.}, ref=RQ\arabic*, leftmargin=2.75\parindent, rightmargin=1\parindent]
    \item \textbf{What is the nature of cloud-based \gls{cvs}?}\label{rq:icsme}
    \begin{enumerate}[label=\textit{RQ1.\arabic*.}, ref=RQ1.\arabic*]
      \item What is their runtime behaviour?\label{rq:icsme:rq1}%KQ: Exploratory question to understand phenomena: Description and Classification question.
      \item What is their evolution profile?\label{rq:icsme:rq2}%KQ: Exploratory question to understand phenomena: Description and Classification question.
    \end{enumerate}
\end{enumerate}
\end{leftbar}

Secondly, we investigate the reliability of the documentation these services offer through the lenses of its completeness. We collate prior knowledge of good \gls{api} documentation and assess the efficacy of such knowledge against practitioners~(\textbf{\cref{ch:jss2020}}). We show that these service's behaviour and evolution is not reliably documented adequately against this knowledge. Formally, we develop the following RQs:

\begin{leftbar}
\begin{enumerate}[label=\faQuestionCircle~~\textbf{RQ\arabic*.}, ref=RQ\arabic*, leftmargin=2.75\parindent, rightmargin=1\parindent,start=2]
    \item \textbf{How complete is current \gls{cvs} \glspl{api} documentation?}\label{rq:esem}%KQ: [Exploratory question] to understand phenomena: Description and Classification question.
    \begin{enumerate}[label=\textit{RQ2.\arabic*.}, ref=RQ2.\arabic*]
      \item What are the dimensions of `\textit{complete}' \gls{api} documentation, according to both literature and practitioners?\label{rq:esem:rq1}%KQ: [Exploratory question] to understand phenomena: Description and Classification question.
      \item What additional information or attributes do developers need in \gls{cvs} \gls{api} documentation to make it more complete?\label{rq:esem:rq2}% Non-Empirical: [Design Question]
    \end{enumerate}
\end{enumerate}
\end{leftbar}

Thirdly, we investigate how software developers approach using these services and directly assess developer pain-points resulting from the nature of \glspl{cvs} and their documentation~(\textbf{\cref{ch:icse2020}}). We show that there is a statistically significant difference in these complaints when contrasted against more established \glslong{se} domains (such as web or mobile development) as expressed as posts on \glslong{so}. We provide a number of exploratory avenues for researchers, educators, software engineers and \gls{iws} providers to alleviate these complaints based on this analysis. We formulate the following RQs:

\begin{leftbar}
\begin{enumerate}[label=\faQuestionCircle~~\textbf{RQ\arabic*.}, ref=RQ\arabic*, leftmargin=2.75\parindent, rightmargin=1\parindent,start=3]
  \item \textbf{How does the developer's comprehension of \glspl{cvs} differ to conventional \glslong{se} domains?}\label{rq:icse} %KQ: [Exploratory question] to understand phenomena: Descriptive-Comparative question.
  \begin{enumerate}[label=\textit{RQ3.\arabic*.}, ref=RQ3.\arabic*]
    \item What aspects of \glspl{cvs} and its documentation do developers struggle with, as expressed as pain-points within \glslong{so} posts?\label{rq:icse:rq1}% %KQ: [Exploratory question] to understand phenomena: Description and Classification question.
    \item Is the distribution of the struggles within these posts different to established \glslong{cvs} fields?\label{rq:icse:rq2}% KQ: [Relationship Question] between two different phenomenon to assess relations
    \item Are developers frustrated with \glspl{cvs}? What emotions do they express? \label{rq:semotion}
  \end{enumerate}
\end{enumerate}
\end{leftbar}


Lastly, based on our analysis of the documentation, runtime and evolution profiles of \glspl{cvs}, in conjunction with our assessment of developer complaints, we developed two specific solutions to address documentation weaknesses and another to ensure greater reliability when developing using \gls{cvs}~(\textbf{\cref{ch:icse-demo2020,ch:fse2020}}). Our first solution, via synthesis of a documentation template, proposes recommendations on where current \gls{cvs} documentation needs improvement to make the documentation more reliable. Our second solution consists of an integration architecture style (or facade) that guards against the runtime and evolution issues identified of \glspl{cvs} to make integrating with a \gls{cvs} more reliable. Our final RQ is:

\begin{leftbar}
\begin{enumerate}[label=\faQuestionCircle~~\textbf{RQ\arabic*.}, ref=RQ\arabic*, leftmargin=2.75\parindent, rightmargin=1\parindent,start=4]
  \item \textbf{What strategies can developers use to integrate with and use \glspl{cvs}, while preserving robustness and reliability?}\label{rq:fse}% Non-Empirical: [Design Question]
\end{enumerate}
\end{leftbar}

\todo{ you might want to include a section on your methodology here - so the flow is like questions and gaps, methodology, and contributions}

\section{Research Methodology}
\todo{Short section on research methodology... greater described in chapter}

%
%    \item 
%    \begin{enumerate}[label=\textbf{RQ2.\arabic*.}, ref=RQ2.\arabic*]
%      \item What makes good \gls{api} documentation, accordingly to both literature and practitioners?
%      \item To what extent do \gls{cvs} providers follow these guidelines in documenting their own \glspl{api}?
%      \item What additional information or attributes do developers need in \gls{cvs} \gls{api} documentation?
%    \end{enumerate}
%    \item How does the nature of \gls{cvs} and quality of \gls{cvs} \gls{api} documentation reflect in developer circles?
%    \begin{enumerate}[label=\textbf{RQ3.\arabic*.}, ref=RQ3.\arabic*]
%      \item How do developers mis-comprehend \gls{cvs}, as expressed ?
%    \end{enumerate}
%    \item How can developers integrate with and use \glspl{cvs} while preserving reliability?
%
%\end{enumerate}

%
%Research questions:
% 
%   (ii) how developers currently approach using these services; and (iii) where their existing documentation artefacts need improvement. 
%
%In this thesis, we show that . We also show that these 
%
%\begin{callout}
%This thesis aims to investigate and better understand the nature of cloud-based \glspl{cvs}.\end{callout}
%
%Based on the motivating case studies in \cref{sec:introduction:motivation}, our investigation 
%
%
% we articulate two Research Hypotheses (RH1--2) below 
%
%
%
%
%
%
\newcommand{\rh}[1]{\hyperref[rh#1]{RH#1}}
%\subsection[Research Hypothesis 1]{\underline{RH1}: \gls{iws} runtime behaviour influences external quality}
%\label{rh1}
%
%\paragraph{Null Hypothesis}
%Large-scale, domain-specific, production systems maintain good reliability (further discussed in \cref{sec:background:software-quality}), an attribute of good quality software. Reliable software is consistent, fault-tolerant and deterministic. \Glspl{iws} are, therefore, reliable and ready for large-scale use.
%
%\paragraph{Research Claim}
%Additional protections, guards and development workflows during pre-development, development, testing and production protect applications maintain internal and external reliability.
%
%\paragraph{Research Goal}
%To improve the reliability of applications built using \gls{api} endpoints of \glspl{iws} by guarding for domain-specific considerations of the application. We aim to achieve this in the context of \gls{cvs}.
%
%\paragraph{Research Questions}
%\begin{enumerate}[label=\textbf{RQ1.\arabic*.}, ref=RQ1.\arabic*, leftmargin=3.5\parindent, rightmargin=1\parindent]
%  \item \tonote{ICSME-RQ1} Do \glspl{cvs} provide consistent responses amongst services?
%  \item \tonote{ICSME-RQ2} What long-term runtime behavioural issues arise when using \glspl{cvs}?
%\end{enumerate}
%
%\subsection[Research Hypothesis 2]{\underline{RH2}: Comprehension of \glspl{iws} influence internal quality}
%\label{rh2}
%
%
%% Null hypothesis on DevX and unintended side effects, productivity etc.
%%  \item What concerns do developers face when using \gls{cvs}, and how does this compare to more established \gls{se} fields?
%
%\paragraph{Null Hypothesis}
%Good \gls{api} documentation and usability assists developers to write software more productively (see \cref{sec:background:api}). The communication of nondeterministic behaviour is sufficiently described in \gls{cvs} \glspl{api}, as reflected in developer community discussion.
%
%% Chiefly, application developers have limited general understanding of the `magic' that occurs behind these probabilistic `intelligent' \glspl{api}. We do not know what key aspects of the documentation matter to them, nor what they do or do not understand of the existing documentation. Applying rigorous \gls{se} documentation knowledge approaches, as defined from existing literature and aligned to the practical needs of software engineers, can help improve existing \glspl{cvs} documentation knowledge.
%
%\paragraph{Research Claim}
%Pain-points on community discussion forums---such as \glslong{so}---reflect the clash of mindsets between application developers and  \gls{cvs} providers, and both poor documentation and education in the area results in lost productivity and poor internal quality.
%
%\paragraph{Research Goal}
%To improve the effectiveness of the usability and documentation of \gls{iws} \glspl{api}---specifically of \gls{cvs} \glspl{api}---by using existing \gls{se} \gls{api} documentation knowledge aligned to the needs of developers, and identifying how developers complaints differ to more established fields. 
%
%\paragraph{Research Questions}
%\begin{enumerate}[label=\textbf{RQ2.\arabic*.}, ref=RQ2.\arabic*, leftmargin=3.5\parindent, rightmargin=1\parindent]
%
%  \item \tonote{ESEM-RQ2} How is \gls{api} documentation studied?
%  \item \tonote{ESEM-RQ1} What knowledge guidelines do \gls{api} documentation studies contribute?
%  \item \tonote{ESEMx-RQ1} To what extent do practitioners agree with the usability of this knowledge?
%  \item \tonote{ESEMx-RQ2} To what extent do \gls{cvs} providers follow these guidelines in documenting their \glspl{api} \label{rqs:apidoc:what-is-in-use}% KQ: Exploratory question to understand phenomena: Description and Classification question.
%  \item \tonote{ESEMx-RQ3} What additional information or attributes do developers need in \gls{cvs} \gls{api} documentation?
%  \label{rqs:apidoc:what-additional-information-needed}% DQ: Design question
%  \item \tonote{ICSE-RQ1} How do developers mis-comprehend \glspl{cvs}, as expressed as pain-points on \glslong{so}?% KQ: Base-rate question to understand normal patterns of the occurrence of the phenomena: Descriptive-Process questions.,
%  \item \tonote{ICSE-RQ2} Are the distribution of these complaints different to established \gls{se} fields?
%%  \label{rqs:apidoc:how-do-devs-understand-it}
%\end{enumerate}
%
%
%
%\section{Thesis Contributions}
%
%Based on the findings and recommendations of the studies conducted in this thesis, we offer a two-fold solution that resolves critical issues arising from our investigation. The first is a  technical solution while the second offers an educational perspective on improvements in the understanding of \glspl{iws}. The thesis offers six major contributions:
%
%\begin{enumerate}[label=(\roman*),leftmargin=2\parindent]
%  \item \tonote{ASE} a novel architectural facade to guard against unintended post-production side effects of \glspl{iws}; 
%  \item \tonote{ASE} a reference implementation of the architecture within the context of \glspl{cvs};
%  \item \tonote{ICSE-Demo} a tool to help fine-tune confidence thresholds against domain-specific datasets when using \glspl{iws}; 
%  \item \tonote{ASE+ICSE-Demo} a software development workflow designed to evaluate and consume \glspl{cvs} in pre-development, development, testing and production;
%  \item \tonote{ESEM} a documentation quality assessment taxonomy, based on \todo{number of primary studies} studies, to evaluate the quality of documentation of \glspl{api};
%  \item \tonote{ESEMx+ICSE} knowledge for both educators and practitioners to improve understanding  \glspl{api} within the context of \glspl{cvs}.
%\end{enumerate}
%
%%\section{Concluding Remarks}
%%
%%Ultimately, we seek to understand the conceptual understanding of software engineers who operationalise stochastic and probabilistic systems, and furthermore understand knowledge representation with these systems' \gls{api} documentation. Our motivation is to provide insight into current practices and compare the best practices with actual practise. We strive for this to  provide developers with a guiding framework on how to best operationalise these systems via the form of some checklist or tool they can use to ensure optimal software quality.
%%
%%It is anticipated that the findings from this study in the \glspl{cvs} space will be generalisable to other areas, such as time-series information, natural language processing and others.
%
%%  Paper 2:
%%* RQ1. How do software engineers evaluate (knowledge representation) machine learning APIs for use in an application?
%%   * Motivation: to provide insights into the current practice
%%   * Method: Survey
%%* RQ2. Do software engineers follow best practices when evaluating machine learning APIs?
%%   * Motivation: to compare best practice with actual practice
%%   * Survey