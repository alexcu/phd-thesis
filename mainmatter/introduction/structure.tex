\section{Thesis Organisation}
\label{sec:introduction:organisation}

We organise the thesis into four parts. \textbf{\Cref{part:preface}, (\textit{The Preface})} includes introductory and background chapters. This is a \textit{PhD by Publication}, and \textbf{\cref{part:publications} (\textit{Publications})} comprises of the six publications resulting from this work over \Cref{ch:icsme2019,ch:esem2019,ch:icse2020,ch:icse2020-demo,ch:ase2020}; publications are included verbatim except for terminology and formatting changes to better fit the suitability of a coherent thesis. \textbf{\Cref{part:postface}, (\textit{The Postface})} includes the conclusion and future works chapter, as well as a list of academic studies and online artefacts referenced within the thesis. \textbf{\Cref{part:appendices}, (\textit{Appendices})} includes all supplementary material, including mandatory authorship statements and ethics approval. Details of each chapter following this introductory chapter are provided in the following section.

\subsection{Overview of Chapters}

\subsubsection{\Cref{part:preface}: Preface}

\paragraph{\cref{ch:background}: Background} This chapter provides an overview of prior studies broadly around three key pillars: the development of an \gls{iws}, the usage of an \gls{iws}, and the nature of an \gls{iws}. We use the three perspectives of software quality (particularly, reliability), probabilistic and non-deterministic systems, and explanation and communication theory to describe prior work.

\paragraph{\cref{ch:research-methodology}: Research Methodology} This chapter provides a summative review of research methods and philosophical stances relevant to \glslong{se}. We illustrate that the methods used within our publications are sound via an analysis of the methodologies used in seminal works referenced in this thesis.

\subsubsection{\Cref{part:publications}: Publications}

\paragraph{\cref{ch:icsme2019}: Exploring the nature of \glspl{cvs}} This chapter was presented at the 2019 International Conference on Software Maintenance and Evolution (ICSME)~\citep{Cummaudo:2019va}. We describe an 11-month longitudinal experiment assessing the behavioural (run-time) issues of three popular \glspl{cvs}: Google Cloud Vision~\citepweb{GoogleCloud:Home}, Amazon Rekognition~\citepweb{AWS:Home} and Azure Computer Vision~\citepweb{Azure:Home}. By using three different data sets---two of which we curate as additional contributions---we demonstrate how the services are inconsistent amongst each other and within themselves. This study provides a detailed answer to \ref{rq:icsme}: Despite presenting conceptually-similar functionality, each service behaves and produces slightly varied (inconsistent) results and demonstrates non-deterministic runtime behaviour. We discuss potential evolution risks to consumers of such services as the services provide non-static outputs for the same inputs, thereby having significant impact to the robustness of consuming applications. Further details in the study include a brief assessment into the lack of sufficient detail of these concerns in their documentation.

\paragraph{\cref{ch:icsme2019}: Investigating improvements to \gls{cvs} \gls{api} documentation} This chapter was originally a short paper presented at the 2019 International Symposium on Empirical Software Engineering and Measurement (ESEM) \citep{Cummaudo:2019th}. To understand where to improve \gls{cvs} documentation, we first need to investigate \textit{what} makes a good \gls{api} document. This short paper initially answered one aspect of \ref{rq:esem:rq1}: what \textit{academic literature} suggests a good (complete) \gls{api} document should comprise of. By conducting an \glslong{sms} resulting in 21 primary studies, we systematically develop a taxonomy that combines the recommendations of scattered work into a structured framework of 5 dimensions and 34 weighted categorisations. We then extend this work\footnote{The extended version of this chapter has been submitted to \todo[the Journal of Systems and Software]{Revise} and is currently within review.} by triangulating the taxonomy with opinions from developers using the \glslong{sus} to assess the efficacy of these recommendations (thereby answering the second aspect of \ref{rq:esem:rq1}). From this, we assess the how well \gls{cvs} providers document their \glspl{api} via a heuristic validation of the taxonomy, using the three services from the ICSME publication to make recommendations where documentation should be more complete, thereby answering \ref{rq:esem:rq2} (and thus \ref{rq:esem}).

\paragraph{\cref{ch:icse2020}: Understanding developer struggles when using \glspl{cvs}} This chapter has been accepted for presentation at the 2020 International Conference on Software Engineering (ICSE) \citep{Cummaudo:2019vi}. We conduct a mining study of 1,425 \glslong{so} posts that provide indications of the types frustrations that developers face when integrating \glspl{cvs} into their applications. To gather what their pain-points are, we use two classification taxonomies that also use \glslong{so} to understand generalised and documentation-specific pain-points in mature \gls{se} domains. This study answers \ref{rq:icse} in detail and provides a validation to our motivation of \ref{rq:esem}: we validate that the \textit{completeness} of current \gls{cvs} \gls{api} documentation is a main concern for developers and there is insufficient explanation into the errors and limitations of the service. We find that the documentation does not adequately cover all aspects of the technical domain. In terms of integrating with the service, developers struggle most with simple errors and ways in which to use the \glspl{api}; this is in stark contrast to mature software domains. Our interpretation is that developers fail to understand the \gls{iws} lifecycle and the `whole' system that wraps such services. We also interpret that developers have a shallower understanding of the core issues within \glspl{cvs} (likely due to the nuances of \gls{ml} as suggested in a discussion in the paper), which warrants an avenue for future work in \glslong{se} education.

\paragraph{\cref{ch:icse2020-demo}: Developing a confidence thresholding tool} This chapter has been submitted to the demonstrations track of the ICSE 2020. In this paper, we develop a tool---Threshy---that helps alleviate one of the primary issues in integrating \gls{cvs}: selecting an appropriate confidence threshold in a particular use-case. 

\paragraph{\cref{ch:ase2020}: Developing a \gls{cvs} integration architecture} Discusses evaluation strategies used to assess the accuracy of our processing pipeline, and presents the implications and limitations found from the results of our findings.

\subsubsection{\Cref{part:postface}: Postface}

\paragraph{\cref{ch:conclusion} - Conclusions} Draws conclusions and alleviates gaps in the findings of this work by proposing future studies.

\subsubsection{\Cref{part:appendices}: Appendices}

\subsection{Coherency of Publications}

In this section, we detail how each publication forms a coherent body of work, thus, illustrates the overall `journey' of this PhD. A recommendation arising from this study was to 

\subsection{Organisation}

In this introductory chapter I have motivated this research and given an overview of it. As noted earlier, the body (i.e., Chapters 2-11) of this dissertation comprise the published papers that have resulted from this research. The papers are presented in chronological order, by date of publication, verbatim as they were accepted for publication by the referees (except for changes to formatting and bibliographic references, which have been consolidated to all use the same style). In the final chapter of this dissertation—Chapter 12 Conclusions and Future Work—I review the contributions of this work, its significance and relevance, I evaluate its outcomes in terms of my stated goals for it, I self-identify some possible criticisms of it, and I discuss future directions of this work, which despite the many works it has led to, there are still many.

