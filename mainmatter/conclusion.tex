\chapter{Conclusions \& Future Work}
\label{ch:conclusions}

\glsunset{api}
\glsunset{iws}


In this chapter, we provide a summary of the contributions within the body of this work. We evaluate the significance of the research outcomes to the software engineering research community, and identify potential criticisms of these outcomes. Lastly, we indicate future avenues of research resulting from this thesis and provide concluding remarks.

\section{Contributions of this Work}

The context of this thesis explored cloud-based services that abstract pre-trained \glsac{ml} models behind \glspl{api} (i.e., \glslongpl{iws} or \glsacpl{iws}). Specifically, we used services that provide computer vision capability (i.e., \glslongpl{cvs} or \glsacpl{cvs}) as a concrete example. In summary, this thesis offers three primary contributions to the body of software engineering knowledge:
\begin{enumerate}[label=\textbf{(\roman*)}]
	\item \textbf{We provided a landscape analysis of \gls{cvs} behaviour and usage patterns.} We analysed their evolutionary patterns and non-deterministic runtime behaviour, thereby highlighting the potential pitfalls to overall solution robustness, and showed how the types of issues faced by developers are substantially different to conventional software components.
	\item \textbf{We demonstrated how developers require more complete service documentation.} We synthesised a taxonomy detailing the requirements of good \glsac{api} documentation which we validated against practitioners and three popular \gls{cvs} providers. From this, we provided a set of documentation artefacts that should be documented to improve the completeness of \gls{cvs} documentation.
	\item \textbf{We developed an architectural design strategy to alleviate integration issues.} The workflows and strategies proposed throughout this thesis can assist developers integrate these services into their software more robustly, thus increasing overall solution quality.
\end{enumerate}
In \cref{ch:icsme2019}, we presented an improved understanding of \glspl{iws}---concretely, those that provide computer vision---by examining their runtime behaviour and evolution profile over an 11-month longitudinal study. The implications of this work emphasise the caution developers need to take before diving deep into using these services, and highlight the substantial impacts to software quality if these considerations are ignored. Under \cref{ch:icse2020}, we provided a statistical analysis which shows how distribution of the types of issues faced by developers differ to that of conventional software components, and that the completeness of the existing \gls{cvs} \glsac{api} documentation is poor. In \cref{ch:semotion2021,ch:caise2021}, we showed that developers find working with these services more frustrating when contrasted to conventional software engineering components, To address documentation quality, we derive a taxonomy of \glsac{api} documentation requirements in \cref{ch:tse2020} (see \cref{fig:conclusion:taxonomy}) from a synthesis of existing literature. This taxonomy offered (i) 12 areas of improvement to the services' documentation, (ii) areas warranting further research exploration in software engineering literature, and (iii) detailed insight into which specific artefacts are more-valued by practitioners than others. This taxonomy further serves as a go-to `checklist' for any software engineer to review a prioritised list of documentation elements worth implementing into their own \glsac{api} documentation. Lastly, our investigations into improved \gls{iws} integration architectures proposes several strategies by which developers can guard against the non-deterministic evolutionary issues found in \cref{ch:icsme2019}. Preliminary solutions such as that presented in \cref{ch:icwe2019} helped informed further investigations into how developers can prevent evolution evident in \glspl{cvs} via a client-server intermediary proxy server strategy (\cref{ch:fse2020}) and use a novel workflow to better select appropriate confidence thresholds calibrated for their application's domain (\cref{ch:fse-demo2020}). A more extensive discussion into the contributions of this thesis is presented in \cref{sec:introduction:research-contributions}.

\subsection{Answers to Research Questions}

In this subsection, we directly answer the four primary research questions that were posed in \cref{sec:introduction:goals}.

\subsubsection*{RQ1: ``\RQOneTextLandscapeAnalysis{}''}
% RQ1.    What is the nature of cloud-based CVSs?
\begin{callout}
These services are in a nascent stage, are difficult to evaluate, and are not easily interchangeable. They present themselves as conceptually similar, but we find they functionally differ between vendors. Their labels are semantically disparate and work needs to be done on consolidating a standardised vocabulary for labels. Evolution within these services occurs and is not sufficiently versioned or documented to developers, since their results are non-static.
\end{callout}

% RQ1.1.  What is their runtime behaviour?
Irrespective of which service is used, the vocabulary used to label an image is disparate. We find that \textbf{there exists no common standard vocabulary} (e.g., `border collie' vs. `collie') and \textbf{semantic consistency for the same image between services is disparate}, for example as that shown in \cref{fig:conclusion:consistency} (left). The runtime behaviour of these services when contrasted against \textit{each other} is, therefore, inconsistent, and thus (without semantic comparison of images, such as that suggested in \cref{ch:icwe2019}) the vendors are not `plug-and-play'. In contrast to deterministic web services, the same result is functionally guaranteed despite which service is used. For instance, conceptually, a cloud storage service will provide the same output for the same input; that is, regardless of whether a developer uses AWS or Google Cloud object storage, when they upload a file, that file is (more or less) guaranteed to be stored. A deterministic input/output is, thereby, conceptually and functionally guaranteed. However, \textbf{we find that the nature of \glspl{cvs} are conceptually similar but functionally different between services}, and therefore developers are likely to become vendor locked. For instance, as we show in \cref{icsme2019:ssec:findings:consistency-of-labels}, one service may return the duplicity of objects in an image (e.g., `several'), while another service may return the subject of the image (e.g., `carrot') or a hypernym of that subject (e.g., `food'), and another service may focus on the environment of the image (e.g., `indoors').
Further, even when a label is consistent between services, we find the consistency of how \textit{well} they agree to that result---as measured by their confidence score in the label---does not always strictly match in their level of agreement. As we show in \cref{fig:conclusion:consistency} (right), \textbf{distributions of agreement can be disparate even where services agree on a label for the same image}.
% RQ1.2.  What is their evolution profile?
Lastly, while \glspl{cvs} are somewhat stable in the responses they return, \textbf{their responses are non-static}. There is no guarantee that a request with the same image sent in testing will return the same response, and we find that this potential evolution risk is not sufficiently communicated to developers.

\begin{figure}[h]
    \centering
    \includegraphics[width=0.45\linewidth]{mainmatter/publications/figures/icsme2019/000000009590.jpg}
    \hfill
    \includegraphics[width=0.45\linewidth]{mainmatter/publications/figures/icsme2019/000000095707.jpg}
    \caption[Results from computer vision services can be disparate and non-static]{\textit{Left:} Semantic consistency \textit{between} services is not always guaranteed. Two services identified this image as `people', while another identified `conversation', which is not registered at all as a possible label from the other two services. \textit{Right:} Even when services agree on a label, the distribution of their level of agreement is (at times) inconsistent---in the above image, `food' is detected at confidence levels of three services ranging from 94.93\% to 41.39\%.}
    \label{fig:conclusion:consistency}
\end{figure}

\subsubsection*{RQ2: ``\RQTwoTextDocumentation{}''}
% RQ2.    Are CVS APIs sufficiently documented?
\begin{callout}
These services are largely well-documented, but areas of improvement can be identified. By applying the five-dimensional taxonomy we propose in \cref{ch:tse2020} to three services, we found there to be  twelve ways vendors can better improve their services' documentation. We found the ways in which developers can use these services could be improved---such as improved tutorials that integrate \textup{multiple} components of the service---and by providing better descriptions to improve developers' conceptual understanding of computer vision.
\end{callout}

\begin{figure}[p]
	\centering
	\includegraphics[width=.65\linewidth]{mainmatter/publications/figures/tse2020/taxonomy}	
	\caption[Our proposed taxonomy]{Our proposed taxonomy from \cref{ch:tse2020}, forming the five requirements of good \glsac{api} documentation (\dimcat{a} through \dimcat{e}) and each of the 34 concrete documentation artefacts that comprise the requirement. A checkmark is used to indicate that the documentation artefact was present in all three \glspl{cvs} assessed, while a cross is used to indicate that the documentation artefact was missing from all three.}
    \label{fig:conclusion:taxonomy}
\end{figure}

\begin{figure}[t]
	\centering
	\includegraphics[width=.5\linewidth]{mainmatter/publications/figures/tse2020/ips-vs-ils}~
	\includegraphics[width=.5\linewidth]{mainmatter/publications/figures/tse2020/ips-vs-cvs2}
	\caption[Comparing IPS values to ILS and CVS assessments]{\textit{Left}: Triangulation of research attention, in literature, of each documentation artefact compared to the value reported by the developers. \textit{Right}: Triangulation of the value each documentation artefact has to developers contrasted to their coverage in \gls{cvs} \glsac{api} documentation. Semi-circles indicate partial presence in one of the three services. Full circles indicate total presence in a service.}
	    \label{fig:conclusion:ils-ips-cvs}
\end{figure}

% RQ2.1.  What are the dimensions of a ‘complete’ API document, according to both literature and practitioners?
To understand if these services are sufficiently documented, we first investigated what documentation artefacts constitute the requirements of good \glsac{api} documentation through a \glslong{sms} of the literature. We systematically developed a taxonomy---reproduced in \cref{fig:conclusion:taxonomy}---that we then validated in a twofold manner. Firstly, we assessed each documentation artefact against practitioner efficacy by surveying \SurveyParticipantsTotal{} software developers to estimate what value they hold for each documentation artefact. Secondly, we assessed inclusivity of each artefact within three popular \glspl{cvs}, noting the artefact as fully present, partially present, or not present at all. These findings are summarised within \cref{fig:conclusion:ils-ips-cvs}.

The taxonomy itself consists of five dimensions, each representing a distinct requirement for good \glsac{api} documentation: 
\begin{enumerate}[label=\textbf{(\roman*)}]
	\item \dima{}, or \textit{how} can developers use the \glsac{api} for their intended use case? 
	\item \dimb{}, or \textit{when} should the developer choose this particular \glsac{api} for their intended use case?
	\item \dimc{}, or \textit{why} does the developer select this particular \glsac{api} for their application's domain and does the \glsac{api}'s domain align with the application's domain?
	\item \dimd{}, or \textit{what} additional \glsac{api} documentation can the developer find to aid their productivity?
	\item \dime{}, or is the \textit{visualisation} of the above information well organised and easy for the developer to digest?
\end{enumerate} 
 This taxonomy is presented with further detail, including usage examples, under \cref{tse2020:tab:taxonomy}. \textbf{Developers argue that code snippets are the most important documentation artefact, followed closely by tutorials and low-level reference documentation. This is largely explored by existing research.}
% RQ2.2.  What additional information or attributes do application developers need in CVS API documentation to make it more complete?
When we apply this taxonomy to \glspl{iws} such as \glspl{cvs}, we find that there can be improvements made to all dimensions except documentation presentation, which is sufficient. \textbf{The largest suggested improvements fall into the usage description dimension}, in which quick-start guides, step-by-step tutorials, reference applications, best-practices, listings of all \glsac{api} components, minimum system dependencies, and installation instructions require further detail. The second largest dimension falls into the domain concepts behind computer vision, where vendors should provide a greater emphasis behind computer vision concepts and definitions of relevant computer vision terminology (especially since many vendors refer to the same concept under different terms, such as `image tagging' and `label detection' for what is essentially object recognition). \textbf{Additional suggested improvements specific to the \glspl{cvs} services we assessed are provided in \cref{tse2020:sec:suggested-improvements}.} The lack of complete documentation in domain concepts was further reflected in developer discussions on \glslong{so}, as found in \cref{ch:icse2020}. \Cref{tse2020:sec:tax-analysis:cvs-improvement} details these suggested improvements in greater detail.

\subsubsection*{RQ3: ``\RQThreeTextDevMiscomprehension{}''}

% RQ3.    Are CVSs more misunderstood than conventional software engineering domains?
\begin{callout}
In conventional software engineering domains, where the technical domain is well-established and well-understood by developers, questions asked by developers are of greater depth. In contrast, their shallow understanding of the technical domain of computer vision is reflected by questions that highlight a poor understanding of the behaviour of these services and the contexts by which they work. Thus, simpler questions are asked, such as help with trying to understand basic error codes, or clarification of basic concepts and terminologies in computer vision. Therefore, we argue that they are more misunderstood seeing as the domain of \glspl{iws} is still immature.
\end{callout}
% RQ3.1.  What types of issues do application developers face most when using CVSs, as expressed as questions on Stack Overflow?
As expressed on \glslong{so}, we find developers struggle most with simple debugging issues, which reflects a shallow understanding of the of the \glsac{ai} concepts that empower these services. \textbf{The technical nuances become so abstracted away that developers begin to lack a full appreciation of the context and proper usage of these systems.} These questions reveal how developers do not have a strong grasp of the behaviour of these services and how further functional capability needs to be overcome by secondary phases of work, such as pre- and post-processing. \textbf{Their conceptual understanding of these services are poor}, with our findings suggesting that developers present a misunderstanding of the vocabulary used within computer vision, such as the differences between object and facial detection, localisation and recognition. The lack of strong conceptual understanding also reflects in discrepancy-based issues where developers cannot appreciate why services result in specific outcomes contrary to what they believe should happen.
% RQ3.2.  Which of these issues are application developers most frustrated with?
\textbf{We find these discrepancy-based issues to be the most frustrating for developers}, and argue that this is rooted in a need for developers to have some basic understanding of computer vision before diving into services such as these. In terms of the documentation of these services, \textbf{developers express frustration towards the completeness of the documentation}, whereby they seek additional information from the official documentation sources but are unable to find anything to help resolve this gap. Further, \textbf{they question the accuracy of the cloud documentation since it is in contrast with the behaviour they observe}, as related to the discrepancy-based issues they find. 
% RQ3.3.  Is the distribution CVS pain-points different to established software engineering domains, such as mobile or web development?
In contrast to more established domains, such as mobile and web-development, the distribution of issues are different (see \cref{fig:conclusion:question-diff}). Rather than trying to interpret simple errors (as is the case for \glspl{cvs}), developers question \glsac{api} usage and high-level conceptual questions. Developers have a greater appreciation for the technical domain in these mature areas, resulting in fewer shallower questions asked. 

\begin{figure}[bth]
  \begin{subfigure}[c]{0.49\linewidth}
    \centering
    \includegraphics[width=\linewidth]{mainmatter/publications/figures/icse2020/a-compare.pdf}
  \end{subfigure}
  \hfill
  \begin{subfigure}[c]{0.49\linewidth}
    \centering
    \includegraphics[width=\linewidth]{mainmatter/publications/figures/icse2020/b-compare.pdf}
  \end{subfigure}
    \centering
    \caption[Distribution of issues on Stack Overflow]{The distribution of documentation-specific questions (\textit{left}) and generalised questions (\textit{right}) differs between prior studies. Descriptions of each category for both question types are found in \cref{icse2020:tab:taxonomies}.}
    \label{fig:conclusion:question-diff}
\end{figure}

%\clearpage%TODO: Ensure this clearpage is needed

\subsubsection*{RQ4: ``\RQFourDesign{}''}
\begin{callout}
Developers can employ the use of a facade-based architecture to merge the responses of \textup{multiple} vendors using a novel, proportional-representation based approach using lexical databases to resolve ontological issues of labels. An integration strategy consisting of four workflows was presented in \cref{ch:fse2020} to assist developers monitor and handle substantial evolution change in these services. Developers can deal with the probabilistic nature of these services by using a representative dataset of their application's data to fine-tune a confidence threshold and monitor threshold changes in a production setting.
\end{callout}

This thesis offers three strategies targeted at improved integration of developer applications with \glspl{cvs}. \cref{ch:icwe2019} successfully demonstrated that multiple services can be combined using lexical databases to better improve the reliability of relying on a single service's label. Further, this strategy outperformed naive merge methods using a novel proportional representation method by 0.015 F-measure. This strategy uses the idea of a client-server intermediary facade to handle these operations and produce a consistent result regardless of which service is being used. This inspired further work presented in \cref{ch:fse2020}. To handle the evolutionary issues found in the services, we developed a novel integration architecture based on the proxy server strategy, integrating four key proposed workflows which can be used to guard against evolution and non-determinism in these services: (i) initialising a representative benchmark of domain-specific data used in the client application; (ii) validating that the service is behaving as expected against that benchmark; (iii) periodically detecting for evolution if behavioural change occurs, thereby notifying change; and lastly (iv) invalidating future requests if substantial evolution is detected in step (iii). This, in turn, resolves a non-deterministic response into a deterministic error when evolution is raised. Lastly, to deal with the uncertainty arising from probabilistic confidence values, we proposed Threshy (see \cref{ch:fse-demo2020}), a tool to help developers select appropriate threshold boundaries resulting from their benchmark datasets and cost factors (due to missed predictions). Ultimately these strategies aim at improving the robustness of applications that are dependent on \glspl{cvs}.

\subsection{Limitations to Research Answers \& Future Research}

In this section, we discuss limitations to our research answers found and the contributions we have made to the overall software engineering body of knowledge. Since the application domain of \glspl{iws} is a relatively new field, we have posed research ideas that still remain to be explored. This leaves the door open to future studies that can further explore and improve upon pre-packaged \glsac{ai} components, such as \glspl{iws}.

\subsubsection{Generalisability Threats Resulting from Computer Vision}

Throughout this thesis, we have used computer vision as a primary exemplar of pre-trained \glsac{ml} models abstracted as intelligent \glsac{ai} components. These components are embedded into cloud platforms, typically provided via \glsac{rest}ful \glsac{api} endpoints. Limiting this research to such a narrow scope is an illustrative example that enables more concrete findings and potential solutions to a specific subset of \glspl{iws}. As discussed in \cref{ssec:introduction:motivation}, these particular type of \glspl{iws} were selected due to both their increasing enthusiasm and uptake in developer communities (see \cref{fig:introduction:stackoverflow-trends}) and their maturity in the area. However, we acknowledge that there are myriad domains in the \gls{iws} space, such as:

\begin{itemize}
	\item sentiment analysis;
	\item text-to-speech and speech-to-text;
	\item natural language processing;
	\item time-series data analysis; or,
	\item signal data analysis
\end{itemize}

Additionally, our analyses of \glspl{cvs} chiefly targets content analysis (or object detection) endpoints of these services; other endpoints such as image description or object localisation exist, and were not considered as the main unit of analysis in this work. Further, this thesis selects three prominent vendors of \glspl{cvs}: Google, Microsoft, and Amazon. While these vendors are considered to be the ubiquitous `go-to' providers for cloud-based services (given their AWS, Google Cloud, Azure platforms) and were the most adopted for enterprise solutions \citep{RightScaleInc:2018kJ}, many other providers of computer vision intelligence exist \citepweb{Pixlab:Home,IBM:Home,Cloudsight:Home,Clarifai:Home,DeepAI:Home,Imagaa:Home,Talkwaler:Home,Kairos:Home,Affectiva:Home,Cognitec:Home}, including those from Asian market \citepweb{Megvii:Home,TupuTech:Home,YiTuTech:Home,SenseTime:Home,DeepGlint:Home} where language barriers (of the author) prevented analysis of these services.

Thus, our findings need focused investigation of areas \textit{beyond} computer vision; that is, future research should explore other types \glspl{iws} to assess whether our findings and solutions are generalisable to other domains, and also to other types of services in the \gls{cvs} market. Further, this thesis emphasises investigations into identifying issues within web-based \glspl{iws}. We establish a better understanding on their nature and run-time behaviour  (\ref{rq:nature}), how they are documented (\ref{rq:docs}), and how well they are understood by developers (\ref{rq:devs}), but only offer partial solutions to these issues (e.g., \ref{rq:design}). 

Future work can consider the issues identified in this work as a stepping-stone into additional solutions yet to be explored, identifying other ways (beyond improved integration techniques) in which developers can handle these issues. For example, the broader concepts of our contributed architecture (e.g., use of a behaviour token, its parameters, and the error codes proposed) can be shifted to handle issues in natural language processing to demonstrate the generalisability of the architecture to other \glspl{iws}, since topic modelling produces labels with confidences and the approach can be largely transferred to this area. Similarly, the nature of \textit{other} \glspl{iws} should be explored in order to understand whether similar evolution and behavioural runtime patterns exist with their computer vision equivalents (as we identified in this thesis). How to better support developers using different types of intelligent components would be an interesting area to explore, especially in applying our design strategies to combat the robustness issues we have identified to these other types of services, and to identify any potential pitfalls of our design.  

Hence, future studies may poset research questions such as: \textit{Are there non-trivial impacts to responses in evolving speech-to-text services?} The pre-trained models that empower such services may transcribe certain spoken words better (or, indeed, worse) as they become updated. \textit{Can the design strategy explored in this thesis be extended for natural language processing services?} Given that many natural language processing services provide confidence values for their predictions, calibrating confidence thresholds is a non-trivial research opportunity that remains open. Many more can be drawn from this thesis, simply by changing the application context of computer vision to a new domain.

\subsubsection{Additional Research Decision Implications}

Certain drawbacks to certain research decisions adopted in the various studies within this thesis should be noted. For example, our proposed architectural usage framework is a preliminary design, and rigorous testing in real-world scenarios is needed. Some suggestions include running a long-term industry case study that implements our design, where a team can identify possible pitfalls to the design, or conducting formal architecture evaluations, such as ATAM \citep{Kazman2000}. These would be a possible avenue of research to \textit{verify} the design in practice, which is necessary to ensure that the design is practical. 

Moreover, our proposal makes use of the benchmark dataset approach, but we are yet to explore and test potential guidelines in developing a benchmark dataset. While we provide some preliminary guidelines in \cref{fse2020:sec:future-work:guidelines}, these will need to be evaluated for practical use again. Certain trade-offs have yet to be considered and be tested; for example, the cost to implement the design, the time needed to educate developers on these concerns, or how much effort is required for project managers to factor this design into software planning.

Another aspect concerning our research design revolves around the documentation contributions of this study. Investigating whether our suggested documentation improvements are applicable to these different services is paramount, as they may be too heavily tied to computer vision concerns. Developing improved documentation and tooling that better support developers when using these \glspl{iws} (and how our proposed architecture fits in) should be explored. 

Methodologically, other concerns exist in our documentation research too: for instance, the survey that was used in \cref{ch:tse2020} utilised a survey distributed via the web. Thus, the population of reference was not truly known, and there was no way for us to reliably validate the representativeness of our sample population (that is, we do not truly know whether they were indeed software engineers). Future studies should explore more conventional survey distribution methods (e.g., an employee database of a large company that is detailed and well-defined, such as that performed in \citep{Robillard:hk}). Unfortunately, we did not have access to such a database; hence, repeating the study with a larger, well-defined population could improve the reliability of our results and improve our understanding of what developer's actually want in \gls{iws} documentation. Further, since our survey did not directly account for the \textit{cost} of creating documentation and the various trade-offs involved; again, the practical concerns need to be addressed with real-world use cases. In the survey, developers can wish for as much documentation as they desire, and this partially explains the heavily biased results to include most documentation features.

The practicalities of such concerns would involve research methods that directly involve human participants (methods this thesis did not include; see \cref{fig:research-methodology:review:field-techniques} and asterisked techniques). As a concrete example, the preliminary investigations provided under \cref{ch:semotion2021,ch:caise2021}\footnote{We acknowledge the similarities between \cref{ch:semotion2021,ch:caise2021}. While the key themes and outcomes described in both papers are different, both the data (from \cref{ch:icse2020}) and tools (EmoTxt) being investigated remain the same. These chapters provide initial exploration into the issue of emotions in \glspl{cvs}, however additional studies are warranted to explore a different dataset or tooling.} have only utilised online textual artefacts (i.e., \glslong{so} questions of \gls{cvs} providers). This decision was based on the curated dataset of \glslong{so} derived in \cref{ch:icse2020}. Using this as data for emotional analysis was a first step, but future studies should validate whether the miscomprehension from \glslong{so} posts are reflected in the real-world. A stronger understanding of the actual emotions experienced software engineers whilst using \glspl{iws} should involve human participants, for example with brainstorming sessions, think-aloud sessions, or interviews with developers who use these services; developers could be asked to draw concept maps of their concerns (which can then be compared to the issues we identified from our analyses).

An important limitation worth noting about the studies reliant on \glslong{so} in this thesis is that the users who post questions on \glslong{so} may not be even be developers, and questions on \glslong{so} can be heavily edited (and moderated) by the \glslong{so} community. It would be important to clarify whether the version of the text that we analysed was the original question by the poster, or an edited version; this was not done during the studies within this thesis, and is a research gap left open for future work. Lastly, in terms of our emotional analysis of \glslong{so} questions, we can presume that most developers are professionals who self-censor their emotions before posting. Only in the most extreme situations would strong emotions be noticeable, which can account for how the vast majority of posts have emotions too weak detect. Again, the emotional state of developers as they write such questions may not be truly reflected in the words they use to express their issues, and as such, the emotional analysis we report is likely still incomplete. Future studies must use more direct techniques on developers to better ascertain their emotional state as they use such services.

Finally, since we find these services are at an early stage and undergoing rapid evolution (much like web technologies in the 1990s and 2000s) we anticipate substantial growth in the understanding of how we will use these services and maturity in the developer's appreciation of its surrounding technical domain. Therefore, it would be beneficial to repeat some of the studies within this thesis and assess whether there is an improved understanding of the phenomena occurring within \glspl{iws} and whether developers have an improved mindset of these services and how they can be used. Thus, different tools, designs, or suggestions may result from repetitional studies 5-10 years in the future. This, therefore, identifies evolution in the \textit{maturity} of \glspl{iws}, and to highlight whether developers are showing a stronger understanding of the surrounding technical domain behind these services.

\section{Concluding Remarks}

As has been achieved with other software components, recent trends to raise abstraction levels of \glsac{ml} (from low-level statistical operations to high-level \glsac{ml} components) aim to reduce the time, effort, and knowledge required for software developers to integrate \glsac{ml} into their application. This trend is warranted given the ever-increasing needs to incorporate \glsac{ai}, and particularly deep learning, into applications, thereby creating \glsac{ai}-first `intelligent' software. \Glsx{iws} are a common form of these high-level \glsac{ml} components that are offered through prominent cloud platforms---such as Google's Cloud Platform, Microsoft's Azure, or Amazon Web Services---and the integration and usage of these components into conventional software is at the core of this work.

To our knowledge, little prior investigation has been conducted to understand \glspl{iws} via the lenses of software quality; primarily the robustness, reliability of the services and completeness of its documentation. In this thesis, we have shown that the non-deterministic and probabilistic properties of computer vision \glspl{iws} present non-trivial impacts to the quality of software that they are integrated with, and it is pivotal that developers have a greater appreciation of the technical domain behind the \glsac{ai} techniques that empower such services. 

In identifying evolutionary and run-time issues of these services, the ways in which they are (currently) documented and these issues communicated (or not), and analysing how developers perceive these services with a deterministic mindset, we have shown just how fragile the use of such services (as they stand) are. We strongly encourage vendors to use suggestions made within this research to improve both their documentation and their integration strategies so that developers can ensure more robust applications when using these services. Ultimately, intelligent \glsac{ai} components are still in a nascent stage, and therefore we strongly suggest one message to eager developers: use with caution and be aware of the consequences!
