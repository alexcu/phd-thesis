\chapter{Research Methodology}
\label{ch:research-methodology}

\graphicspath{{mainmatter/research-methodology/figures/}}

\label{sec:research-methodology:preface}

Investigating software engineering practices is often a complex task as it is imperative to understand the social and cognitive processes around software engineers and not just the tools and processes used \citep{Easterbrook:2007ws}. This chapter explores our research methodology by exploring five key elements of empirical software engineering research: firstly, (i) we provide an extended focus to the study by reviewing our research questions (see \cref{sec:introduction:hypohtesis}) anchored under the context of an existing classification taxonomy, (ii) characterise our research goals through an explicit philosophical stance, (iii) explain how the stance selected impacts our selection of research methods and data collection techniques (by dissecting our choice of methods used to reach these research goals), (iv) discuss a set of criteria for assessing the validity of our study design and the findings of our research, and lastly (v) discuss the practical considerations of our chosen methods. 

The foundations for developing this research methodology has been expanded from that proposed by \citet{Easterbrook:2007ws}, \citet{Wohlin:2014jq}, \citet{Wohlin:2012bu} and \citet{Shaw:2003aa}.

\section{Research Questions Revisited}
\label{sec:research-methodology:research-questions}

%In \cref{sec:introduction:goals}, we rq:nature:runtime to three hypotheses: (i) \glspl{cvs} have non-deterministic properties that confuse developers with a deterministic mindset; (ii) the implicit risks to using these services without fully appreciating their unintended side effects has an impact to software quality; (iii) the service providers do not sufficiently document these behavioural and evolution issues.

\newcontent
To discuss our research strategy, we revisit our \NumPrimaryRQs{} primary and \NumSecondaryRQs{} secondary research questions (RQs) through the classification technique discussed by \citet{Easterbrook:2007ws}, a technique originally proposed in the field of psychology by \citet{Meltzoff:1998wg} but adapted to software engineering. A summary of the classifications made to our research questions are presented in \cref{tab:research-methodology:rqs}.

Our research study involves a mix of \NumEmpiricalRQs{} \textit{empirical}\footnote{Or `knowledge' questions, that extend our \textit{knowledge} on certain phenomena.} RQs, that focus on observing and analysing existing phenomena, and \NumNonEmpiricalRQs{} \textit{non-empirical} RQs, that focuses on designing better approaches to solve software engineering tasks \citep{Simon:1996uw}. The use of empirical \textit{and} non-empirical RQs are best combined in long-term software engineering research studies where the phenomena are under-explored, as is the case with \glspl{cvs}. Further, these approaches help propose solutions to issues found in the phenomena studied \citep{Wieringa:2006vd}. We discuss both our empirical and non-empirical RQs in \cref{ssec:research-methodology:research-questions:empirical,ssec:research-methodology:research-questions:nonempirical} below. 

\begin{table}[tbh]
\centering
\caption[Classification of research questions in this thesis]{A summary of our research questions classified using the strategies presented by \citet{Easterbrook:2007ws} and \citet{Meltzoff:1998wg}.}\label{tab:research-methodology:rqs}
\tablefit{\begin{tabular}{p{0.07\linewidth}p{0.55\linewidth}|p{0.13\linewidth}p{.33\linewidth}}
\toprule
\textbf{\#}&
\textbf{RQ}&
\textbf{Primary/ Secondary}&
\textbf{RQ Classification}
\\
\midrule
\midrule
\ref{rq:nature} &
\RQOneTextLandscapeAnalysis{} &
Primary &
\makecell[tl]{
\textsc{Empirical}\\
$\hookrightarrow$~\textit{Exploratory}\\
$~~\hookrightarrow$~\textit{Description/Classification}
}
\\
\ref{rq:nature:runtime} &
\RQOneTextLandscapeAnalysisRuntime{} &
Secondary &
\makecell[tl]{
\textsc{Empirical}\\
$\hookrightarrow$~\textit{Exploratory}\\
$~~\hookrightarrow$~\textit{Description/Classification}
}
\\
\ref{rq:nature:evolution} &
\RQOneTextLandscapeAnalysisEvolution{} &
Secondary &
\makecell[tl]{
\textsc{Empirical}\\
$\hookrightarrow$~\textit{Exploratory}\\
$~~\hookrightarrow$~\textit{Description/Classification}
}
\\
\midrule
\ref{rq:docs} &
\RQTwoTextDocumentation{}&
Primary &
\makecell[tl]{
\textsc{Empirical}\\
$\hookrightarrow$~\textit{Exploratory}\\
$~~\hookrightarrow$~\textit{Existence}
}
\\
\ref{rq:docs:complete} &
\RQTwoTextDocumentationWhatIsCompleteDocs{}&
Secondary &
\makecell[tl]{
\textsc{Empirical}\\
$\hookrightarrow$~\textit{Exploratory}\\
$~~\hookrightarrow$~\textit{Composition}
}
\\
\ref{rq:docs:missing} &
\RQTwoTextDocumentationMissingAttributes{}&
Secondary &
\makecell[tl]{
\textsc{Non-Empirical}\\
$\hookrightarrow$~\textit{Design}\\
}
\\
\midrule
\ref{rq:devs} &
\RQThreeTextDevMiscomprehension{}&
Primary &
\makecell[tl]{
\textsc{Empirical}\\
$\hookrightarrow$~\textit{Exploratory}\\
$~~\hookrightarrow$~\textit{Descriptive-Comparative}
}
\\
\ref{rq:devs:issues} &
\RQThreeTextDevMiscomprehensionIssueTypes{}&
Secondary &
\makecell[tl]{
\textsc{Empirical}\\
$\hookrightarrow$~\textit{Base-Rate}\\
$~~\hookrightarrow$~\textit{Frequency/Distribution}
}
\\
\ref{rq:devs:frustration} &
\RQThreeTextDevMiscomprehensionFrustration{} &
Secondary &
\makecell[tl]{
\textsc{Empirical}\\
$\hookrightarrow$~\textit{Exploratory}\\
$~~\hookrightarrow$~\textit{Description/Classification}
}
\\
\ref{rq:devs:vs-traditional} &
\RQThreeTextDevMiscomprehensionVsConventional{} &
Secondary &
\makecell[tl]{
\textsc{Empirical}\\
$\hookrightarrow$~\textit{Base-Rate}\\
$~~\hookrightarrow$~\textit{Frequency/Distribution}
}
\\
\midrule
\ref{rq:design} &
\RQFourDesign{} &
Primary &
\makecell[tl]{
\textsc{Non-Empirical}\\
$\hookrightarrow$~\textit{Design}\\
}
\\
\bottomrule
\end{tabular}}
\end{table}


\subsection{Empirical Research Questions}
\label{ssec:research-methodology:research-questions:empirical}

In total, \NumEmpiricalRQs{} empirically-based RQs are posed in this study to help us understand the way developers currently interact and work with web services that provide computer vision. The majority of these questions are \textit{exploratory} questions that contribute to a landscape analysis of these services (\ref{rq:nature}, \ref{rq:nature:runtime} and \ref{rq:nature:evolution}), how well they are documented (\ref{rq:docs}), and the issues developers currently face when using them (\ref{rq:devs}). Our other exploratory questions complement the answers to these questions. For instance, to understand if \glspl{cvs} are sufficiently documented (an \textit{existence} exploratory question posed in \ref{rq:docs}), we need to understand the components of a `sufficient' or `complete' \gls{api} document via \ref{rq:docs:complete} as proposed in both the literature and by software developers. While \ref{rq:docs:complete} does not directly relate to \glspl{cvs}, answering it gives us an understanding the components of complete \gls{api} documentation, and therefore, we can assess what aspects they are missing and where improvements can be made (\ref{rq:docs:missing}). These questions are \textit{descriptive and classification} questions that help describe and classify what practices are in use for existing \gls{cvs} \gls{api} documentation and the nature behind these services. Answering these exploratory questions assists in refining preciser terms of the phenomena, ways in which we find evidence for them and ensuring the data found is valid.

By answering these questions, we have a clearer understanding of the phenomena; we then follow up by posing \NumBaseRateRQs{} additional \textit{base-rate questions} that helps provide a basis to confirm that the phenomena occurring is normal (or unusual) behaviour by investigating the patterns of phenomena's occurrence against other phenomena. \ref{rq:devs:issues} is a \textit{frequency and distribution} question to help us understand what types of issues developers often encounter most, given a lack of formal extended training in artificial intelligence. This achieves us an insight into the developer's mindset and regular thought patterns toward these \glspl{api}. We can then contrast this distribution using our second base-rate question (\ref{rq:devs:vs-traditional}), that assesses the distributional differences between these intelligent components and non-intelligent (conventional) software components. Combined, these two questions can help us answer how the issues raised against \glspl{cvs} are different to normal \glslong{so} issues---our \textit{descriptive-comparative} question posed in \ref{rq:devs}---and, similarly, we can classify and rank which issues developers find most frustrating (\ref{rq:devs:frustration}).

%Lastly, we investigate the relationship between the improved documentation and improvements to other aspects of the software development process. Chiefly, \ref{rqs:implications:do-metrics-improve} is concerned with whether any improvements to metadata or documentation correlate to improvements in software quality, developer productivity, or developer education (and is a \textit{relationship establishment question}). If we establish such a relationship, we refine the question and investigate the specific causes using three \textit{causality questions} defined under \ref{rqs:implications:aspects}, namely by associating three classes of measurable metrics (internal quality metrics, external quality metrics, developer education insight metrics) to the improved documentation.

\subsection{Non-Empirical Research Questions}
\label{ssec:research-methodology:research-questions:nonempirical}

\ref{rq:docs:missing} and \ref{rq:design} are both non-empirically-based \textit{design questions}; they are concerned with ways in which we can improve a \gls{cvs} by investigating what additional attributes are needed in both the documentation of \glspl{cvs} and in the integration architectures developers can employ to improve reliability and robustness in their applications. They are not classified as empirical questions as we investigate what \textit{will be} and not \textit{what is}. By understanding the process by which developers desire additional attributes of documentation and integration strategies, we can help shape improvements to the existing designs of using \glspl{cvs}.
\section{Philosophical Stances}
\label{sec:research-methodology:philosophical-stances}

\todo{JG: do you really need this section? :-) }

Philosophical stances guide the researcher's action by fortifying what constitutes `valid truth' against a fundamental set of core beliefs \citep{Ritzer:1991ge}. In software engineering, four dominant philosophical stances are commonly characterised \citep{Creswell:2017vn,Petersen:2019ji}: positivism (or post-positivism), constructivism (or interpretivism), pragmatism, and critical theory (or advocacy/participatory). To construct such a `validity of truth', we will review these four philosophical stances in this section, and state the stance that we explicitly adopt and our reasoning for this.

\paragraph{Positivism}
Positivists claim truth to be all observable facts, reduced piece-by-piece to smaller components which is incrementally verifiable to form truth. We do not base our work on the positivistic stance as the theories governing verifiable hypothesis must be precise from the start of the research. Moreover, due to its reductionist approach, it is difficult to isolate these hypotheses and study them in isolation from context.
As our hypotheses are not context-agnostic, we steer clear from this stance.

\paragraph{Constructivism}
Constructivists see knowledge embedded within the human context; truth is the \textit{interpretive} observation by understanding the differences in human thought between meaning and action \citep{Klein:1999uv}. That is, the interpretation of the theory is just as important to the empirical observation itself.
We partially adopt a constructivist stance as we attempt to model the developer's mindset, being an approach that is rich in qualitative data on human activity.

\paragraph{Pragmistism}
Pragmatism is a less dogmatic approach that encourages the incomplete and approximate nature of knowledge and is dependent on the methods in which the knowledge was extracted. The utility of consensually agreed knowledge is the key outcome, and is therefore relative to those who seek utility in the knowledge---what is the useful for one person is not so for the other. While we value the utility of knowledge, it is difficult to obtain consensus especially on an ill-researched topic such as ours, and therefore we do not adopt this stance.

\paragraph{Critical Theory}
This study chiefly adopts the philosophy of critical theory \citep{Calhoun:1995ww}. A key outcome of the study is to shift the developer's restrictive deterministic mindset and shed light on developing a new framework actively with the developer community that seeks to improve the process of using such \glspl{api}. In software engineering, critical theory is used to ``actively [seek] to challenge existing perceptions about software practice'' \citep{Easterbrook:2007ws}, and this study utilises such an approach to shift the mindset of \gls{cvs} consumers and providers alike on how the documentation and metadata should not be written with the `traditional' deterministic mindset at heart. Thus, our key philosophical approach is critical theory to seek out \textit{what-can-be} using  partial constructivism to model the current \textit{what-is}.

\section{Research Design}
\label{sec:research-methodology:experiments}

This section discusses an overview of the design of methods used within the experiments conducted under this thesis. For each experiment, we describe an overview of the experiment grounded known methods and techniques (\cref{ssec:research-methodology:review:methods,ssec:research-methodology:review:techniques}) and our approach to analysing the data, as well as relating the selecting method back to a specific RQ. Details of each experiment presented in this thesis, the coherency between them, and where they can be found are given in \cref{sec:introduction:organisation,sec:introduction:research-contributions}.

\subsection{Landscape Analysis of Computer Vision Services}

To understand the behavioural and evolutionary profiles of \glspl{cvs} (i.e., \ref{rq:nature}), we employed a longitudinal study based around a dynamic system analysis combined with system instrumentation \citep{Singer:2007tu}. Specifically, we used structured observations of three services using the same dataset to understand how the responses from these services change with time. Lastly, we utilised documentation analysis to assess the overall `picture' of how these services are documented. Further details on this experiment is given in \textbf{\cref{ch:icsme2019}, \cref{icsme2019:sec:method}}.

\subsection{Utility of API Documentation in Computer Vision Services}

To assess whether these services are sufficiently documented (i.e., \ref{rq:docs}), we conducted a systematic mapping study \citep{Kitchenham:2007dd,Petersen:2008td} of the various academic sources detailing \gls{api} documentation knowledge.\footnote{Refer to \cref{ch:tse2020} for a clear definition of these terms.} We then consolidated this information into a structured taxonomy following a systematic taxonomy development method specific to software engineering studies \citep{Usman:2017hn}.

We followed the triangulation approach proposed by \citet{Jick:1979el} to validate the taxonomy by use of a personal opinion survey. \citet{Kitchenham:2007ux} provide an introduction on methods used to conduct personal opinion surveys which we adopted as an initial reference in (i) shaping our survey objectives around our research goals, (ii) designing a cross-sectional survey, (iii) developing and evaluating our survey instrument, (iv) evaluating our instruments, (v) obtaining the data, and (vi) analysing the data. We were inspired by \citeauthor{Brooke:1996ua}'s \gls{sus} \citep{Brooke:1996ua} technique, thereby basing our research questions against a known surveying instrument.

As is good practice in developing questionnaire instruments to evaluate their reliability and validity \citep{Litwin:1995wt}, we evaluated our instrument design by asking colleagues to critique it via pilot studies within \gls{a2i2}. This assisted in identifying any problems with the questionnaire itself and with any issues that may have occured with the response rate and follow-up procedures.

Findings from the pilot study helped inform us for a widely distributed questionnaire using snow-balling sampling. Human ethics approval by the Deakin University Faculty of Science, Engineering and Built Environment Human Ethics Advisory Group (SEBE HEAG)\footnote{Project identifiers \texttt{STEC-11-2019-CUMMAUDO} and \texttt{STEC-39-2019-CUMMAUDO}.} was attained to externally conduct this survey research (see \cref{ch:ethics}). 
Further details on these methods are detailed within \textbf{\cref{ch:tse2020}, \cref{tse2020:sec:method}}.

\subsection{Developer Issues concerning Computer Vision Services}

Developers typically congregate in search of discourses on issues they face in online forums, such as \glsx{so} and Quora, as well as writing their experiences in personal blogs such as Medium. The simplest of these platforms is \gls{so} (a sub-community of the Stack Exchange family of targeted communities) that specifically targets developer issues on using a simple Q\&A interface, where developers can discuss technical aspects and general software development topics. Moreover, \gls{so} is often acknowledged as \textit{the} `go-to' place for developers to find high-quality code snippets that assist in their problems \citep{Subramanian:2014bg}.

Thus, to begin understanding the issues developers face when using \glspl{cvs} and whether there is a substantial difference to conventional domains (i.e., \ref{rq:devs}), we used repository mining on \gls{so} to help answer RQ3. Specifically, we selected \gls{so} due to its targeted community of developers\footnote{We also acknowledge that there are other targeted software engineering Stack Exchange communities such as Stack Exchange Software Engineering (\url{https://softwareengineering.stackexchange.com}), though (as of January 2019) this much smaller community consists of only 52,000 questions versus \gls{so}'s 17 million.} and the availability of its publicly available dataset released as `data dumps' on the Stack Exchange Data Explorer\footnoteurl{https://data.stackexchange.com/stackoverflow}{17 January 2017} and Google BigQuery.\footnoteurl{https://console.cloud.google.com/marketplace/details/stack-exchange/stack-overflow}{17 January 2017} Studies conducted have also used \gls{so} to mine developer discourse \citep{Choi:2015wo,Sinha:2013tt,Novielli:2015vda,Rosen:2016uk,Pal:2012te,Bajaj:2014wg,LinaresVasquez:2014vj,Wang:2013ue,Barua:2012gz,Reboucas:2016tw,Allamanis:2013is,Tahir:2018ks}.
Further details on how we approached the design for this study can be found in \textbf{\cref{ch:icse2020}, \cref{icse2020:sec:method}}, \textbf{\cref{ch:semotion2021}, \cref{semotion2021:sec:methodology}}, and \textbf{\cref{ch:caise2021}, \cref{caise2021:sec:method}}

\subsection{Designing Improved Integration Strategies}

Our improved integration strategies (i.e., \ref{rq:design}) evolved organically over the duration of this research through the use of industry case studies and action research. We developed several iterative prototypes to the integration strategies and used a mix of statistical and technical evaluations to analyse whether our improved integration strategies can prove useful. Further details about these approaches are detailed in \textbf{\cref{ch:icwe2019}, \cref{icwe2019:sec:evaluation-method}} and \textbf{\cref{ch:fse2020}, \cref{fse2020:sec:eval}} and \textbf{\cref{ch:fse-demo2020}, \cref{fse-demo2020:sec:threshy}}.

\section{Chapter Summary}

This chapter has explored the research methodology and strategy that is adopted throughout the various studies given within this thesis. We began by revisiting the four primary research questions that were posited in our introductory chapter under \cref{sec:introduction:goals}; as given in \cref{sec:research-methodology:research-questions}, we analysed these questions through the lenses of an existing research question classification taxonomy applicable to software engineering research. We identified which of these questions are grounded through both empirical and non-empirical research, and discussed the underlying reasoning behind the design of each research question. We provided insight into various philosophical stances relevant to software engineering research under \cref{sec:research-methodology:philosophical-stances}, and explained our reasoning for adopting the critical theory worldview in this thesis. Lastly, we reviewed a number of common software engineering research methods in \cref{sec:research-methodology:review,sec:research-methodology:experiments} and those that we adopted in the design of the various experiments described in \cref{part:publications} of this thesis. 
\section{Empirical Validity}
\label{sec:research-methodology:empirical-validity}
\todo{Discuss empirical validity of both methods}

Threats to Validity...


%For this study, we propose running several experiments involving developers and several \glspl{cvcis}, using action-based mixed method approaches and involving documentary analysis. This study will organically evolve by observing phenomena surrounding computer vision \gls{api} internal quality, chiefly their documentation and responses. We adopt a mixed methods approach, performing both qualitative and quantitative data collection on these two key aspects by using documentary research methods for inspecting the \gls{api} documentation and structured observations to quantitatively analyse the results over time (RQs 3 and 4).
%
% Our first proposal for usability studies will survey a number of developers from various levels of seniority and experience (gathering such demographical data to assess a wider sample size) to provide insight into how these developers perceive the non-deterministic nature of computer vision \glspl{api}, asking them specific questions about their conceptual understanding of computer vision to identify any outstanding gaps in their knowledge and factor this into known literature (RQs 1 and 2).
%
%We will then conduct a structured interview with a `mock' computer vision \gls{api} to remove any developer bias toward any one particular computer vision \gls{api} that already exists and by which the developer may have already used in the past. Here, we will investigate if developers have any patterns of practice and if they conform to software engineering best practices (RQs 1, 2 and 3).
%
%From these insights, we can then develop a series of assistive recommendations that aide in improving the validation and verification of the existing computer vision \gls{api} tooling. This may involve a third party tool that helps developers evaluate which particular \gls{api} is right for their specific computer vision use case.


% Ground based on works in Guide to Empitricial SE...%
% Rexplain RQs in the context
% Discuss all methods from GtAESE and why which ones are good/bad
% 




% Get feedback on the first round of survey
% Bypass ethics on this -- find out how/where
% 

%\section{Data Collection and Ethics}
%
%\section{Approach}
%
%\section{Evaluation Methods}

%\section{Threats to Validity}
%
%\subsection{Internal Threats}
%
%\subsection{External Threats}
%
%\subsection{Construct}