\section{Research Design}
\label{sec:research-methodology:experiments}

This section discusses an overview of the design of methods used within the experiments conducted under this thesis. For each experiment, we describe an overview of the experiment grounded known methods and techniques (\cref{ssec:research-methodology:review:methods,ssec:research-methodology:review:techniques}) and our approach to analysing the data, as well as relating the selecting method back to a specific RQ. Details of each experiment presented in this thesis, the coherency between them, and where they can be found are given in \cref{sec:introduction:organisation,sec:introduction:research-contributions}.

\subsection{Landscape Analysis of Computer Vision Services}

To understand the behavioural and evolutionary profiles of \glspl{cvs} (i.e., \ref{rq:nature}), we employed a longitudinal study based around a dynamic system analysis combined with system instrumentation \citep{Singer:2007tu}. Specifically, we used structured observations of three services using the same dataset to understand how the responses from these services change with time. Lastly, we utilised documentation analysis to assess the overall `picture' of how these services are documented. Further details on this experiment is given in \textbf{\cref{ch:icsme2019}, \cref{icsme2019:sec:method}}.

\subsection{Utility of API Documentation in Computer Vision Services}

To assess whether these services are sufficiently documented (i.e., \ref{rq:docs}), we conducted a systematic mapping study \citep{Kitchenham:2007dd,Petersen:2008td} of the various academic sources detailing \gls{api} documentation knowledge.\footnote{Refer to \cref{ch:tse2020} for a clear definition of these terms.} We then consolidated this information into a structured taxonomy following a systematic taxonomy development method specific to software engineering studies \citep{Usman:2017hn}.

We followed the triangulation approach proposed by \citet{Jick:1979el} to validate the taxonomy by use of a personal opinion survey. \citet{Kitchenham:2007ux} provide an introduction on methods used to conduct personal opinion surveys which we adopted as an initial reference in (i) shaping our survey objectives around our research goals, (ii) designing a cross-sectional survey, (iii) developing and evaluating our survey instrument, (iv) evaluating our instruments, (v) obtaining the data, and (vi) analysing the data. We were inspired by \citeauthor{Brooke:1996ua}'s \gls{sus} \citep{Brooke:1996ua} technique, thereby basing our research questions against a known surveying instrument.

As is good practice in developing questionnaire instruments to evaluate their reliability and validity \citep{Litwin:1995wt}, we evaluated our instrument design by asking colleagues to critique it via pilot studies within \gls{a2i2}. This assisted in identifying any problems with the questionnaire itself and with any issues that may have occured with the response rate and follow-up procedures.

Findings from the pilot study helped inform us for a widely distributed questionnaire using snow-balling sampling. Human ethics approval by the Deakin University Faculty of Science, Engineering and Built Environment Human Ethics Advisory Group (SEBE HEAG)\footnote{Project identifiers \texttt{STEC-11-2019-CUMMAUDO} and \texttt{STEC-39-2019-CUMMAUDO}.} was attained to externally conduct this survey research (see \cref{ch:ethics}). 
Further details on these methods are detailed within \textbf{\cref{ch:tse2020}, \cref{tse2020:sec:method}}.

\subsection{Developer Issues concerning Computer Vision Services}

Developers typically congregate in search of discourses on issues they face in online forums, such as \glsx{so} and Quora, as well as writing their experiences in personal blogs such as Medium. The simplest of these platforms is \gls{so} (a sub-community of the Stack Exchange family of targeted communities) that specifically targets developer issues on using a simple Q\&A interface, where developers can discuss technical aspects and general software development topics. Moreover, \gls{so} is often acknowledged as \textit{the} `go-to' place for developers to find high-quality code snippets that assist in their problems \citep{Subramanian:2014bg}.

Thus, to begin understanding the issues developers face when using \glspl{cvs} and whether there is a substantial difference to conventional domains (i.e., \ref{rq:devs}), we used repository mining on \gls{so} to help answer RQ3. Specifically, we selected \gls{so} due to its targeted community of developers\footnote{We also acknowledge that there are other targeted software engineering Stack Exchange communities such as Stack Exchange Software Engineering (\url{https://softwareengineering.stackexchange.com}), though (as of January 2019) this much smaller community consists of only 52,000 questions versus \gls{so}'s 17 million.} and the availability of its publicly available dataset released as `data dumps' on the Stack Exchange Data Explorer\footnoteurl{https://data.stackexchange.com/stackoverflow}{17 January 2017} and Google BigQuery.\footnoteurl{https://console.cloud.google.com/marketplace/details/stack-exchange/stack-overflow}{17 January 2017} Studies conducted have also used \gls{so} to mine developer discourse \citep{Choi:2015wo,Sinha:2013tt,Novielli:2015vda,Rosen:2016uk,Pal:2012te,Bajaj:2014wg,LinaresVasquez:2014vj,Wang:2013ue,Barua:2012gz,Reboucas:2016tw,Allamanis:2013is,Tahir:2018ks}.
Further details on how we approached the design for this study can be found in \textbf{\cref{ch:icse2020}, \cref{icse2020:sec:method}}, \textbf{\cref{ch:semotion2021}, \cref{semotion2021:sec:methodology}}, and \textbf{\cref{ch:caise2021}, \cref{caise2021:sec:method}}

\subsection{Designing Improved Integration Strategies}

Our improved integration strategies (i.e., \ref{rq:design}) evolved organically over the duration of this research through the use of industry case studies and action research. We developed several iterative prototypes to the integration strategies and used a mix of statistical and technical evaluations to analyse whether our improved integration strategies can prove useful. Further details about these approaches are detailed in \textbf{\cref{ch:icwe2019}, \cref{icwe2019:sec:evaluation-method}} and \textbf{\cref{ch:fse2020}, \cref{fse2020:sec:eval}} and \textbf{\cref{ch:fse-demo2020}, \cref{fse-demo2020:sec:threshy}}.

\section{Chapter Summary}

This chapter has explored the research methodology and strategy that is adopted throughout the various studies given within this thesis. We began by revisiting the four primary research questions that were posited in our introductory chapter under \cref{sec:introduction:goals}; as given in \cref{sec:research-methodology:research-questions}, we analysed these questions through the lenses of an existing research question classification taxonomy applicable to software engineering research. We identified which of these questions are grounded through both empirical and non-empirical research, and discussed the underlying reasoning behind the design of each research question. We provided insight into various philosophical stances relevant to software engineering research under \cref{sec:research-methodology:philosophical-stances}, and explained our reasoning for adopting the critical theory worldview in this thesis. Lastly, we reviewed a number of common software engineering research methods in \cref{sec:research-methodology:review,sec:research-methodology:experiments} and those that we adopted in the design of the various experiments described in \cref{part:publications} of this thesis. 