\newcontent
\section{Research Design}
\label{sec:research-methodology:experiments}

This section discusses an overview of the design of methods used within the experiments conducted under this thesis. For each experiment, we describe an overview of the experiment grounded known methods and techniques (\cref{ssec:research-methodology:review:methods,ssec:research-methodology:review:techniques}) and our approach to analysing the data, as well as relating the selecting method back to a specific RQ. Details of each experiment presented in this thesis, the coherency between them, and where they can be found are given in \cref{sec:introduction:organisation,sec:introduction:research-contributions}.

\subsection{Landscape Analysis of Computer Vision Services}

To understand the behavioural and evolutionary profiles of \glspl{cvs} (i.e., \ref{rq:nature}), we employ a longitudinal study based around a dynamic system analysis \citep{Singer:2007tu}. Specifically, we employ structured observations of three services using the same dataset to understand how the responses from these services change with time. Lastly, we employ documentation analysis to assess the overall `picture' of how these services are documented. Further details on this experiment is given in \textbf{\cref{ch:icsme2019}, \cref{icsme2019:sec:method}}.

\subsection{Utility of API Documentation in Computer Vision Services}

To assess whether these services are sufficiently documented (i.e., \ref{rq:docs}), we conduct a systematic mapping study \citep{Kitchenham:2007dd,Petersen:2008td} of the various academic sources detailing \gls{api} documentation knowledge. We then consolidate this information into a structured taxonomy following a systematic taxonomy development method specific to software engineering studies \citep{Usman:2017hn}.

We then follow the triangulation approach proposed by \citet{Jick:1979el} to validate the taxonomy by use of a personal opinion survey. \citet{Kitchenham:2007ux} provide an introduction on methods used to conduct personal opinion surveys which we adopt as an initial reference in (i) shaping our survey objectives around our research goals, (ii) designing a cross-sectional survey, (iii) developing and evaluating our survey instrument, (iv) evaluating our instruments, (v) obtaining the data and (vi) analysing the data. We adapt \citeauthor{Brooke:1996ua}'s systematic usability scale \citep{Brooke:1996ua} technique by basing our research questions against a known surveying instrument.

As is good practice in developing questionnaire instruments to evaluate their reliability and validity \citep{Litwin:1995wt}, we evaluate our instrument design by asking colleagues to critique it via pilot studies within \gls{a2i2}. This assists in identifying any problems with the questionnaire itself and with any issues that may occur with the response rate and follow-up procedures.

Findings from the pilot study helps inform us for a widely distributed questionnaire using snow-balling sampling. Ethics approval from the Faculty of Science, Engineering and Built Environment Human Ethics Advisory Group (SEBE HEAG) has been approved to externally conducting this survey research (see \cref{ch:ethics}). 
Further details on \gls{api} these methods are detailed within \textbf{\cref{ch:tse2020}, \cref{tse2020:sec:method}}.

\subsection{Developer Issues concerning Computer Vision Services}

Developers typically congregate in search of discourses on issues they face in online forums, such as \glsx{so} and Quora, as well as writing their experiences in personal blogs such as Medium. The simplest of these platforms is \gls{so} (a sub-community of the Stack Exchange family of targeted communities) that specifically targets developer issues on using a simple Q\&A interface, where developers can discuss technical aspects and general software development topics. Moreover, \gls{so} is often acknowledged as \textit{the} `go-to' place for developers to find high-quality code snippets that assist in their problems \citep{Subramanian:2014bg}.

Thus, to begin understanding the issues developers face when using \glspl{cvs} and wheteher there is a substantial difference to conventional domains (i.e., \ref{rq:devs}), we propose using repository mining on \gls{so} to help answer our research questions. Specifically, we select \gls{so} due to its targeted community of developers\footnote{We also acknowledge that there are other targeted software engineering Stack Exchange communities such as Stack Exchange Software Engineering (\url{https://softwareengineering.stackexchange.com}), though (as of January 2019) this much smaller community consists of only 52,000 questions versus \gls{so}'s 17 million.} and the availability of its publicly available dataset released as `data dumps' on the Stack Exchange Data Explorer\footnoteurl{https://data.stackexchange.com/stackoverflow}{17 January 2017} and Google BigQuery\footnoteurl{https://console.cloud.google.com/marketplace/details/stack-exchange/stack-overflow}{17 January 2017}. Studies conducted have also used \gls{so} to mine developer discourse \citep{Choi:2015wo,Sinha:2013tt,Novielli:2015vda,Rosen:2016uk,Pal:2012te,Bajaj:2014wg,LinaresVasquez:2014vj,Wang:2013ue,Barua:2012gz,Reboucas:2016tw,Allamanis:2013is,Tahir:2018ks}.
Further details on how we approached the design for this study can be found in \textbf{\cref{ch:icse2020,ch:semotion2020}, \cref{icse2020:sec:method,sec:semotion2020:method}}.

% TODO: Fix.
%Due to the enormity of the data produced, we will use qualitative analysis on the questions mined using assistive tools such as NVivo. For this, we will conduct a thematic analysis on the themes of each question mined, the relevance of the question to our research topic, and ensuring strict coding schemes (that reflect our research goals) are adhered to. We refer to \citet{Singer:2007tu} and \citet{Miles:1994ty} on coding and analysing this qualitative data gathered.

\subsection{Designing Improved Integration Strategies}

Our improved integration strategies (i.e., \ref{rq:design}) evolved organically over the duration of this research project through the use of industry case studies and action research. We develop several iterative prototypes and use a mix of statistical and technical-expert assessment to analyse whether our improved integration stratiges can prove useful to developers. Further details about these approaches are detailed in \textbf{\cref{ch:icwe2019,ch:icse-demo2020,ch:fse2020}, \cref{icwe2019:sec:evaluation-method,icse2020-demo:sec:method,fse2020:sec:method}}. \todo{Add more detail later}

\oldcontent

%Experiment I shapes a context-agnostic approach to understand current usage patterns of \gls{iws} \glspl{api} and the ways by which developers interpret them. Briefly, this experiment is comprised under two phases of field survey research: (i) repository mining developer discussion forums (i.e., analysis of databases and documentation analysis) to understand what developers currently complain about on these forums and where their mismatch in understanding lies; (ii) conducting unstructured interviews and distributing a questionnaire to gather personal opinion based on individual developer's anecdotal remarks.
%
%\subsubsection{Relevance and Motivation}
%
%Experiment I aims to better understand the existing mindsets that developers have when approaching to use \glsplx{cvs}. This work therefore ties in to \rh{1}; by understanding the developer mindset in how they interpret \gls{cvs} \glspl{api}, we are better informed to produce a framework that increases the effectiveness of the documentation of those existing \gls{cvs} providers.
%
%\rh{1} postulates that the software engineering community do not fully understand the `magic' behind \gls{iws} \glspl{api}. As described in \cref{sec:introduction:goals}, they face a gap in their understanding around the underlying architecture of pre-built, machine learnt \glspl{api} (\ref{rqs:apidoc:how-do-devs-understand-it}). Software developers are not well-supported by the \gls{iws} providers, and therefore do not have a consistent set of common best practices when approaching to use these \glspl{api} (\ref{rqs:apidoc:what-is-in-use}). It is therefore necessary that \gls{iws} providers provide additional information to gap this mismatched understanding (\ref{rqs:apidoc:what-additional-information-needed}). 
%
%\subsubsection{Data Collection \& Analysis}
%
%\paragraph{Phase 1: Repository Mining}
%Developers typically congregate in search of discourses on issues they face in online forums, such as \glsx{so} and Quora, as well as writing their experiences in personal blogs such as Medium. The simplest of these platforms is \gls{so} (a sub-community of the Stack Exchange family of targeted communities) that specifically targets developer issues on using a simple Q\&A interface, where developers can discuss technical aspects and general software development topics. Moreover, \gls{so} is often acknowledged as \textit{the} `go-to' place for developers to find high-quality code snippets that assist in their problems \citep{Subramanian:2014bg}.
%
%Thus, to begin validating \gls{iws} \gls{api} usage and misunderstanding in a generalised context (i.e., context-agnostic to the project at hand), we propose using repository mining on \gls{so} to help answer our research questions. Specifically, we select \gls{so} due to its targeted community of developers\footnote{We also acknowledge that there are other targeted software engineering Stack Exchange communities such as Stack Exchange Software Engineering (\url{https://softwareengineering.stackexchange.com}), though (as of January 2019) this much smaller community consists of only 52,000 questions versus \gls{so}'s 17 million.} and the availability of its publicly available dataset released as `data dumps' on the Stack Exchange Data Explorer\footnoteurl{https://data.stackexchange.com/stackoverflow}{17 January 2017} and Google BigQuery\footnoteurl{https://console.cloud.google.com/marketplace/details/stack-exchange/stack-overflow}{17 January 2017}. Studies conducted have also used \gls{so} to mine developer discourse \citep{Choi:2015wo,Sinha:2013tt,Novielli:2015vda,Rosen:2016uk,Pal:2012te,Bajaj:2014wg,LinaresVasquez:2014vj,Wang:2013ue,Barua:2012gz,Reboucas:2016tw,Allamanis:2013is,Tahir:2018ks}.
%
%Due to the enormity of the data produced, we will use qualitative analysis on the questions mined using assistive tools such as NVivo. For this, we will conduct a thematic analysis on the themes of each question mined, the relevance of the question to our research topic, and ensuring strict coding schemes (that reflect our research goals) are adhered to. We refer to \citet{Singer:2007tu} and \citet{Miles:1994ty} on coding and analysing this qualitative data gathered.
%
%\paragraph{Phase 2: Personal Opinion Surveys}
%We follow the triangulation approach proposed by \citet{Jick:1979el} to corroborate the qualitative data of developers' discussion of \gls{so} with secondary survey research, thereby validating what people say on git with what is said and done in real life. \citet{Kitchenham:2007ux} provide an introduction on methods used to conduct personal opinion surveys which we adopt as an initial reference in (i) shaping our survey objectives around our research goals, (ii) designing a cross-sectional survey, (iii) developing and evaluating our two survey instruments (consisting of a structured questionnaire and semi-structured interview), (iv) evaluating our instruments, (v) obtaining the data and (vi) analysing the data.
%
%As is good practice in developing questionnaire instruments to evaluate their reliability and validity \citep{Litwin:1995wt}, we evaluate our instrument design by asking colleagues to critique it via pilot studies within \gls{a2i2}. This assists in identifying any problems with the questionnaire itself and with any issues that may occur with the response rate and follow-up procedures. We follow a similar approach by practicing the interview instrument on colleagues within \gls{a2i2}.
%
%Findings from the pilot study helps inform us for a widely distributed questionnaire and conducting interviews out in the field, where we recruit external software engineers in industry through the industry contacts of \gls{a2i2}. Ethics approval from the Faculty of Science, Engineering and Built Environment Human Ethics Advisory Group (SEBE HEAG) will be required prior to externally conducting this survey research (see \cref{ch:ethics}). The quantitative (survey) and qualitative (interview) analysis allows us to shape the research outcome of \rh{1}---an \gls{api} documentation quality assessment framework---and assists in stabilising our general understanding of how developers use these existing \glspl{api}.
%
%\subsection{Developing the Initial Framework}
%
%Our initial framework is developed using the qualitative and quantitative analyses from the findings of Experiment I. As this is a creative phase in which we are developing a new framework, the exact process by which we develop the initial framework will come to light once more insight is determined. However it is anticipated discussion with other researchers and engineers at \gls{a2i2} about the analyses of the findings (i.e., white-boarding sessions of potential ideas from the findings) will help develop our initial documentation framework. This framework will take the shape of a checklist or table, typical of information systems studies (e.g., \citep{Lau:1999vs}), that indicate what attributes should be best suited for what needs.
%
%\subsection{Experiment II: Validate Initial Framework}
%\label{ssec:research-methodology:experiments:2}
%
%Experiment II extends the \textit{generalised} context of Experiment I by evaluating how the findings of Experiment I translates to context-specific applications. We confirm that the generalised findings are (indeed) genuine by conducting action research in combination with an observational study on software engineers. This experiment is also compromised of: (i) development of prototypes using \gls{iws} \glspl{api} of differing contexts; (ii) presenting a solution framework to developers to interpret the improvement of their understanding when using an \gls{iws}.
%
%\subsubsection{Relevance and Motivation}
%
%Experiment II aims to contextualise the findings from Experiment I; that is, if we add \textit{varying contexts} to the applications we write using \gls{iws} \glspl{api}, what is needed to extend the \textit{context-agnostic} framework developed in Experiment I? This work relates back to \rh{2}; adding context-specific metadata to the endpoints of these \glspl{api}, we can highlight what issues exist when such metadata is not present (\ref{rqs:metadata:what-problems-du                                                                e-to-lack-of-metadata}) and what types of metadata developers seek (\ref{rqs:metadata:what-metadata-do-devs-want-and-why}).
%
%Moreover, the implication of the first two hypotheses suggest that applying an \gls{api} documentation and metadata quality assessment framework may have an effect on other aspects within the software engineering process (\rh{3}). Thus, this experiment also confirms if our framework makes an improvement to software quality, developer productivity and/or developer informativeness (\ref{rqs:implications:do-metrics-improve} and \ref{rqs:implications:aspects}).
%
%\subsubsection{Data Collection \& Analysis}
%
%To confirm findings of the method within \rh{1} is genuine, we shift from reviewing the documentation from a general stance to a specialised (context-specific) stance in the use of these \glspl{api}.
%
%This is firstly achieved by using existing \gls{iws} \glspl{api} to develop basic `prototypes', each having differing contexts. The number of prototypes to develop and the use cases they have will be informed by the results of Experiment I, and therefore cannot yet be described at this stage. Our action research in developing the prototypes will help inform any potential gaps that exist in the findings of \rh{1}, especially with regards to context-specifity, and therefore improves the metadata component of our framework (as per the outcome of \rh{2}). 
%
%This outcome will also help us design the next stage of the experiment, consisting of a comparative controlled study \citep{Seaman:2007wa} to capture firsthand behaviours and interactions toward how software engineers approach using an \gls{iws} with and without our framework applied. We will provide improved documentation and metadata responses of a set of popular \glspl{cvs} that is documented with the additional metadata and whose information is organised using our framework. 
%
%We then recruit 20 developers of varying experience (from beginner programmer to principal engineer) to complete five tasks under an observational, comparative controlled study, 10 of which will (a) develop with the \textit{new} framework, and the other 10 will (b) develop with the \textit{as-is/existing} documentation. From this, we compare if the framework makes improvements by capturing metrics and recording the observational sessions for qualitative analysis. We use visual modelling to analyse the qualitative data using matrices \citep{Dey:2003ty}, maps and networks \citep{Miles:1994ty} as these help illustrate any causal, temporal or contextual relationships that may exist to map out the developer's mindset and difference in approaching the two sets of designs of the same tasks.
%
%To validate the findings of developer opinion in the surveys and interviews of \rh{1} are indeed genuine, this helps ensure that there is nothing missing by adding in further context to such opinions.