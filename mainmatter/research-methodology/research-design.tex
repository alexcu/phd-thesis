\section{Research Design}
\label{sec:research-methodology:experiments}

This section discusses an overview of the design of methods used within the experiments conducted under this thesis. For each experiment, we describe an overview of the experiment grounded known methods and techniques (\cref{ssec:research-methodology:review:methods,ssec:research-methodology:review:techniques}) and our approach to analysing the data, as well as relating the selecting method back to a specific RQ. Details of each experiment presented in this thesis, the coherency between them, and where they can be found are given in \cref{sec:introduction:organisation,sec:introduction:research-contributions}.

\subsection{Landscape Analysis of Computer Vision Services}

To understand the behavioural and evolutionary profiles of \glspl{cvs} (i.e., \ref{rq:nature}), we employ a longitudinal study based around a dynamic system analysis \citep{Singer:2007tu}. Specifically, we employ structured observations of three services using the same dataset to understand how the responses from these services change with time. Lastly, we employ documentation analysis to assess the overall `picture' of how these services are documented. Further details on this experiment is given in \textbf{\cref{ch:icsme2019}, \cref{icsme2019:sec:method}}.

\subsection{Utility of API Documentation in Computer Vision Services}

To assess whether these services are sufficiently documented (i.e., \ref{rq:docs}), we conduct a systematic mapping study \citep{Kitchenham:2007dd,Petersen:2008td} of the various academic sources detailing \gls{api} documentation knowledge. We then consolidate this information into a structured taxonomy following a systematic taxonomy development method specific to software engineering studies \citep{Usman:2017hn}.

We then follow the triangulation approach proposed by \citet{Jick:1979el} to validate the taxonomy by use of a personal opinion survey. \citet{Kitchenham:2007ux} provide an introduction on methods used to conduct personal opinion surveys which we adopt as an initial reference in (i) shaping our survey objectives around our research goals, (ii) designing a cross-sectional survey, (iii) developing and evaluating our survey instrument, (iv) evaluating our instruments, (v) obtaining the data and (vi) analysing the data. We adapt \citeauthor{Brooke:1996ua}'s systematic usability scale \citep{Brooke:1996ua} technique by basing our research questions against a known surveying instrument.

As is good practice in developing questionnaire instruments to evaluate their reliability and validity \citep{Litwin:1995wt}, we evaluate our instrument design by asking colleagues to critique it via pilot studies within \gls{a2i2}. This assists in identifying any problems with the questionnaire itself and with any issues that may occur with the response rate and follow-up procedures.

Findings from the pilot study helps inform us for a widely distributed questionnaire using snow-balling sampling. Ethics approval from the Faculty of Science, Engineering and Built Environment Human Ethics Advisory Group (SEBE HEAG) has been approved to externally conducting this survey research (see \cref{ch:ethics}). 
Further details on \gls{api} these methods are detailed within \textbf{\cref{ch:tse2020}, \cref{tse2020:sec:method}}.

\subsection{Developer Issues concerning Computer Vision Services}

Developers typically congregate in search of discourses on issues they face in online forums, such as \glsx{so} and Quora, as well as writing their experiences in personal blogs such as Medium. The simplest of these platforms is \gls{so} (a sub-community of the Stack Exchange family of targeted communities) that specifically targets developer issues on using a simple Q\&A interface, where developers can discuss technical aspects and general software development topics. Moreover, \gls{so} is often acknowledged as \textit{the} `go-to' place for developers to find high-quality code snippets that assist in their problems \citep{Subramanian:2014bg}.

Thus, to begin understanding the issues developers face when using \glspl{cvs} and wheteher there is a substantial difference to conventional domains (i.e., \ref{rq:devs}), we propose using repository mining on \gls{so} to help answer our research questions. Specifically, we select \gls{so} due to its targeted community of developers\footnote{We also acknowledge that there are other targeted software engineering Stack Exchange communities such as Stack Exchange Software Engineering (\url{https://softwareengineering.stackexchange.com}), though (as of January 2019) this much smaller community consists of only 52,000 questions versus \gls{so}'s 17 million.} and the availability of its publicly available dataset released as `data dumps' on the Stack Exchange Data Explorer\footnoteurl{https://data.stackexchange.com/stackoverflow}{17 January 2017} and Google BigQuery\footnoteurl{https://console.cloud.google.com/marketplace/details/stack-exchange/stack-overflow}{17 January 2017}. Studies conducted have also used \gls{so} to mine developer discourse \citep{Choi:2015wo,Sinha:2013tt,Novielli:2015vda,Rosen:2016uk,Pal:2012te,Bajaj:2014wg,LinaresVasquez:2014vj,Wang:2013ue,Barua:2012gz,Reboucas:2016tw,Allamanis:2013is,Tahir:2018ks}.
Further details on how we approached the design for this study can be found in \textbf{\cref{ch:icse2020,ch:semotion2020}, \cref{icse2020:sec:method,sec:semotion2020:method}}.

\subsection{Designing Improved Integration Strategies}

Our improved integration strategies (i.e., \ref{rq:design}) evolved organically over the duration of this research project through the use of industry case studies and action research. We develop several iterative prototypes and use a mix of statistical and technical-expert assessment to analyse whether our improved integration stratiges can prove useful to developers. Further details about these approaches are detailed in \textbf{\cref{ch:icwe2019,ch:fse-demo2020,ch:fse2020}, \cref{icwe2019:sec:evaluation-method,icse2020-demo:sec:method,fse2020:sec:method}}. \todo{Add more detail later}
