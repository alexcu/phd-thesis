\section{Research Questions Revisited}
\label{sec:research-methodology:research-questions}

%In \cref{sec:introduction:goals}, we rq:nature:runtime to three hypotheses: (i) \glspl{cvs} have non-deterministic properties that confuse developers with a deterministic mindset; (ii) the implicit risks to using these services without fully appreciating their unintended side effects has an impact to software quality; (iii) the service providers do not sufficiently document these behavioural and evolution issues.

To discuss our research strategy, we revisit our \NumPrimaryRQs{} primary and \NumSecondaryRQs{} secondary research questions (RQs) through the classification technique discussed by \citet{Easterbrook:2007ws}, a technique originally proposed in the field of psychology by \citet{Meltzoff:1998wg} but adapted to software engineering. A summary of the classifications made to our research questions are presented in \cref{tab:research-methodology:rqs}.

Our research study involves a mix of \NumEmpiricalRQs{} \textit{empirical}\footnote{Or `knowledge' questions, that extend our \textit{knowledge} on certain phenomena.} RQs, that focus on observing and analysing existing phenomena, and \NumNonEmpiricalRQs{} \textit{non-empirical} RQs, that focuses on designing better approaches to solve software engineering tasks \citep{Simon:1996uw}. The use of empirical \textit{and} non-empirical RQs are best combined in long-term software engineering research studies where the phenomena are under-explored, as is the case with \glspl{cvs}. Further, these approaches help propose solutions to issues found in the phenomena studied \citep{Wieringa:2006vd}. We discuss both our empirical and non-empirical RQs in \cref{ssec:research-methodology:research-questions:empirical,ssec:research-methodology:research-questions:nonempirical} below. 

\begin{table}[tbh]
\centering
\caption[Classification of research questions in this thesis]{A summary of our research questions classified using the strategies presented by \citet{Easterbrook:2007ws} and \citet{Meltzoff:1998wg}.}\label{tab:research-methodology:rqs}
\tablefit{\begin{tabular}{p{0.07\linewidth}p{0.55\linewidth}|p{0.13\linewidth}p{.33\linewidth}}
\toprule
\textbf{\#}&
\textbf{RQ}&
\textbf{Primary/ Secondary}&
\textbf{RQ Classification}
\\
\midrule
\midrule
\ref{rq:nature} &
\RQOneTextLandscapeAnalysis{} &
Primary &
\makecell[tl]{
\textsc{Empirical}\\
$\hookrightarrow$~\textit{Exploratory}\\
$~~\hookrightarrow$~\textit{Description/Classification}
}
\\
\ref{rq:nature:runtime} &
\RQOneTextLandscapeAnalysisRuntime{} &
Secondary &
\makecell[tl]{
\textsc{Empirical}\\
$\hookrightarrow$~\textit{Exploratory}\\
$~~\hookrightarrow$~\textit{Description/Classification}
}
\\
\ref{rq:nature:evolution} &
\RQOneTextLandscapeAnalysisEvolution{} &
Secondary &
\makecell[tl]{
\textsc{Empirical}\\
$\hookrightarrow$~\textit{Exploratory}\\
$~~\hookrightarrow$~\textit{Description/Classification}
}
\\
\midrule
\ref{rq:docs} &
\RQTwoTextDocumentation{}&
Primary &
\makecell[tl]{
\textsc{Empirical}\\
$\hookrightarrow$~\textit{Exploratory}\\
$~~\hookrightarrow$~\textit{Existence}
}
\\
\ref{rq:docs:complete} &
\RQTwoTextDocumentationWhatIsCompleteDocs{}&
Secondary &
\makecell[tl]{
\textsc{Empirical}\\
$\hookrightarrow$~\textit{Exploratory}\\
$~~\hookrightarrow$~\textit{Composition}
}
\\
\ref{rq:docs:missing} &
\RQTwoTextDocumentationMissingAttributes{}&
Secondary &
\makecell[tl]{
\textsc{Non-Empirical}\\
$\hookrightarrow$~\textit{Design}\\
}
\\
\midrule
\ref{rq:devs} &
\RQThreeTextDevMiscomprehension{}&
Primary &
\makecell[tl]{
\textsc{Empirical}\\
$\hookrightarrow$~\textit{Exploratory}\\
$~~\hookrightarrow$~\textit{Descriptive-Comparative}
}
\\
\ref{rq:devs:issues} &
\RQThreeTextDevMiscomprehensionIssueTypes{}&
Secondary &
\makecell[tl]{
\textsc{Empirical}\\
$\hookrightarrow$~\textit{Base-Rate}\\
$~~\hookrightarrow$~\textit{Frequency/Distribution}
}
\\
\ref{rq:devs:frustration} &
\RQThreeTextDevMiscomprehensionFrustration{} &
Secondary &
\makecell[tl]{
\textsc{Empirical}\\
$\hookrightarrow$~\textit{Exploratory}\\
$~~\hookrightarrow$~\textit{Description/Classification}
}
\\
\ref{rq:devs:vs-traditional} &
\RQThreeTextDevMiscomprehensionVsConventional{} &
Secondary &
\makecell[tl]{
\textsc{Empirical}\\
$\hookrightarrow$~\textit{Base-Rate}\\
$~~\hookrightarrow$~\textit{Frequency/Distribution}
}
\\
\midrule
\ref{rq:design} &
\RQFourDesign{} &
Primary &
\makecell[tl]{
\textsc{Non-Empirical}\\
$\hookrightarrow$~\textit{Design}\\
}
\\
\bottomrule
\end{tabular}}
\end{table}


\subsection{Empirical Research Questions}
\label{ssec:research-methodology:research-questions:empirical}

In total, we pose \NumEmpiricalRQs{} empirically-based RQs to help us understand the way developers currently interact and work with web services that provide computer vision. The majority of these questions are \textit{exploratory} questions that contribute to a landscape analysis of these services (\ref{rq:nature}), how well they are documented (\ref{rq:docs}), and the issues developers currently face when using them (\ref{rq:devs}). Our other exploratory questions complement the answers to these questions. For instance, to understand if \glspl{cvs} are sufficiently documented (an \textit{existence} exploratory question posed in \ref{rq:docs}), we need to understand the components of a `sufficient' or `complete' \gls{api} document (via \ref{rq:docs:complete}) as proposed in both the literature (i.e., where the majority of research effort has been placed by the research community) and by software developers (i.e., by directly asking which aspects are needed to developers themselves). While \ref{rq:docs:complete} does not directly relate to \glspl{cvs}, answering it gives us an understanding of the components of complete \gls{api} documentation, and therefore, we assess what documentation artefacts are missing and where improvements can be made (\ref{rq:docs:missing}). These  are \textit{descriptive and classification} questions that help describe and classify what practices are in use for existing \gls{cvs} \gls{api} documentation and the nature behind these services. Answering these exploratory questions assists in refining preciser terms of the phenomena, ways in which we find evidence for them, and ensuring the data found is valid.

By answering these questions, we have a clearer understanding of the phenomena; we then follow up by posing \NumBaseRateRQs{} additional \textit{base-rate questions}. These questions help provide a basis to confirm that the phenomena is normally occurring or whether it is unusual behaviour. This is done by investigating the patterns of phenomena's occurrence against other phenomena. \ref{rq:devs:issues} is a \textit{frequency and distribution} question to help us understand what types of issues developers often encounter most, given a lack of formal extended training in \gls{ai}. This provides insight into the developer's mindset and regular thought patterns toward these \glspl{api}. We then contrast this distribution using our second base-rate question (\ref{rq:devs:vs-traditional}) that assesses the distributional differences between these intelligent components and non-intelligent (conventional) software components. Combined, these two questions help us answer how the issues raised against \glspl{cvs} are different to normal \glslong{so} issues---our \textit{descriptive-comparative} question posed in \ref{rq:devs}---and, similarly, we classify and rank which issues developers find most frustrating (\ref{rq:devs:frustration}).

%Lastly, we investigate the relationship between the improved documentation and improvements to other aspects of the software development process. Chiefly, \ref{rqs:implications:do-metrics-improve} is concerned with whether any improvements to metadata or documentation correlate to improvements in software quality, developer productivity, or developer education (and is a \textit{relationship establishment question}). If we establish such a relationship, we refine the question and investigate the specific causes using three \textit{causality questions} defined under \ref{rqs:implications:aspects}, namely by associating three classes of measurable metrics (internal quality metrics, external quality metrics, developer education insight metrics) to the improved documentation.

\subsection{Non-Empirical Research Questions}
\label{ssec:research-methodology:research-questions:nonempirical}

\ref{rq:docs:missing} and \ref{rq:design} are both non-empirically-based \textit{design questions}; they are concerned with ways in which we can improve a \gls{cvs} by investigating what additional attributes are needed in both the documentation of \glspl{cvs} and in the integration architectures developers can employ to improve reliability and robustness in their applications. They are not classified as empirical questions as we investigate what \textit{will be} and not \textit{what is}. By understanding the process by which developers desire additional attributes of documentation and integration strategies, we help shape improvements to the existing designs of using \glspl{cvs}.