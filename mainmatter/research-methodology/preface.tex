\label{sec:research-methodology:preface}

Investigating software engineering practices is oft a complex task as it is imperative to understand the social and cognitive processes around software engineers and not just the tools and processes used \citep{Easterbrook:2007ws}. This chapter explores our research methodology by exploring five key elements of empirical software engineering research: firstly, (i) we provide an extended focus to the study by reviewing our research questions (see \cref{sec:introduction:hypohtesis}) anchored under the context of an existing classification taxonomy, (ii) characterise our research goals through an explicit philosophical stance, (iii) explain how the stance selected impacts our selection of research methods and data collection techniques (by dissecting our choice of methods used to reach these research goals), (iv) discuss a set of criteria for assessing the validity of our study design and the findings of our research, and lastly (v) discuss the practical considerations of our chosen methods. 

The foundations for developing this research methodology has been expanded from that proposed by \citet{Easterbrook:2007ws}, \citet{Wohlin:2014jq}, \citet{Wohlin:2012bu} and \citet{Shaw:2003aa}.