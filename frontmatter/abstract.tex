\chapter*{Abstract}
\addcontentsline{toc}{chapter}{Abstract}

\glsunset{ai}

% General statement introducing the broad research area
Application developers are eager to integrate machine learning (ML) into their software, with a plethora of vendors providing pre-packaged components---typically under the artificial intelligence (AI) banner---to entice them.
% An explanation of the specific problem (difficulty, obstacle, challenge)
Such components are marketed as developer `friendly' ML and easy for them to integrate (being `just another' component added to their toolchain). These components are, however, non-trivial: in particular, developers unknowingly add the risk of mixing non-deterministic ML behaviour into their applications that, in turn, impact the quality of their software.
% Review of existing or standard solutions to this problem
Prior research advocates that a developer's conceptual understanding is critical to effective interpretation of reusable components. However, these ready-made AI components do not present sufficient detail to allow developers to acquire this conceptual understanding.
% Outline of the proposed new solution
In this thesis, we address the clash of mindsets between an application developers' deterministic approach to software development and the mindset needed to incorporate probabilistic, intelligent components, namely cloud-based intelligent web services. (We scope our investigation to the most mature subset of these services: those that provide computer vision intelligence via RESTful web APIs.) We explore how this mindset clash ultimately impacts the reliability of the software developers produce and these concerns are addressed via four research perspectives answered by seven sub-research questions.

% Overview of RQs, methods, and findings...
Firstly, we present a landscape analysis of the nature of these cloud-based intelligent components. What is their runtime behaviour and evolution profile? We performed periodic structured observations against three prominent computer vision services over an 11 month longitudinal study and found inconsistencies in how these services behave and substantial evolution risk in the labels and confidence scores they present. This study is further explored in \cref{ch:icsme2019}.
Secondly, we explore the sufficiency of the documentation presented in these services. Which attributes of `sufficient' API documentation is explored by literature, what is the efficacy of these attributes according to developers, and which of these attributes are currently missing from these services? We perform a triangulation study by (i) synthesising a taxonomy of 34 facets a `complete' API document should sourced via a systematic mapping study resulting in 21 seminal academic studies, (ii) assessing the efficacy of this taxonomy using a survey loosely based on the system usability scale instrument that was distributed to 83 software engineers, and (iii) applying the taxonomy (as assessed by engineers) against three popular computer vision services to produce 12 suggested improvements to the service documentation. This study is further explored in \cref{ch:tse2020}.
Thirdly, we analyse whether these services are more misunderstood than conventional software engineering domains. Which types of issues do developers face most when working with intelligent services, and which of these issues are developers most frustrated with? Is the distribution of these issues different to more established software development domains? We conduct a mining study of the popular software development discussion forum Stack Overflow resulting in from 1,825 posts to analyse the various pain-points developers face, as classified from two existing classification strategies from the literature. We find that developers raise issues due to a primitive understanding of the underlying concepts within these services---which results in conceptual misunderstanding and confusion in interpreting service errors---and the distribution of the types of frustrations raised are substantially different to more mature domains. This study is further explored in \cref{ch:icse2020,ch:semotion2020}.
Lastly, we propose several strategies targeted at better integrating intelligent components into developer's software whilst preserving robustness. These strategies are outlined within \cref{ch:icwe2019,ch:fse-demo2020,ch:fse2020}.

This thesis presents a substantial contribution to the software engineering discipline by presenting a better understanding how AI-based components have non-trivial implications to software reliability, robustness and completeness which arise due to developer misunderstandings. Further, we present a novel service integration architecture by which developers can use to integrate their applications with these AI-based components to reduce any potential risk to the quality of their software. Lastly, this thesis contributes a key list of attributes that should be documented with any API, chiefly to assist service providers on how better to document their services. 