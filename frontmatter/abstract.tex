\cleardoubleevenemptypage
\pdfbookmark[1]{Abstract}{abstract}
\phantomsection
\addcontentsline{toc}{chapter}{Abstract}

\thispagestyle{empty}

\vspace*{\fill}
\hspace{\fill}\textsc{\textbf{abstract}}\hspace{\fill}
\bigskip

\noindent\small
Software components hide implementation complexity by exposing a designed interface that permits easy integration and use. The explosive demand and interest in \gls{ai} and deep learning has led to creation of software components that offer various \gls{ml} functions. The promise is that these \gls{ai} components improve productivity and application developers can use them without a deep understanding of their underlying mechanics. Application developers currently have access to multiple \gls{ai} components with a prominent focus on visual object recognition, natural language processing, audio analysis, anomaly detection and forecasting from numerical data. Simplified variations of these components are offered via cloud computing as intelligent web services; these services are often marketed as `developer friendly' \gls{ml} with the claim of being just another component accessible on the cloud through a web-based \glsac{rest}ful \glsac{api}.

\medskip\noindent
A developer's conceptual understanding of components they use impacts the internal and external quality of software they produce. Hence, vendors of intelligent web services must give sufficient level of conceptual detail to enable integration and effective use of their pre-packaged capabilities, ultimately to help developers who integrate with their services produce high-quality software. 

\medskip\noindent
This thesis investigates these emerging intelligent web services. Based on an analysis of the observable behaviour of intelligent web services, we show that their probabilistic results and evolution is not effectively communicated in the documentation. Our work shows that developers interpret and use these services using anchors built upon their understanding of traditional (i.e., deterministic and non-probabilistic) software components. We show how this mismatch results in a weak conceptual understanding of highly-abstracted forms of \gls{ml}, impacting software quality. To mitigate documentation issues, we propose a taxonomy of the key requirements of good \glsac{api} documentation, which we derive from existing literature and triangulated through a survey with developers. We use this information to assess the value placed by developers on each \glsac{api} documentation artefact and identify gaps in the services' documentation, which can be improved to assist conceptual understanding. Additionally, we propose an architectural tactic designed to reduce and guard against common issues identified when \gls{ml} becomes highly-abstracted. The proposed tactic is intended to better integrate conventional software components with probabilistic and non-deterministic intelligent web services, ultimately to improve overall solution robustness and, thus, software quality.

\medskip\noindent
This thesis makes a substantial contribution to the software engineering discipline by showing the non-trivial implications to software quality resulting from improper usage of such services and offers a pathway to safer use of the exciting new advances from the field of \gls{ai} and deep learning.


\vspace*{\fill}
\normalsize

% % General statement introducing the broad research area
% Application developers are eager to integrate machine learning (\gls{ml}) into their software, with a plethora of vendors providing pre-packaged components---typically under the artificial intelligence (\gls{ai}) banner---to entice them.
% % An explanation of the specific problem (difficulty, obstacle, challenge)
% Such components are marketed as developer `friendly' \gls{ml} and easy for them to integrate (being `just another' component added to their toolchain). These components are, however, non-trivial: in particular, developers unknowingly add the risk of mixing non-deterministic \gls{ml} behaviour into their applications that, in turn, impact the quality of their software.
% % Review of existing or standard solutions to this problem
% Prior research advocates that a developer's conceptual understanding is critical to effective interpretation of reusable components. However, these ready-made \gls{ai} components do not present sufficient detail to allow developers to acquire this conceptual understanding.
% % Outline of the proposed new solution
% In this thesis, we address the clash of mindsets between an application developers' deterministic approach to software development and the mindset needed to incorporate probabilistic, intelligent components, namely cloud-based intelligent web services. (We scope our investigation to the most mature subset of these services: those that provide computer vision intelligence via RESTful web APIs.) We explore how this mindset clash ultimately impacts the reliability of the software developers produce and these concerns are addressed via four research perspectives answered by seven sub-research questions.

% % Overview of RQs, methods, and findings...
% Firstly, we present a landscape analysis of the nature of these cloud-based intelligent components. What is their runtime behaviour and evolution profile? We performed periodic structured observations against three prominent computer vision services over an 11 month longitudinal study and found inconsistencies in how these services behave and substantial evolution risk in the labels and confidence scores they present. This study is further explored in \cref{ch:icsme2019}.
% Secondly, we explore the sufficiency of the documentation presented in these services. Which attributes of `sufficient' API documentation is explored by literature, what is the efficacy of these attributes according to developers, and which of these attributes are currently missing from these services? We perform a triangulation study by (i) synthesising a taxonomy of 34 facets a `complete' API document should sourced via a systematic mapping study resulting in 21 seminal academic studies, (ii) assessing the efficacy of this taxonomy using a survey loosely based on the system usability scale instrument that was distributed to 83 software engineers, and (iii) applying the taxonomy (as assessed by engineers) against three popular computer vision services to produce 12 suggested improvements to the service documentation. This study is further explored in \cref{ch:tse2020}.
% Thirdly, we analyse whether these services are more misunderstood than conventional software engineering domains. Which types of issues do developers face most when working with intelligent services, and which of these issues are developers most frustrated with? Is the distribution of these issues different to more established software development domains? We conduct a mining study of the popular software development discussion forum Stack Overflow resulting in from 1,825 posts to analyse the various pain-points developers face, as classified from two existing classification strategies from the literature. We find that developers raise issues due to a primitive understanding of the underlying concepts within these services---which results in conceptual misunderstanding and confusion in interpreting service errors---and the distribution of the types of frustrations raised are substantially different to more mature domains. This study is further explored in \cref{ch:icse2020,ch:semotion2021}.
% Lastly, we propose several strategies targeted at better integrating intelligent components into developer's software whilst preserving robustness. These strategies are outlined within \cref{ch:icwe2019,ch:fse-demo2020,ch:fse2020}.

% This thesis presents a substantial contribution to the software engineering discipline by presenting a better understanding how \gls{ai}-based components have non-trivial implications to software reliability, robustness and completeness which arise due to developer misunderstandings. Further, we present a novel service integration architecture by which developers can use to integrate their applications with these \gls{ai}-based components to reduce any potential risk to the quality of their software. Lastly, this thesis contributes a key list of attributes that should be documented with any API, chiefly to assist service providers on how better to document their services. 