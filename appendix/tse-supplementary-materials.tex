\chapter{Supplementary Materials to \cref{ch:tse2020}}
\label{ch:tse-supplementary-materials}
\cleardoublepage

\def\circlenotpresent{\faCircleO}
\def\circlepartialpresent{\faAdjust}
\def\circlepresent{\faCircle}
\newcommand{\dimcat}[1]{\textsc{\small[\textbf{#1}]}}

% Taxonomy Dimension Keys
\def \dima{Descriptions of API Usage}
\def \dimb{Descriptions of Design Rationale}
\def \dimc{Descriptions of Domain Concepts}
\def \dimd{Existence of Support Artefacts}
\def \dime{Overall Presentation of Documentation}

% Suggested improvements
\def\SuggestedImprovement{\noindent\itshape\small\textbf{\faHandORight{} Suggested improvement:~}}

\section{Detailed Overview of Our Proposed Taxonomy}
\label[appendixsec]{tse2020:tab:taxonomy}

The following pages detail our proposed taxonomy. Detailed descriptions of the five requirements of good API documentation (dimensions) and 34 generalised API documentation artefacts (categories/sub-dimensions) that help satisfy these requirements within our proposed taxonomy. Descriptions of examples of these documentation artefacts are italicised and provided for illustrative purposes. ILS = In-Literature Score, calculated as a ratio of papers that investigated or reported various issues concerning each artefact. IPS = In-Practice Score, calculated as the average response from our survey instrument. Colour scales indicate relevancy weight within ILS or IPS values for comparative purposes, where red = \textit{lowest} and green = \textit{highest}. GCV, AWS, ACV = Presence of category in Google Cloud Vision, Amazon Rekognition, and Azure Cloud Vision documentation. Presence indicated as \textit{fully present} (\circlepresent{}), \textit{partially present} (\circlepartialpresent{}), and \textit{not present}~(\circlenotpresent{}).

\begin{landscape}
{\def\cn{}
\def\cy{\checkmark}
\footnotesize
\begin{longtable}{rp{0.45\linewidth}|p{0.125\linewidth}|cc|ccc}
%   \caption[Taxonomy proposed in API documentation knowledge study]{}
  %\label{tse2020:tab:taxonomy}\\
  \toprule
  \textbf{Key} &
  \textbf{Description} &
  \textbf{Primary Sources} &
  \textbf{ILS} &
  \textbf{IPS} & 
  \textbf{GCV} &
  \textbf{AWS} &
  \textbf{ACV} \\
  \midrule
  \midrule
  \endfirsthead
%   \caption*{An overview of the 5 dimensions and categories (sub-dimensions) within our proposed taxonomy \textit{(Continued)}.}\\
  \toprule
  \textbf{Key} &
  \textbf{Description} &
  \textbf{Primary Sources} &
  \textbf{ILS} &
  \textbf{IPS} &
  \textbf{GCV} &
  \textbf{AWS} &
  \textbf{ACV} \\
  \midrule
  \midrule
  \endhead
  \bottomrule
  
  \multicolumn{8}{r}{\textit{Continued on next page...}}\\
  \endfoot
  \bottomrule
  \endlastfoot

%%% BEGIN PASTE TAX HERE

   \textbf{A}&
  \multicolumn{7}{l}{\textbf{Requirement 1: API Documentation should include Descriptions of API Usage}}\\
  \midrule

  A1&
  Quick-start guides; \textit{i.e., a guide to rapidly get started using the API in a specific programming language.}&
  \scriptsize S4, S9, S10 &
  \cellcolor[HTML]{f6b26b}Low&\cellcolor[HTML]{57bb8a}V High&\circlepresent{}&\circlepartialpresent{}&\circlepresent{}\\
  
  A2&
  Low-level reference manual; \textit{i.e., a manual documenting all API components to review fine-grade detail.}&
  \scriptsize S1, S3, S4, S8, S9, S10, S11, S12, S15, S16, S17 &
  \cellcolor[HTML]{a7c47d}High&\cellcolor[HTML]{a7c47d}High&\circlepresent{}&\circlepresent{}&\circlepresent{}\\
  
  A3&
  Explanation of high level architecture; \textit{i.e., explanations of the API's high-level architecture to better understand intent and context.}
  &
  \scriptsize S1, S2, S4, S11, S14, S16, S19, S20 &
  \cellcolor[HTML]{ffd666}Med&\cellcolor[HTML]{57bb8a}V High&\circlepresent{}&\circlepresent{}&\circlepresent{}\\

  A4&
  Introspection source code comments; \textit{i.e., code implementation and code comments (where applicable) to understand the API author's mindset.}
  &
  \scriptsize S1, S4, S7, S12, S13, S17, S20 &
  \cellcolor[HTML]{ffd666}Med&\cellcolor[HTML]{a7c47d}High&\circlenotpresent{}&\circlenotpresent{}&\circlenotpresent{}\\

  {A5}&
  Code snippets of basic component function; \textit{i.e., {code snippets (with comments) of no more than 30 LoC to understand a basic component functionality within the API.}}
  &
  \scriptsize {S1, S2, S4, S5, S6, S7, S9, S10, S11, S14, S15, S16, S18, S20, S21} &
  \cellcolor[HTML]{57bb8a}V High&\cellcolor[HTML]{57bb8a}V High&\circlepresent{}&\circlepresent{}&\circlepresent{}\\

  {A6}&
  Step-by-step tutorials with multiple components; \textit{i.e., {step-by-step tutorials, with screenshots to understand  how to build a non-trivial piece of functionality with multiple components of the API.}}
  &
  \scriptsize {S1, S2, S4, S5, S7, S9, S10, S15, S16, S18, S20, S21} &
  \cellcolor[HTML]{57bb8a}V High&\cellcolor[HTML]{57bb8a}V High&\circlepartialpresent{}&\circlepresent{}&\circlepresent{}\\

  A7&
  Downloadable production-ready source code; \textit{i.e., downloadable source code of production-ready applications that use the API to understand implementation in a large-scale solution.}
  &
  \scriptsize S1, S2, S5, S9, S15 &
  \cellcolor[HTML]{f6b26b}Low&\cellcolor[HTML]{57bb8a}V High&\circlepartialpresent{}&\circlepartialpresent{}&\circlepresent{}\\

  A8&
  Best-practices of implementation; \textit{i.e., best-practices of implementation to assist with debugging and efficient use of the API.}
  &
  \scriptsize S1, S2, S4, S5, S7, S8, S9, S14 &
  \cellcolor[HTML]{ffd666}Med&\cellcolor[HTML]{57bb8a}V High&\circlenotpresent{}&\circlepresent{}&\circlepartialpresent{}\\

  A9&
  An exhaustive list of all components; \textit{i.e., a list of all the major components that exist within the API.}
  &
  \scriptsize S4, S16, S19 &
  \cellcolor[HTML]{f6b26b}Low&\cellcolor[HTML]{57bb8a}V High&\circlenotpresent{}&\circlepresent{}&\circlepresent{}\\

  A10&
  Minimum system requirements to use the API; \textit{i.e., requirements and the dependencies to use the API on a particular system.}
  &
  \scriptsize S4, S7, S13, S17, S19 &
  \cellcolor[HTML]{f6b26b}Low&\cellcolor[HTML]{57bb8a}V High&\circlepartialpresent{}&\circlenotpresent{}&\circlepartialpresent{}\\
  
  A11&
  Instructions to install/update the API and its release cycle; \textit{i.e., instructions to install or begin using the API and details on its release cycle and how to update it.}
  &
  \scriptsize S4, S7, S8, S9, S11, S13, S16, S19 &
  \cellcolor[HTML]{ffd666}Med&\cellcolor[HTML]{57bb8a}V High&\circlepartialpresent{}&\circlepartialpresent{}&\circlenotpresent{}\\

  A12&
  Error definitions describing how to address problems
  &
  \scriptsize S1, S2, S4, S5, S9, S11, S13 &
  \cellcolor[HTML]{ffd666}Med&\cellcolor[HTML]{57bb8a}V High&\circlepartialpresent{}&\circlenotpresent{}&\circlenotpresent{}\\

  \midrule
  \textbf{B}&
  \multicolumn{7}{l}{\textbf{Requirement 2: API Documentation should include Descriptions of the API's Design Rationale}}\\
  \midrule
  
  {B1}&
  Entry-point purpose of the API; \textit{i.e., {a brief description of the purpose or overview of the API as a low barrier to entry.}}
  &
  \scriptsize {S1, S2, S4, S5, S6, S8, S10, S11, S15, S16} &
  \cellcolor[HTML]{a7c47d}High&\cellcolor[HTML]{57bb8a}V High&\circlepresent{}&\circlepresent{}&\circlepresent{}\\

  B2&
  What the API can develop; \textit{i.e., descriptions of concrete types of applications the API can develop.}
  &
  \scriptsize S2, S4, S9, S11, S15, S18 &
  \cellcolor[HTML]{ffd666}Med&\cellcolor[HTML]{57bb8a}V High&\circlepartialpresent{}&\circlepartialpresent{}&\circlepresent{}\\

  B3&
  Who should use the API; \textit{i.e., descriptions of the types of users who should use the API.}
  &
  \scriptsize S4, S9 &
  \cellcolor[HTML]{e67c73}V Low&\cellcolor[HTML]{a7c47d}High&\circlepartialpresent{}&\circlenotpresent{}&\circlenotpresent{}\\


  B4&
  Who will use the applications built using the API; \textit{i.e., descriptions of the types of users who will use the product the API creates.}
  &
  \scriptsize S4 &
      \cellcolor[HTML]{e67c73}V Low&\cellcolor[HTML]{ffd666}Med&\circlenotpresent{}&\circlenotpresent{}&\circlenotpresent{}\\


  B5&
  Success stories on the API; \textit{i.e., example success stories of major users that describe how well the API was used in production.}
  &
  \scriptsize S4 &
  \cellcolor[HTML]{e67c73}V Low&\cellcolor[HTML]{57bb8a}V High&\circlepartialpresent{}&\circlepresent{}&\circlepresent{}\\


  B6&
  Documentation comparing similar APIs to this API
  &
  \scriptsize S2, S6, S13, S18 &
  \cellcolor[HTML]{f6b26b}Low&\cellcolor[HTML]{a7c47d}High&\circlepartialpresent{}&\circlenotpresent{}&\circlepresent{}\\


  B7&
  Limitations on what the API can/cannot provide
  &
  \scriptsize S4, S5, S8, S9, S14, S16 &
  \cellcolor[HTML]{ffd666}Med&\cellcolor[HTML]{57bb8a}V High&\circlenotpresent{}&\circlepresent{}&\circlepresent{}\\

  \midrule
  \textbf{C}&
  \multicolumn{7}{l}{\textbf{Requirement 3: API Documentation should include Descriptions of the Domain Concepts behind the API}}\\
  \midrule
  
  C1&
  Relationship between API components and domain concepts
  &
  \scriptsize S3, S10 &
  \cellcolor[HTML]{e67c73}V Low&\cellcolor[HTML]{a7c47d}High&\circlenotpresent{}&\circlenotpresent{}&\circlepresent{}\\

  C2&
  Definitions of domain terminology; \textit{i.e., definitions of the domain-terminology and concepts, with synonyms if applicable.}
  &
  \scriptsize S2, S3, S4, S6, S7, S10, S14, S16 &
  \cellcolor[HTML]{ffd666}Med&\cellcolor[HTML]{57bb8a}V High&\circlepartialpresent{}&\circlenotpresent{}&\circlepartialpresent{}\\

  C3&
  Documentation for nontechnical audiences; \textit{i.e., generalised documentation for non-technical audiences regarding the API and its domain.}
  &
  \scriptsize S4, S8, S16 &
  \cellcolor[HTML]{f6b26b}Low&\cellcolor[HTML]{a7c47d}High&\circlepresent{}&\circlepresent{}&\circlepresent{}\\

  \midrule
  \textbf{D}&
  \multicolumn{7}{l}{\textbf{Requirement 4: API Documentation should include Additional Support Artefacts to aide Developer Productivity}}\\
  \midrule
  
  D1&
  FAQs
  &
  \scriptsize S4, S7 &
  \cellcolor[HTML]{e67c73}V Low&\cellcolor[HTML]{57bb8a}V High&\circlepresent{}&\circlepresent{}&\circlepresent{}\\

  D2&
  Troubleshooting hints
  &
  \scriptsize S4, S8 &
  \cellcolor[HTML]{e67c73}V Low&\cellcolor[HTML]{a7c47d}High&\circlenotpresent{}&\circlepartialpresent{}&\circlenotpresent{}\\

  D3&
  API diagrams; \textit{i.e., diagrammatically representing API components using visual architectural representations.}
  &
  \scriptsize S6, S13, S20 &
  \cellcolor[HTML]{f6b26b}Low&\cellcolor[HTML]{57bb8a}V High&\circlenotpresent{}&\circlenotpresent{}&\circlenotpresent{}\\

  D4&
  Contact for technical support
  &
  \scriptsize S4, S8, S19 &
  \cellcolor[HTML]{f6b26b}Low&\cellcolor[HTML]{ffd666}Med&\circlepresent{}&\circlepresent{}&\circlepresent{}\\

  D5&
  Printed guide
  &
  \scriptsize S4, S6, S7, S9, S16 &
  \cellcolor[HTML]{f6b26b}Low&\cellcolor[HTML]{57bb8a}V High&\circlenotpresent{}&\circlepresent{}&\circlepresent{}\\

  D6&
  Licensing information
  &
  \scriptsize S7 &
  \cellcolor[HTML]{e67c73}V Low&\cellcolor[HTML]{57bb8a}V High&\circlenotpresent{}&\circlenotpresent{}&\circlepartialpresent{}\\

  \midrule
  \textbf{E}&
  \multicolumn{7}{l}{\textbf{Requirement 5: API Documentation should be Presented in an Easily Digestible Format}}\\
  \midrule
  
  E1&
  Searchable knowledge base
  &
  \scriptsize S3, S4, S6, S10, S14, S17, S18 &
  \cellcolor[HTML]{ffd666}Med&\cellcolor[HTML]{57bb8a}V High&\circlepresent{}&\circlepresent{}&\circlepresent{}\\

  E2&
  Context-specific discussion forums
  &
  \scriptsize S4, S10, S11 &
  \cellcolor[HTML]{f6b26b}Low&\cellcolor[HTML]{57bb8a}V High&\circlepresent{}&\circlepresent{}&\circlepartialpresent{}\\

  E3&
  Quick-links to other relevant components
  &
  \scriptsize S6, S16, S20 &
  \cellcolor[HTML]{f6b26b}Low&\cellcolor[HTML]{57bb8a}V High&\circlenotpresent{}&\circlenotpresent{}&\circlenotpresent{}\\

  E4&
  Structured navigation style; \textit{i.e., breadcrumbs}
  &
  \scriptsize S6, S10, S20 &
  \cellcolor[HTML]{f6b26b}Low&\cellcolor[HTML]{a7c47d}High&\circlepresent{}&\circlepresent{}&\circlepresent{}\\

  E5&
  Visualised map of navigational paths; \textit{i.e., to certain API components in the website.}
  &
  \scriptsize S6, S14, S20 &
  \cellcolor[HTML]{f6b26b}Low&\cellcolor[HTML]{57bb8a}V High&\circlenotpresent{}&\circlenotpresent{}&\circlenotpresent{}\\

  {E6}&
  Consistent look and feel
  &
  \scriptsize {S1, S2, S3, S5, S6, S8, S10, S15, S20} &
  \cellcolor[HTML]{a7c47d}High&\cellcolor[HTML]{57bb8a}V High&\circlepresent{}&\circlepresent{}&\circlepresent{}\\
\end{longtable}
\normalsize}
\end{landscape}

\clearpage
\section{Sources of Documentation}\label[appendixsec]{tse2020:tab:docsources}

Sources of documentation used for the validation of the taxonomy. For clarity, exact webpages are not referenced for each category, but can be found in supplementary materials which can be downloaded from the URL listed in the paper.
\bigskip

{\scriptsize
\begin{longtable}{p{.2\linewidth}|p{.725\linewidth}}
\toprule
  \textbf{Service} & \textbf{Document Sources}\\
  \midrule
  \endfirsthead
  \toprule
  \textbf{Service} & \textbf{Document Sources}\\
  \midrule
  \endhead
  \bottomrule
  \multicolumn{2}{r}{\textit{Continued on next page...}}\\
  \endfoot
  \bottomrule
  \endlastfoot
    Google Cloud Vision &
    \vspace{-1.75mm}
    \begin{itemize}[label=,leftmargin=10pt,topsep=0pt,partopsep=0pt,noitemsep,nolistsep,itemindent=-10pt]
\item \url{https://cloud.google.com/vision/docs/quickstart-client-libraries}
\item \url{https://googleapis.github.io/google-cloud-java/google-cloud-clients/apidocs/index.html}
\item \url{https://cloud.google.com/vision/#cloud-vision-use-cases}
\item \url{https://cloud.google.com/vision/docs/quickstart-client-libraries#using_the_client_library}
\item \url{https://cloud.google.com/vision/docs/tutorials}
\item \url{https://cloud.google.com/community/tutorials?q=vision}
\item \url{https://cloud.google.com/vision/docs/samples#mobile_platform_examples}
\item \url{https://cloud.google.com/docs/enterprise/best-practices-for-enterprise-organizations}
\item \url{https://cloud.google.com/functions/docs/bestpractices/tips}
\item \url{https://cloud.google.com/vision/#derive-insight-from-images-with-our-powerful-cloud-vision-api}
\item \url{https://cloud.google.com/vision/docs/quickstart-client-libraries}
\item \url{https://cloud.google.com/vision/docs/release-notes}
\item \url{https://cloud.google.com/vision/docs/reference/rpc/google.rpc#google.rpc.Code}
\item \url{https://cloud.google.com/vision/#insight-from-your-images}
\item \url{https://developers.google.com/machine-learning/glossary/}
\item \url{https://cloud.google.com/vision/docs/resources}
\item \url{https://cloud.google.com/vision/sla}
\item \url{https://cloud.google.com/vision/docs/data-usage}
\item \url{https://cloud.google.com/vision/docs/support#searchbox}
\item \url{https://cloud.google.com/vision/docs/support}
    \end{itemize}\\
    Amazon Rekgonition &
    \vspace{-1.75mm}
    \begin{itemize}[label=,leftmargin=10pt,topsep=0pt,partopsep=0pt,noitemsep,nolistsep,itemindent=-10pt]
\item \url{https://docs.aws.amazon.com/rekognition/latest/dg/getting-started.html}
\item \url{https://docs.aws.amazon.com/AWSJavaSDK/latest/javadoc/index.html}
\item \url{https://aws.amazon.com/blogs/machine-learning/using-amazon-rekognition-to-identify-persons-of-interest-for-law-enforcement/}
\item \url{https://aws.amazon.com/rekognition/#Rekognition_Image_Use_Cases}
\item \url{https://docs.aws.amazon.com/rekognition/latest/dg/labels-detect-labels-image.html}
\item \url{https://aws.amazon.com/rekognition/getting-started/#Tutorials}
\item \url{https://aws.amazon.com/blogs/machine-learning/category/artificial-intelligence/amazon-rekognition/}
\item \url{https://docs.aws.amazon.com/code-samples/latest/catalog/code-catalog-java-example_code-rekognition.html}
\item \url{https://docs.aws.amazon.com/rekognition/latest/dg/best-practices.html}
\item \url{https://docs.aws.amazon.com/rekognition/latest/dg/API_Operations.html}
\item \url{https://aws.amazon.com/rekognition/image-features/}
\item \url{https://aws.amazon.com/releasenotes/?tag=releasenotes\%23keywords\%23amazon-rekognition}
\item \url{https://docs.aws.amazon.com/rekognition/latest/dg/setting-up.html}
\item \url{https://aws.amazon.com/rekognition/}
\item \url{https://aws.amazon.com/rekognition/}
\item \url{https://docs.aws.amazon.com/rekognition/latest/dg/limits.html}
\item \url{https://aws.amazon.com/rekognition/pricing/}
\item \url{https://aws.amazon.com/rekognition/sla/}
\item \url{https://aws.amazon.com/rekognition/faqs/}
\item \url{https://docs.aws.amazon.com/rekognition/latest/dg/video-troubleshooting.html}
\item \url{https://docs.aws.amazon.com/rekognition/latest/dg/rekognition-dg.pdf}
\item \url{https://github.com/awsdocs/amazon-rekognition-developer-guide/issues}
\item \url{https://forums.aws.amazon.com/thread.jspa?threadID=285910}
    \end{itemize}\\
    Azure Computer\newline Vision &
    \vspace{-1.75mm}
    \begin{itemize}[label=,leftmargin=10pt,topsep=0pt,partopsep=0pt,noitemsep,nolistsep,itemindent=-10pt]
\item \url{https://docs.microsoft.com/en-au/azure/cognitive-services/computer-vision/quickstarts-sdk/csharp-analyze-sdk}
\item \url{https://docs.microsoft.com/en-us/java/api/overview/azure/cognitiveservices/client/computervision?view=azure-java-stable}
\item \url{https://docs.microsoft.com/en-us/azure/architecture/example-scenario/ai/intelligent-apps-image-processing}
\item \url{https://docs.microsoft.com/en-us/azure/cognitive-services/computer-vision/tutorials/java-tutorial}
\item \url{https://docs.microsoft.com/en-us/azure/cognitive-services/custom-vision-service/logo-detector-mobile}
\item \url{https://docs.microsoft.com/en-au/azure/cognitive-services/computer-vision/tutorials/storage-lab-tutorial}
\item \url{https://docs.microsoft.com/en-us/azure/cognitive-services/computer-vision/tutorials/csharptutorial}
\item \url{https://docs.microsoft.com/en-us/azure/cognitive-services/custom-vision-service/getting-started-improving-your-classifier}
\item \url{https://docs.microsoft.com/en-au/azure/cognitive-services/computer-vision/home#analyze-images-for-insight}
\item \url{https://docs.microsoft.com/en-au/azure/cognitive-services/computer-vision/vision-api-how-to-topics/howtocallvisionapi}
\item \url{https://docs.microsoft.com/en-us/azure/cognitive-services/custom-vision-service/release-notes}
\item \url{https://docs.microsoft.com/en-au/azure/cognitive-services/computer-vision/}
\item \url{https://azure.microsoft.com/en-au/services/cognitive-services/computer-vision/}
\item \url{https://azure.microsoft.com/en-us/pricing/details/cognitive-services/computer-vision/}
\item \url{https://docs.microsoft.com/en-au/azure/cognitive-services/computer-vision/concept-tagging-images}
\item \url{https://docs.microsoft.com/en-au/azure/cognitive-services/computer-vision/home}
\item \url{https://azure.microsoft.com/en-us/support/legal/sla/cognitive-services/v1_1/}
\item \url{https://docs.microsoft.com/en-au/azure/cognitive-services/computer-vision/faq}
\item \url{https://azure.microsoft.com/en-us/support/legal/}
    \end{itemize}\\
\end{longtable}}

%\clearpage
%\section{List of Online Artefacts}\label{tse2020:sec:online-artefacts}
%\bibliographystyleW{model1-num-names}
%\bibliographyW{webservices,webdocimprovements}

\clearpage
\section{List of Primary Sources}\label[appendixsec]{tse2020:sec:primary-sources}

The following pages list of the primary sources found from our systematic mapping study. Each citation is referenced by a prefixed `S'. We also list the respective citation count, as measured by the number of citations the publication has from Google Scholar as at July 2020. We also list the venue ranking (as at 2020), as measured by Scimago Rankings or Qualis Ranking for Journals and CORE Rankings for conference publications. If no rank can be found, a dash is used.

\begin{landscape}
\small
\begin{longtable}{p{0.03\linewidth}p{.79\linewidth}|cc}
  \toprule
  \textbf{Ref} & \textbf{Citation} & \textbf{Cite\#} & \textbf{Rank}\\
  \midrule
  \endfirsthead
  \toprule
  \textbf{Ref} & \textbf{Citation} & \textbf{Cite\#} & \textbf{Rank}\\
  \midrule
  \endhead
  \bottomrule
  \multicolumn{4}{r}{\textit{Continued on next page...}}\\
  \endfoot
  \bottomrule
  \endlastfoot
  
\relax[S1]&\bibentry{Robillard:2009uk}&305&Q1\\
\relax[S2]&\bibentry{Robillard:2011uv}&254&Q1\\
\relax[S3]&\bibentry{Ko:2011fb}&33&A\\
\relax[S4]&\bibentry{Nykaza:2002td}&56&--\\
\relax[S5]&\bibentry{Watson:2013fx}&14&B1\\
\relax[S6]&\bibentry{Jeong:2009tu}&34&--\\
\relax[S7]&\bibentry{Aghajani:2019bo}&6&A*\\
\relax[S8]&\bibentry{Haselbock:2018jd}&2&C\\
\relax[S9]&\bibentry{Inzunza:2018dn}&3&C\\
\relax[S10]&\bibentry{Meng:2017cx}&12&Q1\\
\relax[S11]&\bibentry{Geiger:2018fv}&4&Q1\\
\relax[S12]&\bibentry{Head:2018baa}&4&A*\\
\relax[S13]&\bibentry{Aversano:2017ic}&4&--\\
\relax[S14]&\bibentry{Robillard:hk}&55&A*\\
\relax[S15]&\bibentry{Watson:2012uy}&10&B1\\
\relax[S16]&\bibentry{Maalej2013}&110&Q1\\
\relax[S17]&\bibentry{Parnas:2007fb}&2&B\\
\relax[S18]&\bibentry{Bottomley:2005fs}&0&--\\
\relax[S19]&\bibentry{Taulavuori:2004el}&40&Q1\\
\relax[S20]&\bibentry{Kotula:1998wp}&27&Q1\\
\relax[S21]&\bibentry{McLellan:1998vu}&105&Q1\\
  
\end{longtable}
\end{landscape}
\normalsize

\clearpage
\section{Detailed Suggested Improvements}\label[appendixsec]{tse2020:sec:suggested-improvements}

For this assessment, we select the ILS or IPS values for categories that are considered either somewhat or very helpful (i.e., a score greater than 0.50). We then match these against categories that are found to be partially or not present within each service. In total, we found 12 categories where improvements can be made across all dimensions except \dime{}, detailed below .

\subsection[Dimension A Issues]{Issues regarding \dima{}}

\noindent
\textbf{Quick-start guides \dimcat{A1}:} 
Quick-start guides should provide a short tutorial that allows programmers to pick up the basics of an API in a programming language of their choice. For the services assessed, each offer various client SDKs (e.g., as Java or Python client libraries). Google Cloud Vision and Azure Computer Vision offer quick-start guides \citepweb{Quicksta92:online,Quicksta70:online} in which sets of articles target various SDKs or are client-agnostic with code snippets that can be changed to the client language/SDK of the developer's choice. Amazon Rekognition offers exercises in setting up the AWS SDK and using the command-line interface to interact with image analysis components \citepweb{Exercise86:online}, however this is client-agnostic nor does it provide details in how to get started with using the client SDKs.

\begin{leftbar}
\SuggestedImprovement
Ensure tutorials detail \uline{all} client-libraries and how developers can produce a minimum working example using the service on their own computer using that client library. For each SDK offered, there should be details on how to install, authenticate and use a component using local data. For example, this may be as simple as using the service to determine if an image of a dog contains the label `dog'.
\end{leftbar}

\noindent
\textbf{Step-by-step tutorials \dimcat{A6}:}  Google Cloud Vision offers tutorials limited to one component. These do not sufficiently demonstrate how to combine \textit{multiple components} of the API together and how developers should integrate it with a different platform, which a good step-by-step tutorial should detail. The official AWS Machine Learning blog \citepweb{AmazonRe55:online} provides extensive tutorials (in some cases, with a suggested tutorial completion time of over an hour) that integrate multiple Amazon Rekognition components with other AWS components. Microsoft provide tutorials \citepweb{Tutorial42:online,Tutorial6:online,GitHubAz56:online} integrating multiple components within their service to mobile applications and the Azure platform. 

\begin{leftbar}
\SuggestedImprovement
Ensure tutorials combine \uline{multiple} components of the service together, are extensive, and require developers to spend a non-trivial amount of time to produce a basic application. For example, the tutorial may detail how to integrate the API into a smartphone application to achieve the following: (i) take a photo with the camera, (ii) detect if a person is within the image, (iii) analyse the visual features of the person.
\end{leftbar}

\noindent
\textbf{Downloadable production-ready applications \dimcat{A7}:} Microsoft provide a downloadable application \citepweb{SampleEx87:online} that explores many components of the Azure Computer Vision API. The application is thoroughly documented with and also provides guidance on how to structure the architecture design of the program. While Rekognition and Google Cloud Vision also provide downloadable source code, they are largely under-documented, do not combine multiple components of the API together, and only use god-classes to handle all requests to the API \citepweb{JavaSDKV25:online,SampleAp41:online}.

\begin{leftbar}
\SuggestedImprovement
Downloadable source code should be thoroughly documented, and should avoid the use of god-classes that demonstrate a single piece of the service's functionality. Ideally, the \uline{architecture} of a production-ready application should be demonstrated to developers.
\end{leftbar}

\noindent
\textbf{Understanding best-practices \dimcat{A8}:} Google Cloud provides best-practices for its platform in both general and enterprise contexts \citepweb{Bestprac23:online,TipsTric26:online}, but there is little advice provided to guide developers on how best to use Google Cloud \textit{Vision}. Microsoft provides guidance on improving results of custom vision classifiers \citepweb{Improvin23:online}, but no further details on non-custom vision classifiers are found. We found the most detailed best-practices to be provided by Amazon Rekognition \citepweb{BestPrac58:online}, which outlines more detailed strategies such as reducing data transfer by storing and referencing images on S3 Buckets or the attributes images should have in various scenarios (e.g., the angles of a person's face in facial recognition).

\begin{leftbar}
\SuggestedImprovement
Document best-practices for all major components of the computer vision service. Guide developers on the types of input data that produce the best results, advisable minimum image sizes and recommended file types, and suggest ways to overcome limitations that improve usage and cost efficiency. Provide guidance in more than one use case; give a range of scenarios that demonstrate different best practices for different domains.
\end{leftbar}

\noindent
\textbf{Exhaustive lists of all major API components \dimcat{A9}:} Amazon provides a two-fold feature list that describes both the key features of Rekognition at a high-level \citepweb{AmazonRe70:online} as well as a detailed, technical breakdown of each API operation provided within the service \citepweb{ActionsA39:online}. Microsoft also provide a list of high-level features that Azure Computer Vision can analyse \citepweb{WhatisCo90:online} which provides hyperlinks to detailed descriptions of each feature. Google's Cloud Vision API provides a partial breakdown of the types of services provided, however this list is not fully complete, nor are there hyperlinks to more detailed descriptions of each of the features \citepweb{VisionAI32:online}.

\begin{leftbar}
\SuggestedImprovement
Document key features that the computer vision classifier can perform at a high level. This should be easy to find from the service's landing page. Each feature should be described with reference to more detailed descriptions of the feature's exact API endpoint and required inputs, outputs and possible errors.
\end{leftbar}

\noindent
\textbf{Minimum system requirements and dependencies \dimcat{A10}:} Although there is no dedicated webpage for this on any of the services investigated, there are listed dependencies for the client libraries in Google's and Azure's quick-start guides \citepweb{Quicksta92:online,CalltheC0:online}. These may be embedded within the quick-start guide as developers are likely to encounter dependency issues when they first start using the API. We found it a challenge to discover similar documentation this in Amazon's documentation.

\begin{leftbar}
\SuggestedImprovement
Any system requirements and dependency issues should be well-highlighted within the documentation's quick-start guide; developers are likely to encounter these issues within the early stages of using an API, and it is highly relevant to provide solutions to these issues within the quick-starts.
\end{leftbar}

\noindent
\textbf{Installation and release cycle notes \dimcat{A11}:} It is imperative that developers know what has changed between releases and how frequently the releases are exported. We found release notes for Amazon Computer Vision, although they are only major releases and have not been updated since 2017 \citepweb{AWSRelea46:online} which does not account for evolution in the service's responses \citep{Cummaudo:2019icsme}. Google's and Microsoft's release notes are generally more frequently updated, therefore developers can get a sense of its release frequency \citepweb{ReleaseN91:online,ReleaseN4:online}. However, there are evolution issues that are not addressed. Installation instructions are detailed within Rekognition's developer guide, outlining how to sign up for an account, and install the AWS command-line interface \citepweb{Step1Set76:online}.

\begin{leftbar}
\SuggestedImprovement
Ensure release notes detail label evolution, including any new additional labels that may have been introduced within the service. Transparency around the changes made to the service should go beyond new features: document potential changes that may influence maintenance of a system using the computer vision service so that developers are aware of potential side-effects of upgrading to a newer release.
\end{leftbar}

\subsection[Dimension B Issues]{Issues regarding \dimb{}}

\noindent
\textbf{Limitations of the API \dimcat{B7}:} The most detailed limitations documented were found on Rekognition's dedicated limitations page \citepweb{Limitsin66:online} that outlines functional limitations such as the maximum number of faces or words that can be detected in an image, the size requirements of images, and file type information. For the other services, functional limitations are generally found within each endpoint's API documentation, instead of within a dedicated page.

\begin{leftbar}\SuggestedImprovement
Document all functional limitations in a dedicated page that outline the maximum and minimum input requirements the classifier can handle. Documentation of the types of labels the service can provide is also desired.  
\end{leftbar}


\subsection[Dimension C Issues]{Issues regarding \dimc{}}

\noindent
\textbf{Conceptual understanding of the API \dimcat{C1}:} Azure Computer Vision provides `concept' pages describing the high-level concepts behind computer vision and where these functions are implemented within the APIs (e.g., \citepweb{Contentt49:online}). We were unable to find similar conceptual documentation for the other services assessed.

\begin{leftbar}\SuggestedImprovement
Document the concepts behind computer vision; differentiate between foundational concepts such as object localisation, object recognition, facial localisation and facial analysis such that developers are able to make the distinction between them. Relate these concepts back to the API and provide references to where the APIs implement these concepts.
\end{leftbar}

\noindent
\textbf{Definitions of domain-specific terminology \dimcat{C2}:}  Terminologies relevant to machine learning concepts powering these computer vision services are well detailed within Google's machine learning glossary \citepweb{MachineL36:online}, however few examples matching computer vision are immediately relevant. While this page is linked from the original Google Cloud Vision documentation, it may be too technical for application developers to grasp. A slightly better example of this is \citepweb{WhatisCo90:online}, where developers can understand computer vision terms in lay terms.

\begin{leftbar}\SuggestedImprovement
Current computer vision services use a myriad of terminologies to refer to the same conceptual feature; for example, while Microsoft refers to object recognition as `image tagging', Google refers to this as `label detection'. If a consolidation of terms is not possible, then computer vision services should provide a glossary that provides synonyms for these terminologies so that developers can easily move between service providers without needing to relink terms back to concepts.
\end{leftbar}

\subsection[Dimension D Issues]{Issues regarding \dimd{}}

\textbf{Troubleshooting suggestions \dimcat{D2}:} The only troubleshooting tips found in our analysis were in Rekognition's video service \citepweb{Troubles2:online}. Further detailed instances of these troubleshooting tips could be expanded to non-video issues. For instance, if developers upload `noisy' images, how can they inform the system of a specific ontology to use or to focus on parts of the foreground of background of the image? These are suggestions which we have proposed in prior work \citep{Cummaudo:2019icsme} that do not seem to be documented.

\begin{leftbar}\SuggestedImprovement
Ensure troubleshooting tips provide advice for testing against different types of valid input images.   
\end{leftbar}

\noindent
\textbf{Diagrammatic overview of the API \dimcat{D3}:} None of the computer vision services provide any overview of the API in terms of the features and processing steps on how they should be use. For instance, pre-processing and post-processing of input and response data should be considered and an understanding of how this fits into the `flow' of an application highlighted. Moreover, no UML diagrams could be found.

\begin{leftbar}\SuggestedImprovement
Provide diagrams illustrating the service within context of use, such as how it can be integrated with other service features or how a specific API endpoint may be used within a client application. Consider integrating interactive UML diagrams so that developers can easily explore various aspects of the documentation in a visual perspective.
\end{leftbar}

\clearpage
\section{Survey Questions}\label[appendixsec]{tse2020:sec:survey}

\def\AgreementScale{{\footnotesize \textit{[Strongly agree, Somewhat agree, Neither agree nor disagree, Somewhat disagree, Strongly disagree]}\bigskip}}

\noindent
This section contains the exact text of the survey described in \cref{tse2020:sec:validation:survey}. Our instrument also included questions where answers were not included in the research reported in this article, e.g. questions 1 and 2 regarding consent and ensuring participants have had development experience. Images used within the survey have been removed.

\bigskip
\hrule\sffamily\small

\subsection*{Developer opinions towards the importance of web API documentation recommendations}\noindent
In this study, we are finding out how important recommendations of web API documentation are to developers. From this, we will improve AI-powered APIs. While there are screenshots of example APIs in the questions, think of an API that you have used based on \textbf{your own prior experience} when answering these questions.   Thanks for taking the time to answer these questions; it should only take you about \textbf{10--20 minutes} to complete. 

\subsubsection*{Attribution Notice}\noindent
Portions of this questionnaire are reproduced from work created and shared by Google and used according to terms described in the Creative Commons 3.0 Attribution License. 

\bigskip\hrule

\subsubsection*{Implementation-specific documentation of web APIs}\noindent
When answering these questions please answer with respect to \textbf{your own experience} in learning web APIs (if applicable). Any examples provided exist solely to help illustrate the statement. For each question, please nominate how much you agree with the following statements: \AgreementScale

\begin{enumerate}[label=Q3\alph*.,leftmargin=2\parindent]
\item I think quick-start guides with code that help me get started with an API’s client library are important.
 e.g., quick-start guides that show how to get started and interact with the API and its responses.
\item I don't find low-level documentation of all classes and methods particularly helpful.
 e.g., a generated online reference manual from Javadoc comments.  
\item I would imagine that explanations of the API's high-level architecture, context and rationale would be important to better understand how to consume the API.
  e.g., a graphic showing how the API could fit into the wider context of an application.  
\item If I want to understand why an API did something that I didn't expect, the source code comments generally don't help me.
  e.g., an example from the Lodash API that describes why set.add isn't directly returned.  
\item I find small code snippets with comments to demonstrate a single component's basic functionality within the API a useful way to learn.
  e.g., 10-30 lines of code to demonstrating various how-tos of a computer vision API.  
\item  I think it's cumbersome to read through step-by-step tutorials that show how to build something non-trivial with multiple components using the API. 
   e.g., a ten-step tutorial documenting how to combine face recognition, face analysis, scene description, and landmark detection API components to generate descriptions of photos.   
\item  I think it's useful to download source code of production-ready applications that demonstrate the use of multiple facets of the API. 
   e.g., a downloadable iOS app that demonstrates how to perform image analysis on an iPhone/iPad.
\item  I think official documentation describing the ‘best-practices’ of how to use the API to assist with debugging and efficiency is not helpful. 
   e.g., an article describing the correct ways of doing things, the best tools to use, and how to write well-performing code.   
\item  I believe an exhaustive list of all major components in the API without excessive detail would be useful when learning an API. 
   e.g., a computer vision web API might list object detection, object localisation, facial recognition, and facial comparison as its 4 components.   
\item  I believe minimum system requirements and/or dependencies to use the API do not always need to be part of official documentation. 
   e.g., I can find descriptions of how to get started with a Python environment for a cloud platform on community forums instead of the API's website.   
\item  I think instructions on how to install or access the API, update it, and the frequency of its release cycle is all useful information to know about. 
   e.g., a list showing the latest releases, what was added and how to update your application to make use of it.   
\item  Error codes describing specific problems with an API are not helpful. 
   e.g., a list of canonical HTTP error codes and how to interpret them.   
\end{enumerate}

\bigskip\hrule
\subsubsection*{Rationale-specific documentation of web APIs}\noindent
When answering these questions please answer with respect to \textbf{your own experience} in learning web APIs (if applicable). Any examples provided exist solely to help illustrate the statement. For each question, please nominate how much you agree with the following statements: \AgreementScale

\begin{enumerate}[label=Q4\alph*.,leftmargin=2\parindent]
\item I think that, as a starting point when beginning to learn about an API, I would like to read about descriptions of the API's purpose and overview. 
\item I don't find descriptions of the types of applications the API can develop helpful. 
\item I believe that descriptions of the types of developers who should and shouldn't use the API is important to know. 
\item I don't think that descriptions of the types of end-users who will use the product built using the API is important to know in advance. 
\item I think that if I read success stories about when the API was previously used in production, I would have a better indicator of how I could use that API. 
\item I think that documentation that compares an API to other, similar APIs confusing and not important. 
\item I believe it is important to know about what the limitations are on what the API can and cannot provide. 
\end{enumerate}

\subsubsection*{Conceptual-specific documentation of web APIs}\noindent
When answering these questions please answer with respect to \textbf{your own experience} in learning web APIs (if applicable). Any examples provided exist solely to help illustrate the statement. For each question, please nominate how much you agree with the following statements: \AgreementScale

\begin{enumerate}[label=Q5\alph*.,leftmargin=2\parindent]
\item I wouldn’t read through theory about the API's domain that relates theoretical concepts to API components and how both work together. 
\item I think it is important to know the definitions of the API’s domain-specific terminology and concepts (with synonyms where needed). 
   e.g., a computer vision API that uses machine learning should list machine learning concepts. 
\item It's not really important to document information about the API to non-technical audiences, such as managers and other stakeholders. 
   e.g., pricing information, uptime information, QoS metrics/SLAs etc.   
\end{enumerate}

\bigskip\hrule
\subsubsection*{General-support documentation of web APIs}\noindent
When answering these questions please answer with respect to \textbf{your own experience} in learning web APIs (if applicable). Any examples provided exist solely to help illustrate the statement. For each question, please nominate how much you agree with the following statements: \AgreementScale

\begin{enumerate}[label=Q6\alph*.,leftmargin=2\parindent]
\item  I find lists of Frequently Asked Questions (FAQs) helpful. 
\item  When something goes wrong, I don't read through troubleshooting suggestions for specific problems straight away as I like to solve it myself. 
\item  I like to see diagrammatic representations of an API's components using visual architectural visualisations. 
   e.g., UML class diagram, sequence diagram. 
\item  I wouldn't look for email addresses and/or phone number for technical support in an API's documentation. 
\item  I generally refer to a programmer's reference guide or textbook about the API when I need to. 
\item  I don't think it's important to read about the licensing information about the API. 
\end{enumerate}

\bigskip\hrule
\subsubsection*{The effect of structure and tooling on web API documentation}\noindent
When answering these questions please answer with respect to \textbf{your own experience} in learning web APIs (if applicable). Any examples provided exist solely to help illustrate the statement. For each question, please nominate how much you agree with the following statements: \AgreementScale

\begin{enumerate}[label=Q7\alph*.,leftmargin=2\parindent]
\item I would like to use a searchable knowledge base to find information.
\item I think a context-specific discussion forum between developers isn't very helpful as it just introduces noise.  
  e.g., issue trackers, Slack group. 
\item I think links to other similar documentation frequently viewed by other developers would be useful. 
   e.g., 'people who viewed this also viewed…' 
\item If I get lost within the API's documentation, a 'breadcrumbs'-style of navigation isn't very useful to me. 
\item A visualised map of navigational paths to common API components in the website would be useful to have. 
   e.g., a large and complex API for Enterprise Service-Oriented Architecture where I could click into various boxes to read about components and arrows to read about how they are related.   
\item I believe ensuring consistent look and feel of all documentation isn't necessary to a good API documentation. 
\end{enumerate}

\bigskip\hrule
\subsubsection*{Demographics}\noindent

\begin{enumerate}[label=Q8\alph*.,leftmargin=2\parindent]
  \item Are you, or do you aspire to be, a professional programmer? Or would you consider programming a hobby?\\\noindent \textit{\footnotesize[Professional, Hobbyist]}
  \item How many years have you been programming? \\\noindent\textit{\footnotesize
[1--5 years,
6--10 years,
11--15 years,
16--20 years,
21--30 years,
31--40 years,
41+ years]}
  \item In what type of role would you say your current job falls into? \\\noindent\textit{\footnotesize
[
Back-end developer,
Data or business analyst,
Data scientist or machine learning specialist,
Database administrator,
Designer,
Desktop or enterprise applications developer,
DevOps specialist,
Educator or academic researcher,
Embedded applications or devices developer,
Engineering manager,
Front-end developer,
Full-stack developer,
Game or graphics developer,
Marketing or sales professional,
Mobile developer,
Product manager,
QA or test developer,
Student,
System administration]}
  \item What level of seniority would you say this role falls into? \\\noindent\textit{\footnotesize
[Intern Role,
Graduate Role,
Junior Role,
Mid-Tier Role,
Senior Role,
Lead Role,
Principal Role,
Management,
N/A (e.g., I am a student),
Other]}

  \item What industry would you say you work in? \\\noindent\textit{\footnotesize
[Cloud-based solutions or services,
Consulting,
Data and analytics,
Financial technology or services,
Healthcare technology or services,
Information technology,
Media, advertising, publishing, or entertainment,
Other software development,
Retail or eCommerce,
Software as a service (SaaS) development,
Web development or design,
N/A (e.g., I am a student),
Other industry not listed here]}
\end{enumerate}
\bigskip\hrule\bigskip
\hspace{\fill}\textit{** End of Survey **}\hspace{\fill}